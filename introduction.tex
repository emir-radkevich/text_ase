\section{Введение}

В геноме человека выявляется множество однонуклеотидных герминативных вариантов (мутаций, возникающие в половых клетках и передаваемых по наследству), в том числе от 459 до 565 тысяч вариантов в регуляторных областях, включая промоторы, энхансеры, инсуляторы, сайты связывания транскрипционных факторов и нетранслируемые области, что может обусловливать межиндивидуальные различия в уровнях экспрессии генов, и от 10 до 12 тысяч миссенс-вариантов в белок-кодирующих участках \cite{1000_genomes_project_consortium_global_2015}. Среди последних у индивидов обнаруживается в среднем 24-30 известных патогенных вариантов, ассоциированных с развитием наследственных заболеваний \cite{1000_genomes_project_consortium_global_2015}, которые, однако, не приводят к их манифестации.

В большинстве случаев экспрессия гена осуществляется в равной степени с обеих гомологичных хромосом. Тем не менее, в последние годы появился ряд работ, результаты которых свидетельствуют о распространенности аллель-специфической экспрессии (АСЭ), то есть неравновесного уровня экспрессии аллелей, выявляемого у 0,5–30 \% генов \cite{gaur_research_2013, lappalainen_transcriptome_2013,li_detection_2013}, и о ее роли в этиологии некоторых наследственных заболеваний, включая наследственно обусловленные формы опухолей \cite{galiatsatos_familial_2006, chan_heritable_2006, buzby_allele-specific_2017}.

Так, ранее была показана связь между пониженной экспрессией одного из аллелей гена опухолевого супрессора \textit{APC} и предрасположенностью к развитию семейного аденоматозного полипоза \cite{galiatsatos_familial_2006}. Также, было показано, что аллель-специфическое метилирование промотора гена \textit{MSH2}, отвечающего за репарацию ошибочно спаренных нуклеотидов, может приводить к развитию синдрома Линча \cite{chan_heritable_2006}. В рассмотренных случаях АСЭ приводила к снижению уровня экспрессии генов, кодирующих опухолевые супрессоры. Помимо этого, АСЭ способна влиять на экспрессию аллелей, содержащих патогенные герминативные варианты. Например, показана роль АСЭ в развитии синдрома Ли-Фраумени у членов семьи, являвшихся носителями функционального герминативного варианта в гене \textit{TP53}, обусловливающего фосфорилирование p53 по серину 15, которое определяет ответ на генотоксический стресс \cite{buzby_allele-specific_2017}. Кроме того, АСЭ выявляют в различных спорадических опухолях. Так, Милани и соавторы идентифицировали в промоторах генов, нарушение экспрессии которых ассоциировано с развитием опухолей, цис-регуляторные полиморфизмы, способствующие аллель-специфическому связыванию транскрипционных факторов \cite{milani_allelic_2007}.

Гепатоцеллюлярная карцинома (ГК) является наиболее распространенной формой первичных опухолей печени \cite{farazi_hepatocellular_2006}. Несмотря на активно проводимые исследования профиля молекулярных нарушений при ГК \cite{cancer_genome_atlas_research_network._electronic_address:_wheelerbcm.edu_comprehensive_2017}, на сегодняшний день изучению герминативных вариантов, которые могут предрасполагать к развитию ГК или способствовать ее прогрессии, уделяется недостаточное внимание. Распространенность, паттерн и возможная роль АСЭ в гепатоканцерогенезе также остаются малоисследованными. ГК свойственна высокая гетерогенность соматических нарушений \cite{cancer_genome_atlas_research_network._electronic_address:_wheelerbcm.edu_comprehensive_2017}, не позволяющих исчерпывающим образом установить молекулярные механизмы, лежащие в основе ее формирования и прогрессии \cite{lawrence_discovery_2014}. Таким образом, выявление полиморфизмов, обусловливающих нарушения аллельной экспрессии при ГК, а также последующая экспериментальная проверка их роли в приобретении опухолевыми клетками более злокачественного фенотипа позволит оценить значимость АСЭ как дополнительного фактора опухолевой прогрессии.

Целью настоящей работы является оценка возможности выявления АСЭ в ГК на основе данных транскриптомного секвенирования и проверка гипотезы о возможном вкладе АСЭ в процесс гепатоканцерогенеза. Для этого планируется исследовать несколько групп генов, с высокой вероятностью способных оказывать влияние на злокачественную трансформацию клеток печени человека, а именно онкогенов, опухолевых супрессоров и печень-специфических генов, в том числе за счет повышенного уровня экспрессии аллелей, содержащих патогенные герминативные варианты.

Для достижения этой цели мы поставили перед собой следующие задачи:
\begin{enumerate}
	\item провести поиск АСЭ среди генов, кодирующих онкогены, опухолевые супрессоры и печень-специфические гены на основании данных транскриптомного секвенирования 45 парных образцов ГК и неопухолевой ткани печени;
	\item выявить потенциально патогенные однонуклеотидные полиморфизмы (SNP) в АСЭ генах;
	\item провести экспериментальную верификацию АСЭ отдельных вариантов независимыми методами для проверки корректности предсказаний о дисбалансе на основе выбранного нами алгоритма.
\end{enumerate}