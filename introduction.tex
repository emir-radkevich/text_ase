\section{Введение}

В геноме человека выявляется множество герминативных вариантов в регуляторных областях, что может обусловливать межиндивидуальные различия в уровнях экспрессии генов, и в белок-кодирующих участках. Среди последних даже у здоровых индивидов обнаруживаются описанные ранее и потенциально патогенные варианты, ассоциированные с развитием наследственных заболеваний, которые, однако, не привели к их манифестации.

В большинстве случаев экспрессия гена осуществляется в равной степени с обеих гомологичных хромосом. Тем не менее, в последние годы появился ряд работ, результаты которых свидетельствуют о роли аллель-специфической экспрессии (АСЭ), то есть неравновесного уровня экспрессии аллелей генов, в этиологии некоторых наследственных заболеваний, включая наследственно обусловленные формы опухолей.

Так, ранее была показана связь между пониженной экспрессией одного из аллелей гена \textit{APC} и предрасположенностью к развитию семейного аденоматозного полипоза \cite{galiatsatos_familial_2006}. Также, было показано, что аллель-специфическое метилирование промотора гена \textit{MSH2} может приводить к развитию синдрома Линча \cite{chan_heritable_2006}. В обоих случаях АСЭ была связана с измененным уровнем экспрессии гена. Помимо этого, АСЭ способна влиять на экспрессию аллелей, содержащих патогенные герминативные варианты. Например, показана роль АСЭ в развитии синдрома Ли-Фраумени у членов семьи, являвшихся носителями функциональной герминативной мутации в гене \textit{TP53}, обусловливающей фосфорилирование p53 \cite{buzby_allele-specific_2017}.

Несмотря на наличие сообщений о существовании АСЭ в спорадических опухолях, работ, исследующих распространенность, паттерн и роль АСЭ при гепатокарциноме (ГК), в том числе применительно к ее влиянию на экспрессию патогенных аллелей, до настоящего времени опубликовано не было. При этом ГК свойственна высокая гетерогенность соматических нарушений, не позволяющих исчерпывающим образом установить лежащие в основе ее формирования и развития молекулярные механизмы. Таким образом, выявление наиболее общих для ГК нарушений аллельной экспрессии и последующая экспериментальная проверка их роли в приобретении опухолевыми клетками более злокачественного фенотипа позволит оценить значимость АСЭ как дополнительного фактора опухолевой прогрессии.

Для достижения этой цели мы поставили перед собой следующие задачи:
\begin{enumerate}
	\item выявить АСЭ генов в транскриптомах 45 парных образцов ГК и неопухолевой ткани печени;
	\item найти потенциально патогенные однонуклеотидные полиморфизмы (SNV) в АСЭ генах;
	\item экспериментально подтвердить АСЭ генов, в том числе в целях проверки корректности предсказаний о дисбалансе на основе выбранного нами алгоритма.
\end{enumerate}