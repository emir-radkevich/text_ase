\subsection{Рецептор гепатоцитарного фактора роста, MET}

MET – рецептор гепатоцитарного фактора роста (HGF), принадлежащий к семейству тирозин-киназных белков \cite{park_sequence_1987}. MET играет важную роль в процессах пролиферации, морфогенеза клеток, эпителиально-мезенхимальном переходе, а также играет ключевую роль в эмбриогенезе \cite{trusolino_met_2010}.  Основным лигандом рецептора является HGF, однако с MET способны связываться киназа фокальной адгезии (FAK), $\upbeta$-катенин, VEGFR (рецептор фактора роста эндотелия сосудов), EGFR (рецептор эпидермального фактора роста), CD44, интегрин $\upalpha$6$\upbeta$4, активирующие канонический и неканонические сигнальные пути соответственно \cite{garcia-vilas_updates_2018}. Помимо лигандов MET может взаимодействовать с внутриклеточными адаптерными белками, способствующими трансдукции сигнала внутри клетки, -- SHC, GAB1 (GRB2-ассоциированный связывающий белок 1), GRB2 (белок 2, связанный с рецептором ростового фактора), а также белками из  семейств Ras, STAT \cite{trusolino_met_2010}. Кроме того, рецептор может взаимодействовать с  дезинтегрин- и металлопротеаза-подобной внеклеточной протеазой (ADAM), расщепляющей MET на внеклеточный и внутриклеточный фрагменты \cite{foveau_down-regulation_2009}.

\subsubsection{Структура MET}

Рецептор MET  состоит из следующих функциональных доменов \cite{trusolino_met_2010}:
\begin{enumerate}
	\item Внеклеточные домены:
	\begin{itemize}
		\item \textbf{SEMA-домен}, представленный $\upalpha$- и $\upbeta$-субъединицами, необходим для связывания с HGF;
		\item \textbf{PSI-домен}, принимающий участие в связывании с HGF;
		\item 4 \textbf{иммунноглобулин-подобных домена} (IPT);
	\end{itemize} 
	\item Внутриклеточные домены:
	\begin{itemize}
		\item \textbf{юкстамембранный домен} (ЮД) участвует в связывании с убиквитин-лигазой;
		\item \textbf{тирозин-киназный домен} -- каталитический регион, модулирующий киназную активность;
		\item \textbf{мультифункицональный домен связывания} (МДС) участвует в связывании с белками-передатчиками сигнала внутрь клетки.
	\end{itemize} 
\end{enumerate}

На рисунке \ref{figure:met} проиллюстрированно строение рецептора.

\begin{figure}[h]
	\centering
	\includegraphics[width=\textwidth]{pics/met.png}
	\caption{\textbf{Строение рецептора MET.}
	Рецептор представлен внеклеточной и внутриклеточной частями. Слева-направо (от N- к C-концу): к внеклеточной части относятся SEMA-домен (представлен $\upalpha$- и $\upbeta$-субдоменами), PSI-домен, 4 иммунноглобулин-подобных домена, к внутриклеточной -- юкстамембранный домен (обозначен как ЮД), тирозин-киназный домен (обозначен как ЮД), мультифункциональный домен связывания (обозначен как МДС). Пунктирной линией обозначена трансмембранная часть белка.}
	\label{figure:met}
\end{figure}

При связывании рецептора с HGF происходит димеризация MET, что приводит к транс-фосфорилированию по двум сайтам -- Tyr1234 и Tyr1235. Следующий шаг активации рецептора -- это автофосфорилирование по двум аминокислотным остаткам МДС -- Tyr1349 и Tyr1356. Фосфорилирование двух последних сайтов -- необходимый шаг, так как в таком состоянии MET способен связываться с адаптерными белками \cite{trusolino_met_2010}.

\subsubsection{Партнеры связывания MET}

С активированным рецептором MET способны взаимодействовать PI3K (фосфоинозитид-3-киназа) \cite{ponzetto_multifunctional_1994}, GRB2  \cite{ponzetto_multifunctional_1994}, SHP2 \cite{fixman_pathways_1996} и SHC \cite{fixman_pathways_1996}, фосфолипаза C (PLC) \cite{ponzetto_multifunctional_1994}, транскрипционные факторы семейства STAT (STAT3) \cite{zhang_requirement_2002}. MET также связывается с  GAB1, мультиадаптeрным белком, который после фосфорилирования рецептора MET обеспечивает дополнительные сайты связывания для SHC, PI3K, SHP2, PLC и p120-Ras-ГТФаза-активирующего белка (p120-Ras-GAP) \cite{trusolino_met_2010}. MET взаимодействует с GAB1 напрямую, через уникальный сайт длиной 13 аминокислотных остатков, или опосредовано через GRB2 \cite{schaeper_coupling_2000, lock_identification_2000}. MET также способен взаимодействовать с интегрином $\upalpha$6$\upbeta$4 – рецептор может фосфорилировать $\upbeta$4-субъединицу по трем сайтам, создавая таким образом дополнительные сайты связывания для SHC, PI3K и SHP2. В таком комплексе GRB2 может связаться с SHC \cite{trusolino_met_2010}.

Внеклеточный домен изоформы v6 гиалуронового рецептора CD44 (CD44v6) образует тройной комплекс с MET и HGF. Образование комплекса необходимо для активации рецептора. Внутриклеточная часть CD44v6 связывает цитоплазматический домен МЕТ с актиновыми микрофиламентами через GRB2 и промежуточные белки, что способствует МЕТ-индуцированной активации RAS при участии фактора обмена гуаниновых нуклеотидов SOS (son of sevenless) \cite{orian-rousseau_hepatocyte_2007}.

\subsubsection{Сигнальные пути, регулируемые MET}

MET-зависимые сигналы, передаваемые от рецептора, оказывают влияние на многие сигнальные пути, которые способны регулировать тирозин-киназные рецепторы \cite{gherardi_targeting_2012}. Рецептор может связываться не только с HGF, активируя канонический сигнальный путь, но и с VEGFR, EGFR, CD44, интегрином $\upalpha$6$\upbeta$4. С этими лигандами рецептор может связаться только в состоянии мономера, а при димеризации способен образовать комплекс с $\upbeta$-катенином и киназой фокальной адгезии (FAK). При связывании с лигандом, отличным от HGF, происходит активация неканонических сигнальных путей \cite{garcia-vilas_updates_2018}.

Каскады митоген-активируемой протеинкиназы (MAPK) состоят из трех подсемейств, каждое из которых включает три протеинкиназы, которые последовательно активируют друг друга. Проксимальные элементы каскада активируются прямо или косвенно с помощью малых ГТФаз Ras. ГТФазы, в свою очередь, активируются факторами обмена нуклеотидов (GEF), в том числе, в случае активации Ras рецептором MET, белками SOS, и инактивируются белками, активирующими ГТФазу (GAP) – например, p120, который ингибируется в результате активации MET. Терминальные эффекторы рассматриваемых MAPK каскадов включают в себя киназы ERK1/2,  JNK1/2/3 и белки p38$\upalpha$,$\upbeta$,$\upgamma$,$\updelta$. Они транслоцируются в ядро, где влияют на активность различных факторов транскрипции. Стимуляция ERK киназ может приводить к активации пролиферации, белков JNK и p38 – к увеличению уровня пролиферации, дифференцировки клеток, а также к индукции апоптоза  \cite{trusolino_met_2010}. Однако JNK и p38 могут ингибировать процесс клеточной пролиферации. Такой разнонаправленный эффект обусловлен интенсивностью и продолжительностью сигнала, а также тканевой принадлежностью клеток, в которых активен JNK и/или p38 MAPK сигнальный каскад \cite{wagner_signal_2009}.

PI3K представляет собой липид-киназу, которая связывается с МДС МЕТ и катализирует образование фосфатидилинозитол-трифосфата (PtdIns3P). Производство PtdIns3P создает сайт связывания для протеин-киназы B (AKT), ключевыми эффекторами которой являются фосфоинозитид-зависимая киназа PDK-1 и белковый комплекс mTORC2 \cite{garcia-vilas_updates_2018}. После компартментализации на внутренней стороне плазматической мембраны AKT активируется и фосфорилирует несколько субстратов, вовлеченных в пролиферацию клеток, выживание и регуляцию размера клеток \cite{trusolino_met_2010}.

Активированный MET фосфорилирует STAT3 по Y705, что приводит к гомодимеризации STAT3 по домену SH2 и транслокации димера в ядро, где он действует как фактор транскрипции. В итоге, ускоряется трансформация клеток, повышается уровень пролиферации, а также ускоряется формирование сосудов (тубулогенез), обусловленное активацией VEGFA \cite{trusolino_met_2010}.

В ответ на МЕТ-опосредованную активацию PI3K-зависимых путей, а также, вероятно, Src, активируется I$\upkappa$B-киназный комплекс (IKK), который фосфорилирует находящийся в комплексе с транскрипционным фактором NF-$\upkappa$B и секвестрирующий его в цитоплазме ингибитор I$\upkappa$B. В результате фосфорилирования I$\upkappa$B подвергается протеасомной деградации, что приводит к ядерной транслокации NF-$\upkappa$B, где последний регулирует транскрипцию генов, стимулирующих пролиферацию, выживание клеток, а также ускорение формирования сосудов \cite{trusolino_met_2010}.

В таблице 1 приведена краткая информация о сигнальных путях, которые регулирует рецептор MET, а также представлен механизм регуляции \cite{trusolino_met_2010, gherardi_targeting_2012}.

\subsubsection{Сигнальные пути, регулирующие работу MET}

Существует несколько процессов, тонко регулирующих уровень экспрессии MET на плазматической мембране: регулируемый протеолиз и захват рецептора внутрь клетки \cite{trusolino_met_2010}. 

MET может подвергаться последовательному протеолитическому расщеплению по двум сайтам – один во внеклеточной части (один из IPT-доменов), а другой – во внутриклеточной (юкстамембранный домен – ЮД). Первое расщепление (известное как выделение (shedding)), которое осуществляется с помощью ADAM, приводит к образованию растворимого внеклеточного фрагмента, который ингибирует активность полноразмерного рецептора. Второе расщепление выполняется $\upgamma$-секретазой и приводит к появлению внутриклеточного фрагмента, который разрушается по убиквитин-независимому механизму протеасомой \cite{foveau_down-regulation_2009}.

Кроме того, в литературе описана эндоцитоз-зависимая регуляция активности MET, которая инициируется ассоциацией MET с белком CBL, убиквитин-лигазой, которая связывает MET и адаптерные белки, необходимые для формирования эндосомы. При этом MET в составе эндосомы подвергается лизосомальной деградации \cite{petrelli_endophilin-cin85-cbl_2002}.

С другой стороны, описаны случаи, когда при эндоцитозе МЕТ сохраняет способность активировать по крайней мере некоторые свои мишени. Так, интернализованный MET поддерживает активацию ERK1/2 через MEK1. Трафик интернализованного MET, а также его способность активировать ERK при этом регулируются протеинкиназой C$\upepsilon$ (PKC$\upepsilon$), которая также необходима для локализации активированного ERK в сайтах фокальной адгезии, где ERK может способствовать клеточной миграции \cite{liu_hepatocyte_2002, kermorgant_pkc_2004}. Дальнейший трафик МЕТ в околоядерные мембранные компартменты опосредуется PKC$\upalpha$. В этих компартментах активность фосфотирозинфосфатаз, которые воздействуют на STAT3, является низкой. Это способствует эффективному фосфорилированию STAT3 с помощью MET и транслокации STAT3 в ядро \cite{carpenter_stat3_2014}.

\begin{table}[H]
	\renewcommand{\arraystretch}{1.4} %increased line spacing
	\caption{\textbf{Сигнальные пути, регулируемые рецептором MET.}}
	\label{table:cascades}
	\enspace
	\begin{adjustbox}{max width=\textwidth}
		\begin{tabu} to \textwidth {XXX}
			\hline
			\multicolumn{1}{c}{\textbf{Сигнальный путь}} & 
			\multicolumn{1}{c}{\textbf{Механизм регуляции}} &
			\multicolumn{1}{c}{\textbf{Результат регуляции}}\\
			\hline
			MAPK каскады & Активация белков p38, ERK и JNK & Увеличение уровня пролиферации, дифференцировки клеток; индукция апоптоза \\
			PI3K-AKT & Активация AKT на внутренней стороне плазматической мембране &  Увеличенный уровень пролиферации клеток, выживания, а также положительная регуляция размера клеток \\
			STAT & Димеризация STAT3 при связывании с активированным MET & Увеличенный уровень пролиферации, а также ускорение формирования сосудов (тубулогенез) \\
			I$\upkappa$B$\upalpha$ - NF-$\upkappa$B комплекс & Активация IKK, последующее фосфорилирование I$\upkappa$B и ядерная транслокация NF-$\upkappa$B & Увеличенный уровень пролиферации, выживания клеток, а также ускорение формирования сосудов \\	
			\hline
		\end{tabu}
	\end{adjustbox}
\end{table}

\subsubsection{Роль MET в гепатоканцерогенезе}

Во многих опухолях транскрипция МЕТ индуцируется гипоксией и провоспалительными цитокинами или проангиогенными факторами, которые присутствуют в строме опухолей. Активация рецептора MET способствует опухолевой прогрессии путем передачи пролиферативных сигналов, а также сигналов к миграции клеток \cite{pennacchietti_hypoxia_2003, bhowmick_stromal_2004}.

При внесении в МДС MET мутации, приводящей к преимущественному связыванию с GRB2, но не с p85 (таким образом PI3K сигнальный путь не активируется), наблюдалось повышение уровня пролиферации, однако инвазии и метастазирования in vivo не наблюдалось. При внесении мутации, приводившей к преимущественной активации PI3K, клетки характеризовались высокой подвижностью, но инвазии и метастазирования не наблюдалось \cite{giordano_point_1997, bardelli_concomitant_1999}. Нативный белок в равной степени способен связываться и с GRB2, и с p85, активируя различные сигнальные пути в клетке. Таким образом, для инициации MET-зависимого инвазивного роста в опухолевых клетках необходима комбинированная активация нескольких сигнальных путей \cite{trusolino_met_2010}. 

HGF, лиганд рецептора MET, был первоначально описан благодаря его митогенным свойствам, дальнейшие исследования выявили его способность ингибировать апоптоз. Показано, что HGF обеспечивает защиту печени от массивного повреждения тканей  за счет подавления вызванного белком FAS апоптоза гепатоцитов \cite{kosai_abrogation_1998}. Антиапоптотическая роль HGF была описана у мышей с нокаутом MET в гепатоцитах, которые приобретали гиперчувствительность к повреждению печени при стимуляции Fas его агонистом JoS. Такие мыши демонстрируют замедленную регенерацию печени и более склонны к развитию фиброза печени \cite{giebeler_c-met_2009}. Эти наблюдения доказывают важное значение системы HGF-MET для регенерации печени.

В трансформированных клетках нерегулируемая передача сигналов MET происходит в случае его гиперэкспрессии, а также в результате возникновения мутаций, приводящих к конституитивной активации рецептора или препятствующих его деградации \cite{tovar_met_2017}. Так, амплификация рецептора наблюдалась в 1-5 \% случаев ГК \cite{takeo_examination_2001, kondo_clinical_2013, wang_genomic_2013}. В одном из исследований анеуплоидия по 7 хромосоме, на которой расположен ген MET, выявлена в 37 \% процентах случаев ГК \cite{kondo_clinical_2013}. В целом ряде исследований описана гиперэкспрессия MET в ГК \cite{boix_c-met_1994, suzuki_expression_1994, selden_expression_1994,daveau_hepatocyte_2003, noguchi_gene_1996}. При этом в рамках одного из исследований была продемонстрирована пониженная экспрессия HGF в опухолевой ткани по сравнению с прилежащей неопухолевой тканью и случаи снижения уровня мРНК MET в опухолевой ткани наряду со случаями гиперэкспрессии \cite{selden_expression_1994}. Серия иммунногистохимических экспериментов также продемонстрировала неоднородный характер экспрессии MET: в части работ обнаружена гиперэкспрессия рецептора \cite{kiss_analysis_1997, tavian_u-pa_2000}, в части – пониженная экспрессия как MET, так и HGF \cite{kiss_analysis_1997}. Таким образом, в большинстве исследований было обнаружено снижение уровней HGF в ГК, что свидетельствует о том, что активация MET должна быть в значительной степени независимой от лиганда, например, из-за мутаций в рецепторе, участия в неканонических сигнальных путях, гиперэкспрессии.

На крысиных клетках гепатомы FaO показано, что стимуляция MET его лигандом HGF приводит к ингибированию пролиферации клеток. В то же время, стимуляция MET ростовым фактором HGF приводила к увеличению инвазивного потенциала клеток \cite{shiota_hepatocyte_1992}. На клетках гепатомы человека HepG2 показано, что анти-пролиферативная активность ростового фактора коррелирует с активацией киназ ERK, а также с активацией транскрипции ингибиторов циклин-зависимых киназ  p16INK4A и p21Cip1, что приводит к ингибированию клеточной пролиферации \cite{shirako_up-regulation_2008}. С другой стороны, ингибирование MET приводит к увеличению уровня пролиферации и способности клеток к инвазии \cite{salvi_vitro_2007}. Причина такого неоднородного клеточного ответа остается неясной. Возможно, это связано с существованием альтернативных лигандов рецептора MET, запускающих внутриклеточный сигналинг. 

Одним из таких лигандов является предшественник протромбина дес-гамма-карбоксипротромбин (DCP). На клетках гепатомы человека Hep3B и SK-Hep-1 было показано, что DCP связывается с MET и вызывает стимуляцию пролиферации за счет активации JAK1/STAT3 сигнального пути, в то время как активации RAF/MAPK и PI3K/AKT не происходит \cite{suzuki_-gamma-carboxy_2005}. Если предложенный механизм считать верным, становится понятным, почему клеточный ответ на связывание DCP-MET отличается от клеточного ответа на образование комплекса HGF-MET. Кроме DCP и HGF, предполагаемым индуктором активации рецептора MET является рецептор эпителиального фактора роста, EGFR. В клетках гепатомы, не экспрессирующих HGF, наблюдалось фосфорилирование MET, вызванное ассоциацией двух рецепторов \cite{jo_cross-talk_2000}.

Кроме описанных in vitro экспериментов, были предприняты попытки продемонстрировать роль HGF-MET системы на мышиных моделях. 

Эмбрионы мышей, не экспрессирующие HGF и MET, не могли полностью дифференцироваться во взрослый организм и погибали in utero. Печень таких мышей характеризуется малым размером и почти полным отсутствием паренхиматозных клеток \cite{schmidt_scatter_1995, uehara_placental_1995}. Результаты работ по изучению влияния гиперэкспрессии и инактивации HGF и MET противоречивы. В части работ по изучению с использованием нокаутных по гену MET мышей сообщается о повышении частоты возникновения ГК, инициированной N-нитрозодиэтиламином (DEN), и увеличении размеров опухоли \cite{marx-stoelting_hepatocarcinogenesis_2009, takami_loss_2007}. Это может быть связано с тем, что в норме MET выполняет функцию белка, поддерживающего гомеостаз печени, и при потере экспрессии рецептора клетка может приобретать опухолевый фенотип. С другой стороны, был продемонстрирован про-опухолевый эффект в мышах, конститутивно гиперэкспрессирующих MET в гепатоцитах \cite{tward_genomic_2005}.  

Результаты экспериментов по гиперэкспрессии и инъекции HGF, проведенные на мышах, так же неоднородны. Часть работ указывает на значительное увеличение размеров опухоли при воздействии HGF вне зависимости от способа инициации канцерогенеза (наличие или отсутствие DEN индуктора) \cite{yaono_hepatocyte_1995, ogasawara_hepatocyte_1998}, тогда как в рамках других исследований демонстрируется обратный эффект \cite{liu_hepatocyte_1995}. При трансфекции генетического конструкта HGF с различными промоторами в эмбрионы мышей выяснилось, что в мышах с экспрессирующимся HGF под металлотионеиновым промотором происходит образование ГК \cite{horiguchi_hepatocyte_2002, sakata_hepatocyte_1996}, в то время как образования опухоли не наблюдается при экспрессии HGF под альбуминовым промотором \cite{shiota_hepatocyte_1994}. Однако, разнородные результаты могут быть вызваны, по меньшей мере, двумя причинами: уровень экспрессии HGF, достигнутый в сыворотке крови с использованием промотора металлотионеина, был в 2–5 раз выше по сравнению с тем, который был получен с промотором альбумина; кроме того, в одном из исследований использовали кДНК HGF мыши вместо кДНК HGF человека \cite{sakata_hepatocyte_1996}, использованной в другой статье \cite{shiota_hepatocyte_1994}; известно, что перекрестное взаимодействие между человеческим HGF и мышиным MET не является достаточным для обеспечения оптимальной активности рецептора \cite{giordano_met_2014}. 


В таблице \ref{table:met_exp} содержится обобщенная информация об in vitro и in vivo экспериментах с мутантными HGF и MET.

\begin{table}[H]
	\renewcommand{\arraystretch}{1.4} %increased line spacing
	\caption{\textbf{\textit{In vitro} и \textit{in vivo} эксперименты с мутантными HGF и MET.}}
	\label{table:met_exp}
	\enspace
	\begin{adjustbox}{max width=\textwidth}
		\begin{tabu}{XX}
			\hline
			\multicolumn{1}{c}{\textbf{Эксперимент}} & 
			\multicolumn{1}{c}{\textbf{Результат эксперимента}} \\
			\hline
			
			Нокаут MET во взрослых мышах & Повышение склонности к фиброзу печени, замедленная регенерация печени \cite{giebeler_c-met_2009} \\
			Нокаут MET в мышиных эмбрионах & Неполная дифференциовка во взрослый организм, гибель \textit{in utero}, малые размеры печени \cite{schmidt_scatter_1995, uehara_placental_1995} \\
			Нокаут MET & Повышение частоты возникновения ГК, инициированной DEN, увеличение размеров опухоли \cite{marx-stoelting_hepatocarcinogenesis_2009, takami_loss_2007} \\
			Ингибирование при помощи антисмысловой РНК MET и РНК интерференции & Увеличение уровня пролиферации и инвазивного потенциала клеток \cite{salvi_vitro_2007} \\	
			Трансфекция MET & Увеличение уровня пролиферации и инвазивного потенциала клеток \cite{tward_genomic_2005} \\	
			Инъекция HGF & Значительное увеличение размеров опухоли \cite{yaono_hepatocyte_1995, ogasawara_hepatocyte_1998} \\
			Инъекция HGF & Уменьшение размеров опухоли \cite{liu_hepatocyte_1995} \\
			Трансфекция HGF с металлотиониновым промотором & Образование ГК \cite{horiguchi_hepatocyte_2002, sakata_hepatocyte_1996} \\
			Трансфекция HGF с альбуминовым промотором & ГК не образуется \cite{shiota_hepatocyte_1994} \\
			
			\multicolumn{2}{c}{\textit{In vitro} эксперименты} \\
			Трансфекция HGF с альбуминовым промотором в FaO крысиные клетки гепатомы & Ингибирование клеточной пролиферации и увеличение инвазивного потенциала клеток \cite{shiota_hepatocyte_1992} \\
			Трансфекция HGF в FaO крысиные клетки гепатомы & Ингибирование клеточной пролиферации \cite{shiota_hepatocyte_1992} \\
			Обработка HGF клеточной культуры гепатомы человека HepG2, экспрессирующей HGF & Ингибирование клеточной пролиферации \cite{shirako_up-regulation_2008} \\

			\hline
		\end{tabu}
	\end{adjustbox}
\end{table}

\subsubsection{Мутации MET}

В трансформированных клетках нерегулируемая передача сигналов MET, описанная выше, происходит в случае гиперэкспрессии, а также активации рецептора при возникновении мутаций \cite{tovar_met_2017}. Впервые активирующие мутации MET были описаны у пациентов с различными формами рака почки \cite{schmidt_novel_1999}. Все найденные мутации находятся в каталитическом домене недалеко от сайтов транс-фосфорилирования (Tyr1234 и Tyr1235). 

В нескольких экспериментах на опухолевых аллографтах и клеточных культурах было показано, что мутации в тирозин-киназном домене индуцируют конститутивную активацию рецептора \cite{jeffers_activating_1997}. На мышиных моделях показано, что некоторые варианты приводят к возникновению различных типов сарком, карцином и лимфом. Например, у мышей с мутацией, приводящей к аминокислотной замене M1248T в MET, развивались различные карциномы и лимфомы, а у мышей с мутациями, приводящими к заменам D1226 и Y1228C – саркомы и лимфомы \cite{graveel_activating_2004}. Поскольку единственными различиями между этими животными были мутации MET, эти результаты позволяют предположить, что либо сама структура мутированной киназы, либо уровень экспрессии рецептора (или оба фактора), влияют на тканеспецифичность образования опухоли. В целом, эти исследования продемонстрировали, что активирующие мутации влияют не только на активность МЕТ, но и на активность сигнальных путей в различных типах тканей. 

После обнаружения активирующих мутаций в тирозин-киназном домене у пациентов с различными формами опухолей почки были проведены скринингоые исследования, в результате которых удалось обнаружить новые мутации как в тирозин-киназном домене, так и в других доменах рецептора, в том числе в юкстамембранном домене.

Первые мутации в юкстамембранном домене были выявлены в образцах опухолей желудка (P1009S) \cite{lee_novel_2000} и молочной железы (T1010I) \cite{lee_novel_2000}. Замена T1010I была обнаружена также в образцах опухолевой ткани немелкоклеточного рака легкого \cite{tengs_transforming_2006}, наследственной форме колоректального рака \cite{neklason_activating_2011}, рака простаты \cite{sethi_comprehensive_2013}, хронического лимфоцитарного рака \cite{brown_systematic_2008}, мелкоклеточного рака легких  \cite{ma_c-met_2003}, паппилярного рака почек \cite{schmidt_novel_1999}, рака щитовидной железы \cite{wasenius_met_2005}, крупноклеточного рака легких \cite{lee_novel_2000}, рака яичников \cite{tang_met_2014}. 

На клеточной линии фибробластов NIH3T3 было показано, что MET T1010I не проявляет конститутивного фосфорилирования \cite{schmidt_novel_1999}, при этом длительность фосфорилирования у белка с заменой и нативного белка не отличалась \cite{lee_novel_2000}. Также было показано, что клетки NIH3T3, экспрессирующие мутантный белок, не образуют участков фокальной адгезии и не демонстрируют значимого повышения клоногенного потенциала по сравнению с клетками, гиперэкспрессирующими MET дикого типа. Подкожная инъекция таких клеток бестимусным мышам приводит лишь к незначительному ускорению развития опухолей по сравнению с контролем  \cite{lee_novel_2000}. 

В то же время, в клеточной культуре гемопоэтических клеток мыши Ba\textbackslash F3 продемонстрировано значительное увеличение уровня конститутивного фосфорилирования MET T1010I, а также незначительное увеличение уровня пролиферации и миграции клеток \cite{ma_c-met_2003}. Клетки эпителия молочной железы человека MCF-10A демонстрировали повышенный уровень миграции и инвазии при экспрессии MET T1010I. При этом незначительно повышался уровень фосфорилирования STAT3, AKT и MAPK. При подкожной инъекции клеток MCF-10A, экспрессирующих MET T1010I, трансгенным иммунодефицитным мышам, экспрессирующим hHGF, развивающиеся опухоли характеризуются более высоким уровнем инвазии по сравнению с опухолями, возникшими из клеток, гиперэкспрессирующих WT MET \cite{liu_functional_2015}. 

Несмотря на неоднозначость имеющихся в литературе данных, наличие замены T1010I в MET может оказатся одним из факторов, способствующих прогрессии ГК. Мутация возникает в юкстамембранном домене, ответственном за связывание с убиквитин-лигазой E3 \cite{trusolino_met_2010}. Известно, что замена MET Y1003F в сайте связывания MET с убиквитин-лигазой E3, находящегося в том же домене, что и T1010I, приводит к трансформации фибробластов и эпителиальных клеток – такие варианты рецепторов могут интернализоваться с образованием эндосом при взаимодействии с лигандом HGF, но не могут быть убиквитинилированы \cite{peschard_mutation_2001}. Учитывая близкое расположение T1010I к сайту связывания с убиквитин-лигазой E3, несинонимичная замена в позиции 1010 может привести к аналогичному эффекту и, тем самым способствовать развитию ГК.

