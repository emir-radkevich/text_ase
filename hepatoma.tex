\subsection{Гепатокарцинома}

Согласно данным GLOBOCAN за 2018 год, рак печени занимает 6 место в мире среди всех типов рака человека по количеству новых случаев и 4 – по количеству летальных исходов \cite{bray_global_2018} (рисунок \ref{fig:statistics}). Высокая смертность от рака печени отчасти связана с поздними сроками его выявления ввиду бессимптомного течения у части пациентов \cite{befeler_hepatocellular_2002}, устойчивостью к применяемым в терапии противоопухолевым препаратам, а также обусловленными наличием сопутствующих заболеваний печени ограничениями на выбор терапии. В случаях локализованного заболевания хирургическая резекция опухоли представляет собой эффективную терапевтическую альтернативу (особенно в условиях отсутствия цирроза печени). Несколько альтернативных методов лечения, таких как инъекция этанола или применение радиочастотной абляции, также используются в случаях локализованного заболевания. В редких случаях применим радикальный подход – пересадка печени, однако лишь в тех случаях, когда пациенты отвечают строго определенным критериям \cite{farazi_hepatocellular_2006}.

\begin{figure}[H]
\captionsetup[subfigure]{justification=centering}
\centering
	\begin{subfigure}{0.49\textwidth}
		\includegraphics[width=\linewidth]{pics/num_inc.png} 
%		\centering
		\caption{Количество летальных исходов от различных типов рака в 2018 году}
		\label{fig:inc}
	\end{subfigure}
	\begin{subfigure}{0.49\textwidth}
		\includegraphics[width=\linewidth]{pics/num_mort.png}
		\caption{Количество новых случаев различных типов рака в 2018 году}
		\label{fig:mort}
	\end{subfigure}
	\caption{\textbf{Статистика раковых заболеваний согласно данным GLOBOCAN 2018} \cite{bray_global_2018}. На рисунках \ref{fig:inc} и \ref{fig:mort} внутренний кружок отражает распределение по полам (розовый -- женщины, фиолетовый -- мужчины). Значения, изображенные на рисунках, приведены в процентах.}
	\label{fig:statistics}
\end{figure}

Рак печени включает в себя разнообразные гистологически отличимые первичные новообразования печени, которые включают гепатоцеллюлярную карциному (ГК), внутрипеченочную карциному желчного протока (холангиокарцинома), гепатобластому, цистаденокарциному желчного протока, гемангиосаркому и эпителиоэдиомеатому. Среди всех типов рака печени ГК является наиболее распространенной формой – она составляет примерно 80-85 \% всех случаев \cite{farazi_hepatocellular_2006}. 

\subsubsection{Этиология гепатокарциномы}

Несмотря на то, что ГК распространена по всему миру, существуют региональные различия в частоте и механизмах возникновения ГК, обусловленные специфическими этиологическими факторами. Наиболее значимыми факторами рискавозникновения ГК являются хроническая вирусная инфекция вирусами гепатита В или С, хроническое употребление алкоголя, употребление пищи, загрязненной афлатоксином B\textsubscript{1}, а также большинство заболеваний, вызывающих цирроз печени \cite{farazi_hepatocellular_2006}.

\paragraph{ГК, вызванная инфекцией вируса гепатита B}\mbox{}\\

Вирус гепатита B человека (HBV) – вирус с частично двухцепочечной молекулой ДНК, кодирующей несколько вирусных белков, необходимых для его жизненного цикла: обратная транскриптаза/ДНК-полимераза, капсидный белок, известный как коровый антиген гепатита B (HBsAg), белок X (HBx), и белки оболочки L, M и S, которые связываются с мембраной эндоплазматического ретикулума (ЭР) \cite{farazi_hepatocellular_2006}. 

В HBV-опосредованной опухолевой трансформации клетки задействовано несколько механизмов. ДНК HBV способна интегрировать в геном, при этом сайты интеграции преимущественно выявляются в участках ДНК человека, подверженных разрывам (в так называемых ломких сайтах хромосом) и в CpG-богатых участках. Кроме того, в геноме человека выделяют ряд генов, преимущественно представленных онкогенами и опухолевыми супрессорами, в том числе генов обратной транскриптазы теломеразы \textit{TERT}, \textit{MLL4}, \textit{CCNE1}, интеграция HBV в которые в опухолях, вероятно, ввиду получения клетками с измененным уровнем экспрессии этих генов селективных преимуществ, наблюдается с более высокой частотой по сравнению другими сайтами интеграции. Таким образом, интеграция генома HBV способна приводить к повышению геномной нестабильности, причем как к крупным перестройкам, так и к микроделециям в ДНК хозяина, возникающим в результате рекомбинации, и оказывать влияние на уровень экспрессии генов \cite{tokino_chromosome_1991, zhao_genomic_2016}. Во-вторых, белок HBx способен напрямую взаимодействовать с компонентами многих сигнальных каскадов, таких как Src тирозинкиназы, RAS/ERK, JNK, NF$\upkappa$B, Wnt/$\upbeta$-catenin \cite{feitelson_genetic_2002}. Также, HBx может связывать и инактивировать опухолевый супрессор p53, что способствует повышению пролиферации и выживаемости клеток \cite{feitelson_genetic_2002}. Наконец, HBx способен фосфорилировать и инактивировать опухолевый супрессор рRb, подавляя таким образом контроль над клеточным циклом и системы репарации ДНК \cite{arzumanyan_pathogenic_2013}.

При развитии вирусной инфекции иммунный ответ на постоянную репликацию вируса способствует развитию гепатита – происходит некроз ткани с последующей регенерацией и/или фиброзом печени. Во время регенерации ДНК вируса HBV все чаще интегрируются в ДНК хозяина, что приводит к дестабилизации генома и иным последствиям, в частности, к повышению продукции HBx. Вирус способствуют пролиферации и выживанию инфицированных клеток во время хронического заболевания печени и активирует несколько перекрывающихся сигнальных путей (например, RAS, PI3K, рецептор эпидермального фактора роста (EGFR) и рецептор инсулиноподобного фактора роста 1 (IGFR1)), которые способствуют злокачественной трансформации клеток \cite{arzumanyan_pathogenic_2013}. Другой предполагаемый механизм гепатоканцерогенеза, вызванного HBV,  связан с накоплением белков вируса в эндоплазматическом ретикулуме (ЭР), что провоцирует стресс ЭР \cite{choi_naturally_2019} и, в конечном счете, индукцию окислительного стресса, который может активировать пути передачи сигналов, стимулирующие пролиферацию и выживание клеток, возникновение мутаций за счет повышения уровня свободных радикалов и активировать звездчатые клетки печени. Таким образом, HBV может способствовать инициации ГК посредством множества механизмов \cite{farazi_hepatocellular_2006}.

\paragraph{ГК, вызванная инфекцией вируса гепатита C}\mbox{}\\

Вирус гепатита C человека (HCV) – вирус, геном которого представлен положительной цепью молекулы РНК, кодирующей неструктурные белки (NS2, NS3, NS4A, NS5A и NS5B), вирусную репликазу (коровый белок), ионный канал, а также гликопротеины E1 и E2 \cite{arzumanyan_pathogenic_2013}.

Патогенез ГК, ассоциированный с инфекцией HCV, имеет ряд отличительных особенностей по сравнению с процессом развития ГК на фоне инфекции HBV. Во-первых, HCV демонстрирует более высокую склонность к возникновению хронической инфекции – в хроническую стадию переходит около 10 \% случаев HBV против 60-80 \% HCV. Это может быть связано со способностью квази-видов HCV, геном которых содержит большое количество мутаций, обходить иммунную систему хозяина. Второе ключевое отличие заключается в более высокой частоте развития цирроза печени у пациентов с инфекцией HCV. Примерно у 10 \% пациентов с HCV после 10 лет инфицирования развивается цирроз печени, что примерно на порядок выше, чем при инфекции HBV. В-третьих, поскольку HCV является РНК-вирусом без промежуточной формы ДНК, он не может интегрироваться в геном хозяина \cite{farazi_hepatocellular_2006}.

Так же, как и в случае с инфекцией HBV, непрерывные циклы гибели гепатоцитов, вызванные иммунным ответом на HCV и последующей регенерацией, приводят к накоплению мутаций. Коровый белок HCV и неструктурный белок NS5A участвуют в процессе избегания иммуно-опосредованного уничтожения клеток путем взаимодействия с различными факторами, вовлеченными в этот процесс, – фактором некроза опухоли (TNF$\upalpha$), интерферонами (IFN$\upalpha$) \cite{farazi_hepatocellular_2006}. Кроме того, белки HCV NS3 и NS4A используют свою протеазную активность для расщепления и активации компонентов, которые участвуют в передаче сигнала иммунного ответа \cite{li_immune_2005}.

HCV, как и HBV, связывается с ЭР, вызывая тем самым ЭР стресс. Кроме того, было показано, что коровые белки HCV взаимодействуют с компонентами сигнального пути MAPK (такими как ERK, MEK и RAF) и, таким образом, способствуют пролиферации клеток \cite{macdonald_hepatitis_2003}. При этом при инфекции HCV наблюдается накопление жирных кислот и повышенное образование активных форм кислорода \cite{moriya_oxidative_2001}. Также было показано, что NS5A взаимодействует с p53 и инактивирует его путем секвестрирования в околоядерное пространство, влияя, таким образом, на сигнальные пути, регулируемые p53 \cite{farazi_hepatocellular_2006}. 

\paragraph{ГК, вызванная хроническим употреблением алкоголя}\mbox{}\\

Употребление алкоголя является важным фактором риска возникновения ГК. При хроническом употреблении алкоголя в результате активации моноцитов образуются провоспалительные цитокины \cite{mcclain_monocyte_2002}. В условиях хронического воздействия этанола гепатоциты проявляют повышенную чувствительность к цитотоксическим эффектам TNF$\upalpha$, что, с одной стороны, индуцирует гибель гепатоцитов, с другой - индуцирует регенерацию печени, активацию звездчатых клеток и продукцию ими компонентов внеклеточного матрикса и развитие цирроза, что, в конечном итоге, создает условия для развития ГК \cite{farazi_hepatocellular_2006}.

Алкоголь также повреждает печень путем индукции окислительного стресса, который способствует развитию ГК несколькими путями. Во-первых, окислительный стресс способствует развитию фиброза и цирроза. Про-канцерогенный эффект цирротического микроокружения был продемонстрирован на трансгенных мышах, гиперэкспрессирующих PDGF$\upbeta$, у которых развивался фиброз, способствующий возникновению ГК \cite{campbell_platelet-derived_2005}. Поскольку звездчатые клетки являются основным источником коллагена в поврежденной печени, индукция окислительного стресса культивируемых звездчатых клеток может способствовать увеличению пролиферации клеток и синтезу коллагена \cite{farazi_hepatocellular_2006}. Во-вторых, вызванный этанолом окислительный стресс может вызывать изменения сигнальных путей, связанных с возникновением ГК – снижение уровня фосфорилирования STAT1, снижение STAT1-направленной активации передачи сигналов IFN$\upalpha$ и потерю защитных эффектов IFN$\upalpha$ с последующим повреждением гепатоцитов. Окислительный стресс может также вызвать накопление онкогенных мутаций. Так, была описана ассоциация окислительного стресса при нарушении гомеостаза железа с повышением частоты мутаций в гене р53 при ГК \cite{marrogi_oxidative_2001}.

\paragraph{ГК, вызванная афлатоксином B\textsubscript{1}}\mbox{}\\

Употребление микотоксина -- афлатоксина B\textsubscript{1} -- также представляет повышенный риск развития ГК. Афлатоксин B\textsubscript{1}, по-видимому, функционирует как мутаген и связан со специфической мутацией р53, а также с активирующей мутацией в HRAS \cite{ozturk_p53_1991, riley_vitro_1997}. В отличие от ГК, вызванной HCV и алкоголем, четкой связи между воздействием афлатоксина B\textsubscript{1} и развитием цирроза не выявлено, что указывает на то, что основным фактором развития ГК в этом случае может быть мутагенное воздействие этого токсина. В некоторых регионах воздействие афлатоксина B\textsubscript{1} часто сочетается с инфекцией HBV, что повышает риск развития ГК в более раннем возрасте \cite{kew_synergistic_2003}.

\subsubsection{Генетические и эпигенетические события при ГК}

Молекулярный анализ 196 образцов ГК человека позволил выявить множество генетических и эпигенетических изменений в ключевых онкогенах и опухолевых супрессорах. К таким изменениям относятся точечные соматические мутации в часто мутирующих генах (\textit{\textit{TP53}}, \textit{CTNNB\textsubscript{1}} (кодирует $\upbeta$-катенин), промотор \textit{TERT}), изменение числа копий аллелей гена (\textit{CDKN2A}, \textit{MET}, \textit{RB\textsubscript{1}}), активация экспрессии генов при интеграции генома вируса гепатита B (\textit{TERT}), а также гиперметилирование промоторов (\textit{CDKN2A}). Кроме того, была обнаружена высокая частота мутаций в генах, кодирующих компоненты сигнальных путей WNT, SHH (Sonic Hedgehog), RTK/RAS \cite{cancer_genome_atlas_research_network._electronic_address:_wheelerbcm.edu_comprehensive_2017}.

Наиболее распространенными соматическими мутациями, обнаруженными в 44 \% исследуемых образцов, были мутации в промоторе \textit{TERT}. Пациенты с мутацией промотора \textit{TERT} были как правило старше (р=$0.0006$), преимущественно мужского пола (р=$0.006$), с большей вероятностью были инфицированы HCV (р=$0.04$) и реже были инфицированы HBV (р=$0.02$), чем пациенты без мутации. В исследуемых образцах наблюдался повышенный уровень экспрессии \textit{TERT}, однако корреляции между наличием мутации в промоторе гена и уровнем экспрессии выявлено не было \cite{cancer_genome_atlas_research_network._electronic_address:_wheelerbcm.edu_comprehensive_2017}.

Описана корреляция между наличием мутации в промоторе \textit{TERT} и инактивацией \textit{CDKN2A} за счет гиперметилирования промотора. \textit{CDKN2A} кодирует опухолевый супрессор p16\textsuperscript{INK4A}. Подавление экспрессии p16\textsuperscript{INK4A} в сочетании с повышенной экспрессией \textit{TERT} приводит к приобретению эпителиальными клетками опухолевого фенотипа, которое обусловлено повышением теломеразной активности и увеличением уровня пролиферации \cite{kiyono_both_1998}. У 31 \% пациентов были найдены мутации в \textit{TP53}. По-видимому, наличие мутаций не связано с низким уровнем экспрессии \textit{TP53} – только в одном образце с высокой экспрессией \textit{TP53} выявлена мутация в этом гене, в то время как в 23 \% образцов с низкой экспрессией \textit{TP53} ген не содержал ни одной мутации. Такой эффект, по-видимому, объясняется активностью ингибиторов p53, например MDM4, экспрессия которого была значительно повышена \cite{cancer_genome_atlas_research_network._electronic_address:_wheelerbcm.edu_comprehensive_2017}. С другой стороны, было показано, что большинство мутаций \textit{TP53} в опухолевых клетках являются миссенс-мутациями, затрагивающими примерно 190 кодонов, причем 8 наиболее частых вариантов составляют около 28 \% от всех встречающихся мутаций. 7 из 8 горячих точек мутаций приходятся на CpG динуклеотиды \cite{baugh_why_2018}.

В некоторых IDH1/2 мутантных опухолях выявлено снижение экспрессии микроРНК-122 (miR-122), которое может быть связано с гиперметилированием промотора этого гена \cite{cancer_genome_atlas_research_network._electronic_address:_wheelerbcm.edu_comprehensive_2017}. miR-122 является преобладающей в клетках печени, где выполняет функцию регулятора метаболизма жирных кислот \cite{liu_mir-122_2014}. Мишенями miR-122 являются белки, контролирующие адгезию к субстрату, клеточный рост, пролиферацию – дезинтегрин/металлопротеиназа домен-содержащие белки (ADAM10/ADAM17), рецептор инсулин-подобного ростового фактора (IGFR1) и циклин G1 (CCNG1).

Мутации в гене альбумина (\textit{ALB}) и гене аполипопротеина B (\textit{APOB}) выявлены в 13 \% и 10 \% ГК, соответственно. Кроме того, описано снижение уровня экспрессии \textit{ALB} и \textit{APOB} в ГК относительно нормальной ткани печени. Описана связь между снижением экспрессии этих генов и повышенным уровнем пролиферации и синтеза нуклеотидов \cite{cancer_genome_atlas_research_network._electronic_address:_wheelerbcm.edu_comprehensive_2017}.

Помимо характерных для ГК точечных соматических мутаций, изменений в копийности аллелей генов и в профиле метилирования, в литературе описаны эффекты сегрегации хромосом при ГК. Нарушение сегрегации хромосом во время митоза приводят к анеуплоидии, распространенной цитогенетической особенности разных типов опухолевых клеток, в том числе ГК. Белок Aurora kinase A и мишень этого белка, протеин киназа HURP,играющие важную роль в сегрегации хромосом,– гиперэкспрессированы в ГК \cite{farazi_hepatocellular_2006}.

\subsubsection{Роль аллель-специфической экспрессии при ГК}

В последние годы появился ряд работ, результаты которых свидетельствуют о роли неравновесного уровня экспрессии аллелей генов, или аллель-специфической экспрессии (АСЭ), в этиологии некоторых наследственно обусловленных форм опухолей \cite{galiatsatos_familial_2006, chan_heritable_2006, buzby_allele-specific_2017}. К настоящему моменту роль АСЭ в инициации и прогрессии ГК малоизучена, однако было описано влияние аллельного дисбаланса гена ингибитора роста, индуцированного оксидативным стрессом 1 (\textit{OSGIN1}), на прогрессию ГК \cite{liu_allele-specific_2014}. \textit{OSGIN1} является опухолевым супрессором, играющим важную роль в регуляции клеточной гибели. При возникновении замены c.G1494A в гене \textit{OSGIN1} экспрессируемый белок утрачивает способность к транслокации из ядра в митохондрию, что приводит к утрате его про-апоптотической функции. При этом в опухолевых клетках нередко наблюдается аллельный дисбаланс экспрессии этого гена ввиду потери гетерозиготности. Кроме изменения в локализации белка, обусловленной наличием однонуклеотидной замены, менялся характер экспрессии аллелей гена \textit{OSGIN1} в ГК – происходило уменьшение уровня экспрессии. При этом пациенты со сниженным в два и более раза уровнем экспрессии характеризовались сниженной общей и безрецидивной выживаемостью. Общая выживаемость в группе пациентов с рассматриваемым герминативным вариантом была наиболее низкой по сравнению с группами с нормальной и сниженной экспрессией. Кроме того, в опухолях пациентов, являющихся носителями рассматриваемого варианта, апоптотический индекс был ниже, чем в ГК других пациентов \cite{liu_allele-specific_2014}.

На ранних стадиях ГК была описана АСЭ гена инсулин-подобного ростового фактора 2 (\textit{IGF2}), а также повышенный уровень экспрессии этого гена \cite{takeda_allelic-expression_1996}. АСЭ гена \textit{IGF2} наблюдалась в опухолевой ткани, а также при дисплазии, причем уровень экспрессии гена и соотношение экспрессии его аллелей изменялись в зависимости от уровня дифференцировки клеток. Хорошо дифференцированные диспластические узелки характеризовалась значительным увеличением уровня экспрессии \textit{IGF2}, в то время как на ранней стадии развития ГК уровень экспрессии гена был незначительно выше, чем в неопухолевой ткани печени. В то же время, аллельный дисбаланс чаще наблюдался в умеренно-,  или высокодифференцированных опухолях, чем при дисплазии \cite{aihara_allelic_1998}.
 
