\section{Материалы и методы}

\subsection{Клинические образцы}

Для исследования были использованы клинические образцы ткани ГК человека и прилежащей к опухоли неопухолевой ткани печени тех же пациентов, полученные во время проведения резекции опухолей пациентов с ГК в отделении опухолей печени и поджелудочной железы ФГБУ «НМИЦ им. Н. Н. Блохина» Минздрава РФ. Сразу после резекции взятые образцы замораживали в жидком азоте, после чего хранили их при температуре -70 \textdegree{}C. Геномную ДНК и тотальную РНК выделяли согласно протоколам производителей с использованием наборов Wizard SV Genomic DNA Purification System (Promega) с последующим переосаждением ДНК и PureLink RNA Mini Kit (Life Technologies) с применением ДНКазы, соответственно. Концентрацию ДНК и РНК определяли по оптической плотности раствора, которую измеряли на спектрофотометре NanoDrop1000 (Thermo Scientific, США) при длине волны 260 нм. Присутствие примесей в образцах оценивали по соотношению значений оптической плотности раствора ДНК, РНК при длинах волн 230, 260 и 280 нм. Данные полнотранскриптомного секвенирования клинических образцов ГК человека и соответствующих им неопухолевых тканей печени (bam-файлы) получены ранее в сотрудничестве с лабораторией эволюционной геномики ФББ МГУ.

\subsection{Поиск потенциально патогенных герминативных SNV в аллель-специфически экспрессирующихся генах}

Для нахождения герминативных однонуклеотидных вариантов (SNV) в аллель-специфически экспрессирующихся (АСЭ) генах были использованы транскриптомные данные 45 парных образцов гепатокарциномы и прилежащей ткани печени, представленные в двух биологических репликах. Транскриптомы представлены в виде выровненных картированных прочтений на геном человека сборки GRCh37. Списки генов, в которых был произведен поиск SNV, – онкогены (138 генов), опухолевые супрессоры (278 генов) и печень-специфические гены (156 генов), – были выгружены из базы данных соматических мутаций (COSMIC v87 \cite{forbes_cosmic:_2017}) и The Human Protein Atlas \cite{uhlen_pathology_2017}. 

Герминативные SNV были выявлены  при помощи набора инструментов GATK (Genome Analysis Toolkit, версия 4.0.10.1) \cite{mckenna_genome_2010}. В частности, были использованы HaplotypeCaller для поиска SNV в экзонах генов из составленного нами перечня, CombineGVCF для иерархического объединения gVCF-файлов, полученных на предыдущем шаге, GenotypeGVCF для получения корректных вероятностей генотипа, VariantFiltration для жесткой фильтрации SNV – пороговые значения QD, ReadPosRankSum, FS, MQ и MQRankSum были повышены. Кроме того, для повышения точности предсказания пороговое значение качества нуклеотида в прочтении было повышено с 10 до 20 единиц. В целях снижения вероятности получения ложноположительных результатов из дальнейшего рассмотрения исключались SNV, расположенные в участках с высокой гомологией с другими областями генома человека согласно информации, полученной в базах данных Segmental Duplications и SelfChain. Также исключались инделы, поскольку, несмотря на локальную пересборку участков в HaplotypeCaller, большая часть замен такого типа была выявлена на границах протяженных участков повторяющихся нуклеотидов и являлась артефактом.

Полученные SNV, отвечающие критериям качества на глубину покрытия, картирования, и равномерно представленные в прямой и обратной цепях парно-концевых прочтений, а также прошедшие дополнительную фильтрацию по глубине покрытия и долям референсного и альтернативного вариантов использовали для поиска АСЭ генов алгоритмом MBASED \cite{mayba_mbased:_2014}. Этот статистический метод позволяет предсказывать аллель-специфическую экспрессию гена на основе транскриптомных данных о покрытии множества сайтов, содержащих SNV, в том числе без опоры на информацию о гаплотипе. По результатам вычислений, проведенных с использованием MBASED в режиме анализа отдельных образцов, аллель-специфической экспрессией считали экспрессию гена, в которой доля мажорного аллеля превышала 70 \% при FDR < 5 \%. FDR вычисляли методом Бенджамини-Хохберга на основе данных о p-value, полученных алгоритмом MBASED для предсказанных долей мажорных аллелей.

Для получения списка потенциально патогенных SNV в АСЭ генах были установлены критерии функциональной значимости – замена должна быть несинонимичной, а частота варианта – низкой (< 5 \% в популяции). Кроме того, были использованы алгоритмы предсказания функциональной значимости (Polyphen-2 \cite{adzhubei_predicting_2013}), также учитывалось наличие данных об этом варианте в литературе.

\subsection{Олигонуклеотидные праймеры и зонды, использованные при проведении ПЦР}

Для дизайна зондов и фланкирующих праймеров были использованы референсные последовательности из базы данных NCBI, а также программа PrimerBlast \cite{ye_primer-blast:_2012}. Аллель-специфические зонды TaqMan, меченные FAM, для гена MET и его транскрипта NM\_001324402.1 комплементарны референсному или альтернативному варианту нуклеотидной последовательности, которые различаются одной позицией – rs56391007 (нуклеотидная замена c.C3029T, аминокислотная замена T1010I), а также содержат дополнительную замену вблизи от нуклеотида, специфичного к одному из аллелей. Для аллель-специфических зондов, меченных FAM, был подобран общий прямой зонд TaqMan, меченный HEX. Все зонды были также помечены с 3’ конца гасителем BHQ1. 

Последовательности праймеров и зондов, длины соответствующих ПЦР-продуктов приведены в таблице \ref{table:primers} (жирным шрифтом выделен нуклеотид, соответствующий замене c.C3029T). Праймеры и зонды были синтезированы фирмой «ДНК-синтез», Россия.

\begin{table}[H]
	\renewcommand{\arraystretch}{1.4} %increased line spacing
	\caption{\textbf{Специфические праймеры, использованные при проведении ПЦР.}}
	\label{table:primers}
	\enspace
	\begin{adjustbox}{max width=\textwidth}
		\begin{tabu} spread 0pt {XXXXX}
			\hline
			\multicolumn{1}{p{2.5cm}}{\textbf{Матрица}} & 
			\multicolumn{1}{p{8.5cm}}{\textbf{Последовательность прямого праймера/зонда (от 5' к 3')}} & 
			\multicolumn{1}{p{9.0cm}}{\textbf{Последовательность обратного праймера/зонда (от 5' к 3')}} & 
			\multicolumn{1}{p{3.0cm}}{\textbf{Длина ПЦР-продукта, пн}} &
			\multicolumn{1}{p{2.5cm}}{\textbf{Температура отжига, \textdegree{}C}}\\
			\hline
			Фланкирующие праймеры на \textit{MET} & CACTCCTCATTTGGATAG & GAAAAGTAGCTCGGTAG & 95 & 49 \\
			Зонды на \textit{MET} & HEX\_CTTGTAAGTGCCCGAAG\_BHQ1 & FAM\_AGCGCAA\textbf{\textit{C}}TACAGAAATGG\_BHQ1 \newline FAM\_AGCGCAA\textbf{\textit{T}}TACAGAAATGG\_BHQ1 & -- & 49 \\
			\hline
		\end{tabu}
	\end{adjustbox}
\end{table}

Рабочие растворы праймеров содержали прямой и обратный праймеры в концентрации 4 пкМ/мкл каждый. Концентрация прямого и обратного зондов в рабочем растворе составила 2,5 пкМ/мкл.

\subsection{Получение кДНК путем обратной транскрипции мРНК}

Обратную транскрипцию препаратов тотальной РНК проводили в 12,3 мкл смеси, содержащей 2,0 мкг РНК и 0,1 мкг случайных гексамерных олигонуклеотидов, которую денатурировали при 72 \textdegree{}C в течение 10 мин, охлаждали до комнатной температуры и добавляли 7,7 мкл смеси, состоящей из 4 мкл MMLV-буфера (Promega, США), по 2 мкл каждого из dNTP, 2,5 мМ (Силекс, Россия), 50 ед. обратной транскриптазы MMLV (Promega, США), 1 мкл 2 мМ дитиотритола (Sigma, США), и 2,5 ед. ингибитора рибонуклеаз RNasin Ribonuclease Inhibitor (Promega, США). Реакцию проводили при 42 \textdegree{}C в течение 60 мин и останавливали путем инактивации обратной транскриптазы при 95 \textdegree{}C в течение 10 минут. Объем реакционной смеси доводили деионизованной водой до 100 мкл и использовали аликвоты для постановки ПЦР со специфическими праймерами.

\subsection{Цифровая капельная ПЦР}

Количественное определение уровня экспрессии генов или их аллелей проводили методом цифровой капельной ПЦР с зондами TaqMan с использованием генератора капель Bio-Rad QX200 Droplet Generator, амплификатора C1000 Touch Thermal Cycler и счетчика капель Bio-Rad QX200 Droplet Reader (Bio-Rad Laboratories, США). Реакционная смесь 1 пробы объемом 25 мкл содержала: 10 мкл раствора кДНК в количестве, эквивалентном 10 нг исходной РНК, или 1 нг геномной ДНК, и 15 мкл смеси, содержащей 2 мкл специфических зондов, 4,5 мкл специфических праймеров, 5 мкл 2x Bio-Rad Supermix. Детекцию проводили в программе QuantaSoft (Bio-Rad Laboratories, США).

При проведении реакции амплификации были использованы следующие температурные режимы:

\begin{enumerate}
	\item 95 \textdegree{}C, 10 мин -- первичная денатурация;
	\item 40 циклов:
	\begin{itemize}
		\item 95 \textdegree{}C, 30 сек -- денатурация;
		\item T\textsubscript{отжига} = 49 \textdegree{}C, 1 мин -- отжиг праймеров/зондов (таблица \ref{table:primers});
		\item 72 \textdegree{}C, 30 сек -- элонгация;
	\end{itemize} 
	\item 72 \textdegree{}C, 3 мин -- полная элонгация;
\end{enumerate}

Для каждой комбинации образцов тканей, для которых в ходе скрининга было выявлено наличие полиморфизма rs56391007, и наборов праймеров/зондов было сделано и проанализировано по 3 технических реплики.

