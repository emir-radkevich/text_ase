\section{Список сокращений}

\begin{enumerate}
	\footnotesize
	\item АСЭ -- аллель-специфическая экспрессия
	\item SNV -- однонуклеотидный вариант
	\item SNP -- однонуклеотидный полиморфизм
	\item ГК -- гепатокарцинома
	\item HBV -- вирус гепатита B человека
	\item HCV -- вирус гепатита C человека
	\item ЭР -- эндоплазматический ретикулум, или эндоплазматическая сеть
	\item HBsAg -- антиген вируса гепатита B
	\item ЮД -- юкстамембранный домен 
	\item МДС -- мультифункциональный домен связывания
	\item PtdIns3P -- фосфатидилинозитол-трифосфат
	\item \textit{APC} -- аденоматозный полипоз кишечника (adenomatous polyposis coli gene)
	\item \textit{MSH2} -- гомолог 2 белка MutS (MutS protein homolog 2 gene)
	\item \textit{TP53} -- опухолевый белок 53 (tumor protein 53 gene)
	\item \textit{MLL4} -- смешанная лейкемия 4 (mixed-lineage leukemia 4 gene)
	\item \textit{CCNE1} -- циклин E1 (cyclin E1 gene)
	\item \textit{MET} -- рецептор фактора мезенхимально-эпителиального перехода ( mesenchymal-epithelial transition factor receptor gene) 
	\item \textit{CTNNB1} -- $\upbeta$-катенин (catenin $\upbeta$1 gene)
	\item \textit{CDKN2A} -- циклин-зависимый киназный ингибитор 2A (cyclin-dependent kinase inhibitor 2A gene)
	\item \textit{RB1} -- ретинобластома 1 (retinoblastoma 1 gene)
	\item \textit{PKM2} -- изозимы пируват киназы M1/M2 (pyruvate kinase isozymes M1/M2 gene)
	\item \textit{ALB} -- альбумин (albumin gene)
	\item \textit{APOB} -- аполипопротеин B (apolipoprotein B gene)
	\item \textit{OSGIN1} -- ростовой фактор, индуцированный оксилительным стрессом (oxidative stress-induced growth factor 1 gene)
	\item \textit{IGF2} -- инсулиноподобный ростовой фактор 2 (insulin-like growth factor 2 gene)
	\item HGF -- гепатоцитарный ростовой фактор (hepatocyte growth factor)
	\item TERT -- обратная транскриптаза полимеразы (telomerase reverse transcriptase)
	\item PDGF$\upbeta$ -- тромоцитарный фактор роста $\upbeta$ (platelet-derived growth factor$\upbeta$)
	\item EGFR -- рецептор эпидермального фактора роста (epidermal growth factor receptor)
	\item IGFR -- рецептор инсулиноподобного фактора роста (insulin-like growth factor receptor)
	\item TNF$\upalpha$ -- фактор некроза опухоли $\upalpha$ (tumor necrosis factor$\upalpha$)
	\item IFN$\upalpha$ -- интерферон $\upalpha$ (interferon$\upalpha$)
	\item HRAS -- Харви Ras белок (Harvey Ras protein)
	\item miR-122 -- микроРНК-122 (microRNA-122)
	\item MDM4 -- E3 убиквитин-лигаза Mdm4 (E3 ubiquitin-protein ligase Mdm4)
	\item PI3K -- фосфоинозитид-3-киназы (phosphoinositide 3-kinases)
	\item GRB2 -- белок 2, связывающийся с рецептором ростового фактора (growth factor receptor-bound protein 2)
	\item SHP2 или PTPN11 -- фосфатаза нерецепторного типа 11 (tyrosine-protein phosphatase non-receptor type 11)
	\item SHC -- домен Src гомологии 2, содержащий трансформирующий белок 1 (Src homology 2 domain containing transforming protein 1)
	\item PLC -- фосфолипаза C (phospholipase C protein)
	\item STAT3 -- трансдуктор сигнала и активатор транскрипции 3 (signal transducer and activator of transcription 3)
	\item GAB1 -- GRB2-связывающий белок 1 (GRB2-associated-binding protein 1)
	\item MAPK -- митоген-активируемые протеин киназы (mitogen-activated protein kinases)
	\item SOS -- son of sevenless protein
	\item ERK --  внеклеточные сигнал-регулируемые киназы (extracellular signal-regulated kinases)
	\item p120-Ras-GAP -- p120 Ras ГТФаза активирующий белок (p120 Ras GTPase activating protein)
	\item JNK -- c-Jun N-концевые киназы (c-Jun N-terminal kinases)
	\item AKT, или PKB -- протеин киназа B (protein-kinase B)
	\item IKK -- I$\upkappa$B киназа (I$\upkappa$B kinase)
	\item I$\upkappa$B$\upalpha$ -- NF$\upkappa$B ингибитор $\upalpha$ (NF$\upkappa$B inhibitor$\upalpha$)
	\item NF$\upkappa$B -- ядерный фактор энхансера гена $\upkappa$ легкого полипептида в В-клетках (nuclear factor of $\upkappa$ light polypeptide gene enhancer in B-cells)
	\item ADAM -- дезинтегрин и металлопротеиназа домен-содержащие белки (a disintegrin and metalloproteinase domain-containing protein)
	\item CBL -- E3 убиквитин лигаза CBL (E3 ubiquitin-protein ligase casitas B-lineage lymphoma)
\end{enumerate}