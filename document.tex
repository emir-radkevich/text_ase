\documentclass[12pt]{article} %font size, document class

%-------------------LANGUAGE-------------------%
\usepackage[utf8]{inputenc} %translates input encodings into "LATEX internal encoding"
\usepackage[T2A]{fontenc} %font encoding; T2A -- enough for cyrillic and english, but if umlauts or similar stuff are used it's recommended to use T1 either
\usepackage[english,russian]{babel} %language-specific hyphenation, easy use of special symbols
%\usepackage{float} %для плавающих картинок и таблиц

%-------------------STUFF-------------------%
\usepackage{cmap}  %search in pdf enabled
\usepackage[pdftex]{graphicx} %works with graphics 
 

%-------------------PAPER_AND_TEXT-------------------%
\usepackage{vmargin} %page layout
\setpapersize{A4} %paper format
\setmarginsrb{30mm}{25mm}{25mm}{25mm}{0pt}{0mm}{0pt}{13mm} %paper field sizes: left, up, right, down, 3*колонтитулы, расстояние между нижним краем нижней строки и нижним краем номера страницы
\usepackage[hidelinks]{hyperref}
\usepackage{setspace}
\usepackage{indentfirst} %красная строка для первого абзаца главы или параграфа
\pagestyle{plain} %включена нумерация страниц
%\sloppy %борьба с залезанием строк на поля путём изменения размеров пробелов
%\usepackage{amsmath} %дополнительные средства для вёрстки формул
%\everymath{\displaystyle}
%\usepackage{breqn} %для dmath разбить длинную формулу
%\usepackage{amsfonts} %дополнительные шрифты для формул
%\usepackage{amssymb} %дополнительные символы для формул
%\usepackage[nottoc,notlot,notlof]{tocbibind}
 
%\renewcommand{\thesection}{\arabic{section}} %одинарная нумерация формул


%\usepackage{verbatim} %comments

%\usepackage[font=small]{caption}
%\usepackage{booktabs} %отступы в tabular
%\usepackage{colortbl} %раскаршивание таблиц
%\usepackage{xcolor} %название цветов


%\usepackage{lineno} %нумерация всех строк для отладки
%\usepackage{enumitem} %особенности enumerate
%\usepackage{pbox} % для переносов внутри ячейки таблицы
%\usepackage{array} %??
%\usepackage{longtable}% перенос таблиц на страницах

%\newlength{\width}
%\setlength{\width}{0.97\textwidth}
%\setlength{\parindent}{3em}
%\setlength{\parskip}{0.3em}

%\usepackage{textgreek} % \textalpha греческие буквы не в math mode
%\usepackage{subcaption} % несколько картинок в одной
%\usepackage{multirow} % объединение ячеек
%\setcounter{tocdepth}{2} % глубина содержания


%-------------------CITE-------------------%
\usepackage{cite}

\begin{document}
        \begin{titlepage}

\newcommand{\HRule}{\rule{\linewidth}{0.3mm}} % Defines a new command for the horizontal lines, change thickness here

\center

\textbf{\textsc{Московский государственный университет имени М.В.Ломоносова}}
\\[0.3cm] 
\HRule 
\\[0.3cm]
\textbf{\textsc{Факультет биоинженерии и биоинформатики}}
\\[2.0cm]

\begin{spacing}{1.4}
{\LARGE \textbf{Identification of putatively pathogenic single nucleotide polymorphisms in genes demonstrating allele-specific expression in hepatocellular carcinoma}} \\[1.0cm]


{\LARGE \textbf{Идентификация потенциально патогенных однонуклеотидных полиморфизмов в аллель-специфически экспрессирующихся генах при гепатоканцерогенезе}} \\[2.0cm]
\end{spacing}
 
 
\Large \textit{Курсовая работа студента четвертого курса:}\\
Радкевича Эмира Мустафовича
\\[1.0cm]


\begin{flushright} \large
Научные руководители: \\
проф.,\,д.б.н.\,Лазаревич\,Н.Л. \\
аспирант\,Кривцова\,О.М. \\
[0.6cm]
\end{flushright}

\raggedright
\begin{minipage}[t]{0.45\textwidth}
	\vspace{0pt}
	\begin{flushleft} \large
		\hrulefill{Лазаревич Н.Л.} \\
		\hrulefill{Кривцова О.М.} 
	\end{flushleft}
\end{minipage}
\vspace{\fill}

\center
{\large Москва \\ 2019}


\end{titlepage} \newpage
        \tableofcontents \newpage
        \section{Список сокращений}

\begin{enumerate}
	\footnotesize
	\item АСЭ -- аллель-специфическая экспрессия
	\item SNV -- однонуклеотидный вариант
	\item SNP -- однонуклеотидный полиморфизм
	\item ГК -- гепатокарцинома
	\item HBV -- вирус гепатита B человека
	\item HCV -- вирус гепатита C человека
	\item ЭР -- эндоплазматический ретикулум, или эндоплазматическая сеть
	\item HBsAg -- антиген вируса гепатита B
	\item ЮД -- юкстамембранный домен 
	\item МДС -- мультифункциональный домен связывания
	\item PtdIns3P -- фосфатидилинозитол-трифосфат
	\item \textit{APC} -- аденоматозный полипоз кишечника (adenomatous polyposis coli gene)
	\item \textit{MSH2} -- гомолог 2 белка MutS (MutS protein homolog 2 gene)
	\item \textit{TP53} -- опухолевый белок 53 (tumor protein 53 gene)
	\item \textit{MLL4} -- смешанная лейкемия 4 (mixed-lineage leukemia 4 gene)
	\item \textit{CCNE1} -- циклин E1 (cyclin E1 gene)
	\item \textit{MET} -- рецептор фактора мезенхимально-эпителиального перехода ( mesenchymal-epithelial transition factor receptor gene) 
	\item \textit{CTNNB1} -- $\upbeta$-катенин (catenin $\upbeta$1 gene)
	\item \textit{CDKN2A} -- циклин-зависимый киназный ингибитор 2A (cyclin-dependent kinase inhibitor 2A gene)
	\item \textit{RB1} -- ретинобластома 1 (retinoblastoma 1 gene)
	\item \textit{PKM2} -- изозимы пируват киназы M1/M2 (pyruvate kinase isozymes M1/M2 gene)
	\item \textit{ALB} -- альбумин (albumin gene)
	\item \textit{APOB} -- аполипопротеин B (apolipoprotein B gene)
	\item \textit{OSGIN1} -- ростовой фактор, индуцированный оксилительным стрессом (oxidative stress-induced growth factor 1 gene)
	\item \textit{IGF2} -- инсулиноподобный ростовой фактор 2 (insulin-like growth factor 2 gene)
	\item HGF -- гепатоцитарный ростовой фактор (hepatocyte growth factor)
	\item TERT -- обратная транскриптаза полимеразы (telomerase reverse transcriptase)
	\item PDGF$\upbeta$ -- тромоцитарный фактор роста $\upbeta$ (platelet-derived growth factor$\upbeta$)
	\item EGFR -- рецептор эпидермального фактора роста (epidermal growth factor receptor)
	\item IGFR -- рецептор инсулиноподобного фактора роста (insulin-like growth factor receptor)
	\item TNF$\upalpha$ -- фактор некроза опухоли $\upalpha$ (tumor necrosis factor$\upalpha$)
	\item IFN$\upalpha$ -- интерферон $\upalpha$ (interferon$\upalpha$)
	\item HRAS -- Харви Ras белок (Harvey Ras protein)
	\item miR-122 -- микроРНК-122 (microRNA-122)
	\item MDM4 -- E3 убиквитин-лигаза Mdm4 (E3 ubiquitin-protein ligase Mdm4)
	\item PI3K -- фосфоинозитид-3-киназы (phosphoinositide 3-kinases)
	\item GRB2 -- белок 2, связывающийся с рецептором ростового фактора (growth factor receptor-bound protein 2)
	\item SHP2 или PTPN11 -- фосфатаза нерецепторного типа 11 (tyrosine-protein phosphatase non-receptor type 11)
	\item SHC -- домен Src гомологии 2, содержащий трансформирующий белок 1 (Src homology 2 domain containing transforming protein 1)
	\item PLC -- фосфолипаза C (phospholipase C protein)
	\item STAT3 -- трансдуктор сигнала и активатор транскрипции 3 (signal transducer and activator of transcription 3)
	\item GAB1 -- GRB2-связывающий белок 1 (GRB2-associated-binding protein 1)
	\item MAPK -- митоген-активируемые протеин киназы (mitogen-activated protein kinases)
	\item SOS -- son of sevenless protein
	\item ERK --  внеклеточные сигнал-регулируемые киназы (extracellular signal-regulated kinases)
	\item p120-Ras-GAP -- p120 Ras ГТФаза активирующий белок (p120 Ras GTPase activating protein)
	\item JNK -- c-Jun N-концевые киназы (c-Jun N-terminal kinases)
	\item AKT, или PKB -- протеин киназа B (protein-kinase B)
	\item IKK -- I$\upkappa$B киназа (I$\upkappa$B kinase)
	\item I$\upkappa$B$\upalpha$ -- NF$\upkappa$B ингибитор $\upalpha$ (NF$\upkappa$B inhibitor$\upalpha$)
	\item NF$\upkappa$B -- ядерный фактор энхансера гена $\upkappa$ легкого полипептида в В-клетках (nuclear factor of $\upkappa$ light polypeptide gene enhancer in B-cells)
	\item ADAM -- дезинтегрин и металлопротеиназа домен-содержащие белки (a disintegrin and metalloproteinase domain-containing protein)
	\item CBL -- E3 убиквитин лигаза CBL (E3 ubiquitin-protein ligase casitas B-lineage lymphoma)
\end{enumerate} \newpage
        \section{Введение}

В геноме человека выявляется множество герминативных вариантов в регуляторных областях, что может обусловливать межиндивидуальные различия в уровнях экспрессии генов, и в белок-кодирующих участках. Среди последних даже у здоровых индивидов обнаруживаются описанные ранее и потенциально патогенные варианты, ассоциированные с развитием наследственных заболеваний, которые, однако, не привели к их манифестации.

В большинстве случаев экспрессия гена осуществляется в равной степени с обеих гомологичных хромосом. Тем не менее, в последние годы появился ряд работ, результаты которых свидетельствуют о роли аллель-специфической экспрессии (АСЭ), то есть неравновесного уровня экспрессии аллелей генов, в этиологии некоторых наследственных заболеваний, включая наследственно обусловленные формы опухолей.

Так, ранее была показана связь между пониженной экспрессией одного из аллелей гена \textit{APC} и предрасположенностью к развитию семейного аденоматозного полипоза \cite{galiatsatos_familial_2006}. Также, было показано, что аллель-специфическое метилирование промотора гена \textit{MSH2} может приводить к развитию синдрома Линча \cite{chan_heritable_2006}. В обоих случаях АСЭ была связана с измененным уровнем экспрессии гена. Помимо этого, АСЭ способна влиять на экспрессию аллелей, содержащих патогенные герминативные варианты. Например, показана роль АСЭ в развитии синдрома Ли-Фраумени у членов семьи, являвшихся носителями функциональной герминативной мутации в гене \textit{TP53}, обусловливающей фосфорилирование p53 \cite{buzby_allele-specific_2017}.

Несмотря на наличие сообщений о существовании АСЭ в спорадических опухолях, работ, исследующих распространенность, паттерн и роль АСЭ при гепатокарциноме (ГК), в том числе применительно к ее влиянию на экспрессию патогенных аллелей, до настоящего времени опубликовано не было. При этом ГК свойственна высокая гетерогенность соматических нарушений, не позволяющих исчерпывающим образом установить лежащие в основе ее формирования и развития молекулярные механизмы. Таким образом, выявление наиболее общих для ГК нарушений аллельной экспрессии и последующая экспериментальная проверка их роли в приобретении опухолевыми клетками более злокачественного фенотипа позволит оценить значимость АСЭ как дополнительного фактора опухолевой прогрессии.

Для достижения этой цели мы поставили перед собой следующие задачи:
\begin{enumerate}
	\item выявить АСЭ генов в транскриптомах 45 парных образцов ГК и неопухолевой ткани печени;
	\item найти потенциально патогенные однонуклеотидные полиморфизмы (SNV) в АСЭ генах;
	\item экспериментально подтвердить АСЭ генов, в том числе в целях проверки корректности предсказаний о дисбалансе на основе выбранного нами алгоритма.
\end{enumerate}
        \section{Литература}

\subsection{Канцерогенез}



\subsection{Гепатокарцинома}

\subsection{Рецептор гепатоцитарного фактора роста, MET}

Рецептор MET -- это рецептор гепатоцитарного фактора роста (HGF). В 1987 году было показано \cite{park_sequence_1987}, что MET принадлежит к семейству тирозин-киназных белков. Однако, в то время считалось, что лигандом MET является не только HGF \cite{nakamura_molecular_1989}, но и так называемый scatter factor (SF), белковый фактор подвижности эпителия, полученный из фибробластов \cite{stoker_scatter_1987}. К 1991 году удалось доказать, что HGF и SF -- это один и тот же белок \cite{weidner_evidence_1991}. Интересной особенностью HGF-MET системы является разнообразный клеточный ответ, вызванный опосредованным взаимодействием активированного рецептора MET с белками различных сигнальных путей клетки \cite{trusolino_met_2010}.

Рецептор MET -- это тирозин-киназный рецептор, состоящий из следующих функциональных доменов \cite{trusolino_met_2010}:
\begin{enumerate}
	\item Внеклеточные домены:
	\begin{itemize}
		\item \textbf{SEMA-домен}, представленный $\alpha$- и $\beta$-субъединицами, необходим для связывания с HGF;
		\item \textbf{PSI-домен}, принимающий участие в связывании с HGF;
		\item 4 \textbf{иммунноглобулин-подобных домена};
	\end{itemize} 
	\item Внутриклеточные домены:
		\begin{itemize}
		\item \textbf{юкстамембранный домен} (ЮД) участвует в связывании с убиквитин-лигазой;
		\item \textbf{тирозин-киназный домен} -- каталитический регион, модулирующий киназную активность;
		\item \textbf{мультифункицональный домен связывания} (МДС) участвует в связывании с белками-передатчиками сигнала внутрь клетки.
	\end{itemize} 
\end{enumerate}

Внутриклеточные домены могут находиться в фосфорилированом состоянии, что влияет на работу рецептора. При связывании рецептора с HGF происходит димеризация MET, что приводит к транс-фосфорилиированию по двум сайтам -- Tyr1234 и Tyr1235. Следующий шаг в механизме работы рецептора -- это автофосфорилирование по двум аминокислотным остаткам МДС -- Tyr1349 и Tyr1356. Фосфорилирование двух последниз сайтов -- необходимый шаг, так как в таком состоянии MET способен связываться с адапторными белками -- эффекторы, содержащие Src-гомолог-2 домен (PI3K) \cite{ponzetto_multifunctional_1994}, белок 2, связанный с рецептором ростового фактора (GRB2) \cite{ponzetto_multifunctional_1994}, SHP2 \cite{fixman_pathways_1996} и SHC \cite{fixman_pathways_1996}, фосфолипаза C$\gamma$ (PLC$\gamma$) \cite{ponzetto_multifunctional_1994}, транскрипционные факторы семейства STAT (STAT3) \cite{zhang_requirement_2002}. 

MET также связывается с GRB2-ассоциированным связывающим белком 1 (GAB1), мультиадаптерным белком, который после фосфорилирования рецептора MET, обеспечивает дополнительные сайты связывания для SHC, PI3K, SHP2, PLC$\gamma$ и p120 ras-GTPase-активирующего белок (p120-ras-GAP) \cite{trusolino_met_2010}. MET взаимодействует с GAB1 напрямую, через уникальный сайт длиной 13 аминокислотных остатков, или опосредовано через GRB2 \cite{schaeper_coupling_2000} \cite{lock_identification_2000}.

Правильное функционирование рецептора необходимо для различных морфогенетических событий как в эмбриональной, так и во взрослой жизни. Рецептор принимает участие в управлении злокачественным прогрессированием нескольких различных типов опухолей. Для этого MET распространяет сложную систему сигнальных каскадов, в результате чего
перестройка экспрессии генов

\subsection{Мутации в MET}


        \section{Материалы и методы}

\subsection{Клинические образцы}

Для исследования были использованы клинические образцы ткани ГК человека и прилежащей к опухоли неопухолевой ткани печени тех же пациентов, полученные во время проведения резекции опухолей пациентов с ГК в отделении опухолей печени и поджелудочной железы ФГБУ «НМИЦ им. Н. Н. Блохина» Минздрава РФ. Сразу после резекции взятые образцы замораживали в жидком азоте, после чего хранили их при температуре -70 \textdegree{}C. Геномную ДНК и тотальную РНК выделяли согласно протоколам производителей с использованием наборов Wizard SV Genomic DNA Purification System (Promega) с последующим переосаждением ДНК и PureLink RNA Mini Kit (Life Technologies) с применением ДНКазы, соответственно. Концентрацию ДНК и РНК определяли по оптической плотности раствора, которую измеряли на спектрофотометре NanoDrop1000 (Thermo Scientific, США) при длине волны 260 нм. Присутствие примесей в образцах оценивали по соотношению значений оптической плотности раствора ДНК, РНК при длинах волн 230, 260 и 280 нм. Данные полнотранскриптомного секвенирования клинических образцов ГК человека и соответствующих им неопухолевых тканей печени (bam-файлы) получены ранее в сотрудничестве с лабораторией эволюционной геномики ФББ МГУ.

\subsection{Поиск потенциально патогенных герминативных SNV в аллель-специфически экспрессирующихся генах}

Для нахождения герминативных однонуклеотидных вариантов (SNV) в аллель-специфически экспрессирующихся (АСЭ) генах были использованы транскриптомные данные 45 парных образцов гепатокарциномы и прилежащей ткани печени, представленные в двух биологических репликах. Транскриптомы представлены в виде выровненных картированных прочтений на геном человека сборки GRCh37. Списки генов, в которых был произведен поиск SNV, – онкогены (138 генов), опухолевые супрессоры (278 генов) и печень-специфические гены (156 генов), – были выгружены из базы данных соматических мутаций (COSMIC v87 \cite{forbes_cosmic:_2017}) и The Human Protein Atlas \cite{uhlen_pathology_2017}. 

Герминативные SNV были выявлены  при помощи набора инструментов GATK (Genome Analysis Toolkit, версия 4.0.10.1) \cite{mckenna_genome_2010}. В частности, были использованы HaplotypeCaller для поиска SNV в экзонах генов из составленного нами перечня, CombineGVCF для иерархического объединения gVCF-файлов, полученных на предыдущем шаге, GenotypeGVCF для получения корректных вероятностей генотипа, VariantFiltration для жесткой фильтрации SNV – пороговые значения QD, ReadPosRankSum, FS, MQ и MQRankSum были повышены. Кроме того, для повышения точности предсказания пороговое значение качества нуклеотида в прочтении было повышено с 10 до 20 единиц. В целях снижения вероятности получения ложноположительных результатов из дальнейшего рассмотрения исключались SNV, расположенные в участках с высокой гомологией с другими областями генома человека согласно информации, полученной в базах данных Segmental Duplications и SelfChain. Также исключались инделы, поскольку, несмотря на локальную пересборку участков в HaplotypeCaller, большая часть замен такого типа была выявлена на границах протяженных участков повторяющихся нуклеотидов и являлась артефактом.

Полученные SNV, отвечающие критериям качества на глубину покрытия, картирования, и равномерно представленные в прямой и обратной цепях парно-концевых прочтений, а также прошедшие дополнительную фильтрацию по глубине покрытия и долям референсного и альтернативного вариантов использовали для поиска АСЭ генов алгоритмом MBASED \cite{mayba_mbased:_2014}. Этот статистический метод позволяет предсказывать аллель-специфическую экспрессию гена на основе транскриптомных данных о покрытии множества сайтов, содержащих SNV, в том числе без опоры на информацию о гаплотипе. По результатам вычислений, проведенных с использованием MBASED в режиме анализа отдельных образцов, аллель-специфической экспрессией считали экспрессию гена, в которой доля мажорного аллеля превышала 70 \% при FDR < 5 \%. FDR вычисляли методом Бенджамини-Хохберга на основе данных о p-value, полученных алгоритмом MBASED для предсказанных долей мажорных аллелей.

Для получения списка потенциально патогенных SNV в АСЭ генах были установлены критерии функциональной значимости – замена должна быть несинонимичной, а частота варианта – низкой (< 5 \% в популяции). Кроме того, были использованы алгоритмы предсказания функциональной значимости (Polyphen-2 \cite{adzhubei_predicting_2013}), также учитывалось наличие данных об этом варианте в литературе.

\subsection{Олигонуклеотидные праймеры и зонды, использованные при проведении ПЦР}

Для дизайна зондов и фланкирующих праймеров были использованы референсные последовательности из базы данных NCBI, а также программа PrimerBlast \cite{ye_primer-blast:_2012}. Аллель-специфические зонды TaqMan, меченные FAM, для гена MET и его транскрипта NM\_001324402.1 комплементарны референсному или альтернативному варианту нуклеотидной последовательности, которые различаются одной позицией – rs56391007 (нуклеотидная замена c.C3029T, аминокислотная замена T1010I), а также содержат дополнительную замену вблизи от нуклеотида, специфичного к одному из аллелей. Для аллель-специфических зондов, меченных FAM, был подобран общий прямой зонд TaqMan, меченный HEX. Все зонды были также помечены с 3’ конца гасителем BHQ1. 

Последовательности праймеров и зондов, длины соответствующих ПЦР-продуктов приведены в таблице \ref{table:primers} (жирным шрифтом выделен нуклеотид, соответствующий замене c.C3029T). Праймеры и зонды были синтезированы фирмой «ДНК-синтез», Россия.

\begin{table}[H]
	\renewcommand{\arraystretch}{1.4} %increased line spacing
	\caption{\textbf{Специфические праймеры, использованные при проведении ПЦР.}}
	\label{table:primers}
	\enspace
	\begin{adjustbox}{max width=\textwidth}
		\begin{tabu} spread 0pt {XXXXX}
			\hline
			\multicolumn{1}{p{2.5cm}}{\textbf{Матрица}} & 
			\multicolumn{1}{p{8.5cm}}{\textbf{Последовательность прямого праймера/зонда (от 5' к 3')}} & 
			\multicolumn{1}{p{9.0cm}}{\textbf{Последовательность обратного праймера/зонда (от 5' к 3')}} & 
			\multicolumn{1}{p{3.0cm}}{\textbf{Длина ПЦР-продукта, пн}} &
			\multicolumn{1}{p{2.5cm}}{\textbf{Температура отжига, \textdegree{}C}}\\
			\hline
			Фланкирующие праймеры на \textit{MET} & CACTCCTCATTTGGATAG & GAAAAGTAGCTCGGTAG & 95 & 49 \\
			Зонды на \textit{MET} & HEX\_CTTGTAAGTGCCCGAAG\_BHQ1 & FAM\_AGCGCAA\textbf{\textit{C}}TACAGAAATGG\_BHQ1 \newline FAM\_AGCGCAA\textbf{\textit{T}}TACAGAAATGG\_BHQ1 & -- & 49 \\
			\hline
		\end{tabu}
	\end{adjustbox}
\end{table}

Рабочие растворы праймеров содержали прямой и обратный праймеры в концентрации 4 пкМ/мкл каждый. Концентрация прямого и обратного зондов в рабочем растворе составила 2,5 пкМ/мкл.

\subsection{Получение кДНК путем обратной транскрипции мРНК}

Обратную транскрипцию препаратов тотальной РНК проводили в 12,3 мкл смеси, содержащей 2,0 мкг РНК и 0,1 мкг случайных гексамерных олигонуклеотидов, которую денатурировали при 72 \textdegree{}C в течение 10 мин, охлаждали до комнатной температуры и добавляли 7,7 мкл смеси, состоящей из 4 мкл MMLV-буфера (Promega, США), по 2 мкл каждого из dNTP, 2,5 мМ (Силекс, Россия), 50 ед. обратной транскриптазы MMLV (Promega, США), 1 мкл 2 мМ дитиотритола (Sigma, США), и 2,5 ед. ингибитора рибонуклеаз RNasin Ribonuclease Inhibitor (Promega, США). Реакцию проводили при 42 \textdegree{}C в течение 60 мин и останавливали путем инактивации обратной транскриптазы при 95 \textdegree{}C в течение 10 минут. Объем реакционной смеси доводили деионизованной водой до 100 мкл и использовали аликвоты для постановки ПЦР со специфическими праймерами.

\subsection{Цифровая капельная ПЦР}

Количественное определение уровня экспрессии генов или их аллелей проводили методом цифровой капельной ПЦР с зондами TaqMan с использованием генератора капель Bio-Rad QX200 Droplet Generator, амплификатора C1000 Touch Thermal Cycler и счетчика капель Bio-Rad QX200 Droplet Reader (Bio-Rad Laboratories, США). Реакционная смесь 1 пробы объемом 25 мкл содержала: 10 мкл раствора кДНК в количестве, эквивалентном 10 нг исходной РНК, или 1 нг геномной ДНК, и 15 мкл смеси, содержащей 2 мкл специфических зондов, 4,5 мкл специфических праймеров, 5 мкл 2x Bio-Rad Supermix. Детекцию проводили в программе QuantaSoft (Bio-Rad Laboratories, США).

При проведении реакции амплификации были использованы следующие температурные режимы:

\begin{enumerate}
	\item 95 \textdegree{}C, 10 мин -- первичная денатурация;
	\item 40 циклов:
	\begin{itemize}
		\item 95 \textdegree{}C, 30 сек -- денатурация;
		\item T\textsubscript{отжига} = 49 \textdegree{}C, 1 мин -- отжиг праймеров/зондов (таблица \ref{table:primers});
		\item 72 \textdegree{}C, 30 сек -- элонгация;
	\end{itemize} 
	\item 72 \textdegree{}C, 3 мин -- полная элонгация;
\end{enumerate}

Для каждой комбинации образцов тканей, для которых в ходе скрининга было выявлено наличие полиморфизма rs56391007, и наборов праймеров/зондов было сделано и проанализировано по 3 технических реплики.


        \section{Результаты}

\subsection{Поиск аллель-специфически экспрессирующихся генов}

На основе данных транскриптомного секвенирования 45 пар образцов пациентов с гепатокарциномой (ГК), выполненного в двух биологических репликах, предоставленных в формате картированных на референсный геном человека (GRCh37) прочтений, с использованием набора инструментов Genome Analysis Toolkit, в онкогенах, опухолевых супрессорах и печень-специфических генах рассматриваемых образцов было выявлено 13137 однонуклеотидных вариантов (SNV) в экзонах генов. При помощи алгоритма MBASED \cite{mayba_mbased:_2014} были вычислены значения доли мажорного аллеля (MAF), позволяющие оценить наличие аллель-специфической экспрессии (АСЭ) в ГК, на основе данных о покрытии множества сайтов, содержащих SNV. В результате работы алгоритма АСЭ (MAF $\geq 0.7$, FDR $< 5$ \%) в ГК и неопухолевой ткани была обнаружена в 125 генах из 572 исследуемых (21.8 \%). При этом на гены со статистически значимой MAF приходилось 4189 SNV, в том числе 1617 SNV находилось в генах, проявлявших аллельный дисбаланс.

Для каждого гена, демонстрирующего АСЭ (MAF $\geq 0.7$, FDR $< 5$ \%) в ГК и неопухолевой ткани печени, были рассчитаны частоты возникновения аллельного дисбаланса. На рисунке \ref{fig:normal} показаны все гены, у которых АСЭ в неопухолевой ткани наблюдается в 2 и более раза чаще, чем в ГК, – всего было обнаружено 4 таких гена. Существенно больше, 28 генов, демонстрируют обратную картину – частота АСЭ в ГК в 2 и более раза выше, чем частота АСЭ в неопухолевой ткани (рисунок \ref{fig:tumor}). Исходя из полученных результатов, можно предположить, что при приобретении клеткой опухолевого фенотипа наблюдается усиление аллельного дисбаланса некоторых генов, которые до трансформации характеризовались неравновесным уровнем экспрессии аллелей. 

Было построено распределение генов по частоте аллельного дисбаланса в ГК и прилежащей ткани печени (рисунок \ref{fig:all_ase}). На рисунке по $O_x$ и $O_y$ отложены медианы значений MAF для отдельных генов. Значения MAF в неопухолевой и опухолевой тканях демонстрируют высокий уровень корреляции друг с другом (коэффициент корреляции Спирмена=$0.85$, p-value=$7.6\times10^{-48}$). Частоты аллельного дисбаланса в опухоли и неопухолевой ткани (посчитаны как отношение количества случаев АСЭ на количество образцов) демонстрируют гораздо меньший уровень корреляции – коэффициент корреляции Спирмена=$0.4$, p-value=$6.3\times10^{-8}$. На рисунке отмечены некоторые онкогены, в скобках указаны значения количества образцов с АСЭ в неопухолевой/опухолевой тканях (полный список онкогенов в таблице \ref{table:onco}).

В таблице приведен полный перечень названий онкогенов, которые проявляли аллельный дисбаланс в ГК и прилежащей неопухолевой ткани. В скобках дана информация о количестве образцов нормальной и опухолевой ткани, в которых была предсказана статистически значимая АСЭ (MAF $\geq 0.7$, FDR $< 5$ \%).

\begin{figure}[H]
	\centering
	\begin{subfigure}[h]{0.49\textwidth}
		\includegraphics[width=\linewidth]{pics/high_tumor_ase_frequency_new.png}
		\caption{}
		\label{fig:tumor}
	\end{subfigure}
	\begin{subfigure}[h]{0.49\textwidth}
		\includegraphics[width=\textwidth]{pics/high_normal_ase_frequency_new.png}
		\caption{}
		\label{fig:normal}
	\end{subfigure}
	\caption{\textbf{Результаты поиска АСЭ генов с использованием алгоритма MBASED}. 
	 \ref{fig:tumor} -- Гены, у которых аллель-специфическая экспрессия (АСЭ) в ГК наблюдается в 2 раза чаще (и больше, чем в 2 раза), чем в неопухолевой ткани (на рисунке \ref{fig:normal} отражена обратная картина -- частота АСЭ выше в неопухолевой ткани). В скобках указано число образцов с АСЭ в неопухолевой ткани и ГК.}
	\label{fig:ase_freq}
\end{figure}

\begin{table}[H]
	\renewcommand{\arraystretch}{1.4} %increased line spacing
	\caption{\textbf{Список онкогенов с АСЭ в ГК и неопухолевой ткани.}}
	\label{table:onco}
	\enspace
	\centering
	\begin{adjustbox}{max width=\textwidth}
		\normalsize{
		\begin{tabu} spread 0pt {XXXXX}
			\hline
			\multicolumn{5}{c}{Онкогены. Число (опухолевая/неопухолевая ткани) образцов с АСЭ} \\
			\hline
			MET (1/8) & FGFR2 (2/7) & CSF1R (2/4) & CD74 (17/21) & TFE3 (1/2) \\
			CTNNB1 (1/6) & BIRC6 (7/13) & AKT1 (1/2) & NOTCH1 (2/2) & A1CF (6/4) \\
			ERBB3 (3/7) & GNAS (2/9) & ELK4 (1/1) & EGFR (1/4) & CCND1 (2/3) \\
			ERBB2 (1/5) & EPAS1 (7/4) & FGFR3 (1/4) & DDX5 (1/4) & MAPK1 (1/6) \\
			AR (5/1) & HIF1A (2/8) & ATP1A1 (3/4) & SND1 (5/11) & SKI (2/1) \\
			ESR1 (1/1) & BRD4 (4/3) & NFE2L2 (1/3) & TRRAP (2/2) & CHD4 (3/11) \\
			KDR (4/1) & KMT2D (2/5) & MTOR (1/7) & BCL9L (1/3) &  \\
			\hline
		\end{tabu}
		}
	\end{adjustbox}
\end{table}



\subsection{Идентификация потенциально патогенных герминативных вариантов в АСЭ генах}

После получения списка генов, проявляющих дифференциальную экспрессию аллелей, была выполнена оценка изменений в экспрессии аллелей, содержащих  однонуклеотдные герминативные миссенс- и нонсенс-варианты в них. На рисунке \ref{fig:sun} представлена информация о статистически значимых MAF аллелей, содержащих герминативные миссенс- и нонсенс-варианты в онкогенах, опухолевых супрессорах и печень-специфических генах (FDR $< 5$ \%, количество полученных вариантов -- 880) для пар образцов опухолевой и прилежащей ткани печени. В 50 \% случаев наблюдается преобладание MAF в неопухолевой ткани, в 40 \% случаев -- преобладание MAF в опухоли, в остальных 10 \% случаев наблюдается одинаковая доля мажорного аллеля в опухоли и прилежащей ткани печени. Полученные результаты свидетельствуют о том, что при трансформации неопухолевой клетки происходит изменение уровня экспрессии аллелей генов. Неравновесный уровень экспрессии выражается в отклоняющемся от 0.5 значении доли мажорного аллеля (MAF). При этом в случае патогенного аллеля изменения уровня его экспрессии может быть функционально значимым.


Для поиска таких вариантов мы установили следующие критерии: предсказание функциональной значимости варианта алгоритмом Polyphen-2 \cite{adzhubei_predicting_2013}, низкая частота встречаемости варианта в популяции ($< 5$ \%), а также наличие экспериментальных данных о функциональной значимости в литературе. В результате было выявлено незначительное количество потенциально патогенных герминативных вариантов в АСЭ генах, в том числе:

\begin{itemize}
	\item у 7 пациентов (в том числе у 3 пациентов с АСЭ) rs56325023 в гене MBL2, кодирующем лектин, связывающий маннозу, который ассоциирован с дефицитом MBL2 в сыворотке и предрасполагает к развитию аутоиммунных и инфекционных заболеваний и встречается в популяции с частотой 0.5 \%;
	\item у 4 пациентов (в том числе у 2 пациентов с АСЭ) rs56391007 в гене MET;
	\item у 3 пациентов (в том числе у 2 пациентов с АСЭ) rs72550870 в гене MASP2, кодирующем сериновую протеазу, дефицит которой приводит к аутоиммунным заболеваниям и хроническому воспалению и встречается в популяции с частотой 2.2 \%;
	\item у 1 пациента (в том числе у 1 пациента с АСЭ) rs121913407 в гене CTNNB1, приводящий к конститутивной активации $\upbeta$-катенина.
\end{itemize}

\begin{figure}[H]
	\centering
	\includegraphics[width=.8\textwidth]{pics/sun_ru.png}
	\caption{\textbf{Распределение всех генов всех исследуемых образцов по значению доли мажорного аллеля (MAF)}. 50 \% случаев -- преобладание MAF в неопухолевой ткани, 40 \% -- преобладание MAF в ГК, 10 \% -- значения MAF в опухоли и неопухолевой ткани одинаковы.}
	\label{fig:sun}
\end{figure}


На основании полученных данных для дальнейшей экспериментальной проверки был выбран герминативный вариант rs56391007 (c.C3029T) в гене MET, встречающийся в 0.793 \% популяции \cite{lek_analysis_2016} и приводящий к аминокислотной замене T1010I. Этот вариант был выявлен в трех случаях из основной выборки, в одном из которых содержащий его аллель, исходя из расчетов, проявлял АСЭ в ГК.

\begin{figure}[H]
	\centering
	\includegraphics[width=.8\textwidth]{pics/all_ase_onc_ru.png}
	\caption{\textbf{Распределение генов по числу случаев аллельного дисбаланса в опухоли и неопухолевой ткани печени}. Некоторые онкогены вынесены на рисунке (полный список в таблице \ref{table:onco}); в скобках указано число образцов с АСЭ в неопухолевой ткани и ГК.}
	\label{fig:all_ase}
\end{figure}

Ранее несколькими группами исследователей на различных модельных системах было описано влияние исследуемой замены на фосфорилирование белка MET \cite{schmidt_novel_1999, ma_c-met_2003, liu_functional_2015}, образование фокальных адгезионных контактов \cite{lee_novel_2000}, скорость приобретения клетками опухолевого фенотипа \cite{lee_novel_2000}, а также на изменение уровня миграции, инвазии и пролиферации клеток \cite{lee_novel_2000, liu_functional_2015}. В то же время, экспериментальных работ, исследующих функциональную роль MET T1010I в клетках культур гепатомы до настоящего момента опубликовано не было.

\subsection{Экспериментальная проверка предсказанного аллельного дисбаланса гена \textit{MET} с заменой C3029T}

Для проведения экспериментальной верификации наличия предсказанной АСЭ в гене MET с рассматриваемой заменой мы использовали метод цифровой капельной ПЦР (ddPCR) с зондами TaqMan. Были подобраны и синтезированы три флуоресцентных зонда – один, меченный флуорофором HEX, на общий участок, не содержащий замены, а также пара аллель-специфических зондов, меченных флуорофором FAM. Для подтверждения статуса герминативных вариантов была проведена серия ПЦР с ДНК, полученной из образцов основной (45 пар образцов) и дополнительной (14 пар образцов ГК и неопухолевой ткани печени) панелей. Для оценки числа копий аллелей производилась нормировка на количество копий аллелей в ДНК, выделенной из периферических мононуклеарных клеток крови гетерозиготного пациента. Для проверки соответствия расчетных данных об уровне дисбаланса фактическому уровню, а также для установления уровня дисбаланса в экспрессии аллелей MET пациентов из расширенной выборки, гетерозиготных по рассматриваемой позиции, была проведена ddPCR с кДНК этих пациентов.

На рисунке \ref{fig:example} представлены типовые результаты проведенных экспериментов. График показывает распределение капель по интенсивности флуоресценции в каналах FAM и HEX. На верхней панели представлены результаты для образца ткани пациента, гетерозиготного по рассматриваемой позиции, на нижней (отрицательный контроль) – гомозиготного по референсному варианту. Серый кластер соответствует каплям, свечение которых детектируется только в канале HEX и, таким образом, характеризует общее количество ампликонов MET, в то время как желтый кластер содержит капли, свечение которых наблюдается в обоих каналах. Таким образом, данный подход позволяет точно оценить представленность аллеля в ДНК и в экспрессии гена.

Для оценки полученных результатов было построено распределение расстояний между кластерами (типовые кластеры представлены на рисунке \ref{fig:example}) для образцов, в которых была выявлена замена (рисунок \ref{fig:hist}) – всего было обнаружено 5 пар образцов с вариантом c.C3029T. Расстояние 1-2 соответствует расстоянию между зеленым и желтым кластерами, расстояние 1-3 – между зеленым и серым кластерами, расстояние 2-3 – между желтым и серым.  Практически во всех случаях наблюдаются дискретные пики, за исключением альтернативного варианта кДНК. 

Наличие герминативного варианта MET C3029T было экспериментально подтверждено в 5 парах образцов, то есть в 7 \% случаев ГК из расширенной панели (OR$=9.85$; $95$ \% CI [$3.96$; $24.54$]). Предсказанный в одной паре образцов аллельный дисбаланс экспрессии MET был подтвержден экспериментально и стал следствием амплификации в опухолевой ткани потенциально патогенного аллеля, что подтверждено результатами цифровой капельной ПЦР с нормировкой на ДНК, выделенной из клеток периферической крови гетерозиготного по этому SNV пациента, с ДНК, выделенной из нормальной и опухолевой ткани пациента с аллельным дисбалансом MET. При исследовании расширенной панели был выявлен дополнительный случай АСЭ MET C3029T, не связанной с изменением числа копий аллеля. 

Таким образом, использованный нами подход позволяет выявить аллельный дисбаланс экспрессии герминативных SNV в отсутствие информации о гаплотипе. Неравновесный уровень экспрессии аллелей онкогенов, генов опухолевых супрессоров и печень-специфических генов был обнаружен в 21.8 \% исследованных генов. Было построено распределение значений статистически значимых MAF (FDR $< 5$ \%) для исследуемых генов. В 50 \% случаев ГК наблюдалось преобладание MAF в неопухолевой ткани, в 40 \% -- в опухоли (рисунок \ref{fig:sun}). Это свидетельствует об изменении уровня экспрессии аллелей генов при приобретении клеткой опухолевого фенотипа. Были обнаружены 28 генов, характеризующихся преобладанием частоты АСЭ в ГК по сравнению с неопухолевой тканью печени (рисунок \ref{fig:tumor}), в то время как всего для 4 генов было показано преобладание частоты АСЭ в неопухолевой ткани. Таким образом, можно предположить, что при трансформации клетки наблюдается усиление АСЭ генов, демонстрирующих аллельный дисбаланс до приобретения клеткой опухолевого фенотипа.

В ходе экспериментальной проверки наличия АСЭ MET c c.C3029T в основной и дополнительной панелях была подтверждена АСЭ гена – среди 5 пар образцов с MET c.C3029T было обнаружено 2 случая ГК с АСЭ этого варианта в опухолевой ткани (3.3 \% случаев). Учитывая распространенность рассматриваемого варианта среди индивидов с другими типами опухолей и наличие данных о его функциональности в некоторых модельных системах, наличие герминативного варианта MET c.C3029T в 8 \% случаев ГК нашей выборки и аллельный дисбаланс в экспрессии этого варианта могут рассматриваться как потенциальный фактор инициации и/или прогрессии ГК. В дальнейшем планируется провести исследование влияния рассматриваемого варианта на биологические свойства клеток путем экзогенной экспрессии мутантного MET в клетках культур гепатомы человека HepG2 и Huh7. Кроме того, будет проверена чувствительность клеток культур гепатомы, экспрессирующих MET c.C3029T, к воздействию ингибитора рецепторов тирозинкиназ кабозантиниба, обладающего селективностью в отношении MET и недавно одобренного для терапии ГК, для выяснения целесообразности назначения такой терапии пациентам, являющимся носителями этого варианта.

\begin{figure}[H]
	\centering
	\includegraphics[width=\textwidth]{pics/new_scatter_stitch_ru.png}
	\caption{\textbf{Распределение капель по интенсивности флуоресценции в каналах FAM и HEX}. Верхняя и нижняя панели -- образцы ткани пациентов, гетеро- и гомозиготного по C3029T \textit{MET} соответственно. Серый кластер -- капли, свечение которых детектируется только в канале HEX, зеленый кластер -- капли, флуоресценция которых не наблюдается ни в одном из каналов, желтый кластер -- дважды положительный кластер. На графиках представлены распределения только для геномной ДНК.}
	\label{fig:example}
\end{figure}

\begin{figure}[H]
	\centering
	\includegraphics[width=\textwidth]{pics/hist_stitch_ru.png}
	\caption{\textbf{Распределение расстояний между кластерами для образцов с заменой C3029T \textit{MET}}. Расстояние 1-2 соответствует расстоянию между зеленым и желтым кластерами, расстояние 1-3 -- между зеленым и серым, расстояние 2-3 -- между желтым и серым (рисунок \ref{fig:example}).}
	\label{fig:hist}
\end{figure}

\newpage

\section{Выводы}

\begin{enumerate}
	\item При помощи описанного подхода, включающего набор инструментов GATK и алгоритм MBASED, в основной панели, состоящей из 45 парных образцов ГК и неопухолевой ткани печени, было обнаружено 13137 однонуклеотидных вариантов. Cреди 572 генов, включающих в себя онкогены, опухолевые супрессоры и печень-специфические гены, обнаружено 125 генов, экспрессия которых носит аллель-специфический характер;
	\item Медианные значения долей мажорных аллелей в экспрессии генов в неопухолевой и опухолевой тканях демонстрируют высокий уровень корреляции (коэффициент корреляции Спирмена=$0.85$, p-value=$7.6\times10^{-48}$). При этом на уровне отдельных генов и образцов возможны различия в уровне аллельного дисбаланса. Так, в 50 \% случаев в ГК наблюдалось преобладание экспрессии нереференсного аллеля в неопухолевой ткани, в 40 \% -- в опухоли. Таким образом, АСЭ может приводить к преимущественной экспрессии потенциально патогенных вариантов;
	\item В результате поиска потенциально патогенных вариантов в АСЭ генах выявлен и подтвержден методом цифровой капельной ПЦР встречающийся у 8 \% пациентов с ГК герминативный вариант MET c.C3029T. Частота АСЭ MET c.C3029T составляет 3.3 \% для опухолевой и прилежащей тканей печени, что, в совокупности с его предполагаемой функциональной значимостью, может способствовать инициации и/или прогрессии ГК. Кроме того, у пациентов с ГК в АСЭ генах были выявлены другие потенциально патогенные варианты, ассоциированные с развитием аутоиммунных заболеваний, хронической инфекции и воспалением.
\end{enumerate} \newpage
        \bibliographystyle{mybib.bst}  
        \bibliography{mybib.bib}
\end{document}