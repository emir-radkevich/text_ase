\documentclass[12pt]{article} %font size, document class

%-------------------LANGUAGE-------------------%
\usepackage[utf8]{inputenc} %translates input encodings into "LATEX internal encoding"
\usepackage[T2A]{fontenc} %font encoding; T2A -- enough for cyrillic and english, but if umlauts or similar stuff are used it's recommended to use T1 either
\usepackage[english,russian]{babel} %language-specific hyphenation, easy use of special symbols
%\usepackage{float} %для плавающих картинок и таблиц

%-------------------STUFF-------------------%
\usepackage{cmap}  %search in pdf enabled
\usepackage[pdftex]{graphicx} %works with graphics 
 

%-------------------PAPER_AND_TEXT-------------------%
\usepackage{vmargin} %page layout
\setpapersize{A4} %paper format
\setmarginsrb{30mm}{25mm}{25mm}{25mm}{0pt}{0mm}{0pt}{13mm} %paper field sizes: left, up, right, down, 3*колонтитулы, расстояние между нижним краем нижней строки и нижним краем номера страницы
\usepackage[hidelinks]{hyperref}
\usepackage{setspace}
\usepackage{indentfirst} %красная строка для первого абзаца главы или параграфа
\pagestyle{plain} %включена нумерация страниц
%\sloppy %борьба с залезанием строк на поля путём изменения размеров пробелов
%\usepackage{amsmath} %дополнительные средства для вёрстки формул
%\everymath{\displaystyle}
%\usepackage{breqn} %для dmath разбить длинную формулу
%\usepackage{amsfonts} %дополнительные шрифты для формул
%\usepackage{amssymb} %дополнительные символы для формул
%\usepackage[nottoc,notlot,notlof]{tocbibind}
 
%\renewcommand{\thesection}{\arabic{section}} %одинарная нумерация формул


%\usepackage{verbatim} %comments

%\usepackage[font=small]{caption}
%\usepackage{booktabs} %отступы в tabular
%\usepackage{colortbl} %раскаршивание таблиц
%\usepackage{xcolor} %название цветов


%\usepackage{lineno} %нумерация всех строк для отладки
%\usepackage{enumitem} %особенности enumerate
%\usepackage{pbox} % для переносов внутри ячейки таблицы
%\usepackage{array} %??
%\usepackage{longtable}% перенос таблиц на страницах

%\newlength{\width}
%\setlength{\width}{0.97\textwidth}
%\setlength{\parindent}{3em}
%\setlength{\parskip}{0.3em}

%\usepackage{textgreek} % \textalpha греческие буквы не в math mode
%\usepackage{subcaption} % несколько картинок в одной
%\usepackage{multirow} % объединение ячеек
%\setcounter{tocdepth}{2} % глубина содержания


%-------------------CITE-------------------%
\usepackage{cite}

\begin{document}
        \begin{titlepage}

\newcommand{\HRule}{\rule{\linewidth}{0.3mm}} % Defines a new command for the horizontal lines, change thickness here

\center

\textbf{Московский государственый университет имени М.В.Ломоновсова}
\\[0.3cm] 
\HRule 
\\[0.3cm]
\textbf{\textsc{\large Факультет биоинженерии и биоинформатики}}
\\[3.0cm]

\begin{spacing}{1.4}
{\LARGE \textbf{Identification of potentially pathogenic single nucleotide polymorphisms in allele-specifically expressed genes during hepatocarcinogenesis}} \\[1.0cm]


{\LARGE \textbf{Идентификация потенциально патогенных однонуклеотидных полиморфизмов в аллель-специфически экспрессирующихся генах при гепатоканцерогенезе}} \\[3.0cm]
\end{spacing}
 
 
\Large \textit{Курсовая работа студента четвертого курса:}\\
Радкевич Эмира Мустафовича
\\[1.5cm]


\begin{flushright} \large
Научные руководители: \\
проф.,\,д.б.н.\,Лазаревич\,Н.Л. \\
аспирант\,Кривцова\,О.М.
\\[1.5cm]
\end{flushright}


{\large Москва \\ 2019}


\end{titlepage} \newpage
        \tableofcontents \newpage
        \section{Список сокращений}

\begin{enumerate}
	\footnotesize
	\item АСЭ -- аллель-специфическая экспрессия
	\item SNV -- однонуклеотидный вариант
	\item SNP -- однонуклеотидный полиморфизм
	\item ГК -- гепатокарцинома
	\item HBV -- вирус гепатита B человека
	\item HCV -- вирус гепатита C человека
	\item ЭР -- эндоплазматический ретикулум, или эндоплазматическая сеть
	\item HBsAg -- антиген вируса гепатита B
	\item ЮД -- юкстамембранный домен 
	\item МДС -- мультифункциональный домен связывания
	\item PtdIns3P -- фосфатидилинозитол-трифосфат
	\item \textit{APC} -- аденоматозный полипоз кишечника (adenomatous polyposis coli gene)
	\item \textit{MSH2} -- гомолог 2 белка MutS (MutS protein homolog 2 gene)
	\item \textit{TP53} -- опухолевый белок 53 (tumor protein 53 gene)
	\item \textit{MLL4} -- смешанная лейкемия 4 (mixed-lineage leukemia 4 gene)
	\item \textit{CCNE1} -- циклин E1 (cyclin E1 gene)
	\item \textit{MET} -- рецептор фактора мезенхимально-эпителиального перехода ( mesenchymal-epithelial transition factor receptor gene) 
	\item \textit{CTNNB1} -- $\upbeta$-катенин (catenin $\upbeta$1 gene)
	\item \textit{CDKN2A} -- циклин-зависимый киназный ингибитор 2A (cyclin-dependent kinase inhibitor 2A gene)
	\item \textit{RB1} -- ретинобластома 1 (retinoblastoma 1 gene)
	\item \textit{PKM2} -- изозимы пируват киназы M1/M2 (pyruvate kinase isozymes M1/M2 gene)
	\item \textit{ALB} -- альбумин (albumin gene)
	\item \textit{APOB} -- аполипопротеин B (apolipoprotein B gene)
	\item \textit{OSGIN1} -- ростовой фактор, индуцированный оксилительным стрессом (oxidative stress-induced growth factor 1 gene)
	\item \textit{IGF2} -- инсулиноподобный ростовой фактор 2 (insulin-like growth factor 2 gene)
	\item HGF -- гепатоцитарный ростовой фактор (hepatocyte growth factor)
	\item TERT -- обратная транскриптаза полимеразы (telomerase reverse transcriptase)
	\item PDGF$\upbeta$ -- тромоцитарный фактор роста $\upbeta$ (platelet-derived growth factor$\upbeta$)
	\item EGFR -- рецептор эпидермального фактора роста (epidermal growth factor receptor)
	\item IGFR -- рецептор инсулиноподобного фактора роста (insulin-like growth factor receptor)
	\item TNF$\upalpha$ -- фактор некроза опухоли $\upalpha$ (tumor necrosis factor$\upalpha$)
	\item IFN$\upalpha$ -- интерферон $\upalpha$ (interferon$\upalpha$)
	\item HRAS -- Харви Ras белок (Harvey Ras protein)
	\item miR-122 -- микроРНК-122 (microRNA-122)
	\item MDM4 -- E3 убиквитин-лигаза Mdm4 (E3 ubiquitin-protein ligase Mdm4)
	\item PI3K -- фосфоинозитид-3-киназы (phosphoinositide 3-kinases)
	\item GRB2 -- белок 2, связывающийся с рецептором ростового фактора (growth factor receptor-bound protein 2)
	\item SHP2 или PTPN11 -- фосфатаза нерецепторного типа 11 (tyrosine-protein phosphatase non-receptor type 11)
	\item SHC -- домен Src гомологии 2, содержащий трансформирующий белок 1 (Src homology 2 domain containing transforming protein 1)
	\item PLC -- фосфолипаза C (phospholipase C protein)
	\item STAT3 -- трансдуктор сигнала и активатор транскрипции 3 (signal transducer and activator of transcription 3)
	\item GAB1 -- GRB2-связывающий белок 1 (GRB2-associated-binding protein 1)
	\item MAPK -- митоген-активируемые протеин киназы (mitogen-activated protein kinases)
	\item SOS -- son of sevenless protein
	\item ERK --  внеклеточные сигнал-регулируемые киназы (extracellular signal-regulated kinases)
	\item p120-Ras-GAP -- p120 Ras ГТФаза активирующий белок (p120 Ras GTPase activating protein)
	\item JNK -- c-Jun N-концевые киназы (c-Jun N-terminal kinases)
	\item AKT, или PKB -- протеин киназа B (protein-kinase B)
	\item IKK -- I$\upkappa$B киназа (I$\upkappa$B kinase)
	\item I$\upkappa$B$\upalpha$ -- NF$\upkappa$B ингибитор $\upalpha$ (NF$\upkappa$B inhibitor$\upalpha$)
	\item NF$\upkappa$B -- ядерный фактор энхансера гена $\upkappa$ легкого полипептида в В-клетках (nuclear factor of $\upkappa$ light polypeptide gene enhancer in B-cells)
	\item ADAM -- дезинтегрин и металлопротеиназа домен-содержащие белки (a disintegrin and metalloproteinase domain-containing protein)
	\item CBL -- E3 убиквитин лигаза CBL (E3 ubiquitin-protein ligase casitas B-lineage lymphoma)
\end{enumerate} \newpage
        \section{Введение}

В геноме человека выявляется множество однонуклеотидных герминативных вариантов (мутаций, возникающие в половых клетках и передаваемых по наследству), в том числе от 459 до 565 тысяч вариантов в регуляторных областях, включая промоторы, энхансеры, инсуляторы, сайты связывания транскрипционных факторов и нетранслируемые области, что может обусловливать межиндивидуальные различия в уровнях экспрессии генов, и от 10 до 12 тысяч миссенс-вариантов в белок-кодирующих участках \cite{1000_genomes_project_consortium_global_2015}. Среди последних у индивидов обнаруживается в среднем 24-30 известных патогенных вариантов, ассоциированных с развитием наследственных заболеваний \cite{1000_genomes_project_consortium_global_2015}, которые, однако, не приводят к их манифестации.

В большинстве случаев экспрессия гена осуществляется в равной степени с обеих гомологичных хромосом. Тем не менее, в последние годы появился ряд работ, результаты которых свидетельствуют о распространенности аллель-специфической экспрессии (АСЭ), то есть неравновесного уровня экспрессии аллелей, выявляемого у 0,5–30 \% генов \cite{gaur_research_2013, lappalainen_transcriptome_2013,li_detection_2013}, и о ее роли в этиологии некоторых наследственных заболеваний, включая наследственно обусловленные формы опухолей \cite{galiatsatos_familial_2006, chan_heritable_2006, buzby_allele-specific_2017}.

Так, ранее была показана связь между пониженной экспрессией одного из аллелей гена опухолевого супрессора \textit{APC} и предрасположенностью к развитию семейного аденоматозного полипоза \cite{galiatsatos_familial_2006}. Также, было показано, что аллель-специфическое метилирование промотора гена \textit{MSH2}, отвечающего за репарацию ошибочно спаренных нуклеотидов, может приводить к развитию синдрома Линча \cite{chan_heritable_2006}. В рассмотренных случаях АСЭ приводила к снижению уровня экспрессии генов, кодирующих опухолевые супрессоры. Помимо этого, АСЭ способна влиять на экспрессию аллелей, содержащих патогенные герминативные варианты. Например, показана роль АСЭ в развитии синдрома Ли-Фраумени у членов семьи, являвшихся носителями функционального герминативного варианта в гене \textit{TP53}, обусловливающего фосфорилирование p53 по серину 15, которое определяет ответ на генотоксический стресс \cite{buzby_allele-specific_2017}. Кроме того, АСЭ выявляют в различных спорадических опухолях. Так, Милани и соавторы идентифицировали в промоторах генов, нарушение экспрессии которых ассоциировано с развитием опухолей, цис-регуляторные полиморфизмы, способствующие аллель-специфическому связыванию транскрипционных факторов \cite{milani_allelic_2007}.

Гепатоцеллюлярная карцинома (ГК) является наиболее распространенной формой первичных опухолей печени \cite{farazi_hepatocellular_2006}. Несмотря на активно проводимые исследования профиля молекулярных нарушений при ГК \cite{cancer_genome_atlas_research_network._electronic_address:_wheelerbcm.edu_comprehensive_2017}, на сегодняшний день изучению герминативных вариантов, которые могут предрасполагать к развитию ГК или способствовать ее прогрессии, уделяется недостаточное внимание. Распространенность, паттерн и возможная роль АСЭ в гепатоканцерогенезе также остаются малоисследованными. ГК свойственна высокая гетерогенность соматических нарушений \cite{cancer_genome_atlas_research_network._electronic_address:_wheelerbcm.edu_comprehensive_2017}, не позволяющих исчерпывающим образом установить молекулярные механизмы, лежащие в основе ее формирования и прогрессии \cite{lawrence_discovery_2014}. Таким образом, выявление полиморфизмов, обусловливающих нарушения аллельной экспрессии при ГК, а также последующая экспериментальная проверка их роли в приобретении опухолевыми клетками более злокачественного фенотипа позволит оценить значимость АСЭ как дополнительного фактора опухолевой прогрессии.

Целью настоящей работы является оценка возможности выявления АСЭ в ГК на основе данных транскриптомного секвенирования и проверка гипотезы о возможном вкладе АСЭ в процесс гепатоканцерогенеза. Для этого планируется исследовать несколько групп генов, с высокой вероятностью способных оказывать влияние на злокачественную трансформацию клеток печени человека, а именно онкогенов, опухолевых супрессоров и печень-специфических генов, в том числе за счет повышенного уровня экспрессии аллелей, содержащих патогенные герминативные варианты.

Для достижения этой цели мы поставили перед собой следующие задачи:
\begin{enumerate}
	\item провести поиск АСЭ среди генов, кодирующих онкогены, опухолевые супрессоры и печень-специфические гены на основании данных транскриптомного секвенирования 45 парных образцов ГК и неопухолевой ткани печени;
	\item выявить потенциально патогенные однонуклеотидные полиморфизмы (SNP) в АСЭ генах;
	\item провести экспериментальную верификацию АСЭ отдельных вариантов независимыми методами для проверки корректности предсказаний о дисбалансе на основе выбранного нами алгоритма.
\end{enumerate}
        \section{Обзор литературы}

\subsection{Гепатокарцинома}

Согласно данным GLOBOCAN за 2018 год, рак печени занимает 6 место в мире среди всех типов рака человека по количеству новых случаев и 4 – по количеству летальных исходов \cite{bray_global_2018} (рисунок \ref{fig:statistics}). Высокая смертность от рака печени отчасти связана с поздними сроками его выявления ввиду бессимптомного течения у части пациентов \cite{befeler_hepatocellular_2002}, устойчивостью к применяемым в терапии противоопухолевым препаратам, а также обусловленными наличием сопутствующих заболеваний печени ограничениями на выбор терапии. В случаях локализованного заболевания хирургическая резекция опухоли представляет собой эффективную терапевтическую альтернативу (особенно в условиях отсутствия цирроза печени). Несколько альтернативных методов лечения, таких как инъекция этанола или применение радиочастотной абляции, также используются в случаях локализованного заболевания. В редких случаях применим радикальный подход – пересадка печени, однако лишь в тех случаях, когда пациенты отвечают строго определенным критериям \cite{farazi_hepatocellular_2006}.

\begin{figure}[H]
\captionsetup[subfigure]{justification=centering}
\centering
	\begin{subfigure}{0.49\textwidth}
		\includegraphics[width=\linewidth]{pics/num_inc.png} 
%		\centering
		\caption{Количество летальных исходов от различных типов рака в 2018 году}
		\label{fig:inc}
	\end{subfigure}
	\begin{subfigure}{0.49\textwidth}
		\includegraphics[width=\linewidth]{pics/num_mort.png}
		\caption{Количество новых случаев различных типов рака в 2018 году}
		\label{fig:mort}
	\end{subfigure}
	\caption{\textbf{Статистика раковых заболеваний согласно данным GLOBOCAN 2018} \cite{bray_global_2018}. На рисунках \ref{fig:inc} и \ref{fig:mort} внутренний кружок отражает распределение по полам (розовый -- женщины, фиолетовый -- мужчины). Значения, изображенные на рисунках, приведены в процентах.}
	\label{fig:statistics}
\end{figure}

Рак печени включает в себя разнообразные гистологически отличимые первичные новообразования печени, которые включают гепатоцеллюлярную карциному (ГК), внутрипеченочную карциному желчного протока (холангиокарцинома), гепатобластому, цистаденокарциному желчного протока, гемангиосаркому и эпителиоэдиомеатому. Среди всех типов рака печени ГК является наиболее распространенной формой – она составляет примерно 80-85 \% всех случаев \cite{farazi_hepatocellular_2006}. 

\subsubsection{Этиология гепатокарциномы}

Несмотря на то, что ГК распространена по всему миру, существуют региональные различия в частоте и механизмах возникновения ГК, обусловленные специфическими этиологическими факторами. Наиболее значимыми факторами рискавозникновения ГК являются хроническая вирусная инфекция вирусами гепатита В или С, хроническое употребление алкоголя, употребление пищи, загрязненной афлатоксином B\textsubscript{1}, а также большинство заболеваний, вызывающих цирроз печени \cite{farazi_hepatocellular_2006}.

\paragraph{ГК, вызванная инфекцией вируса гепатита B}\mbox{}\\

Вирус гепатита B человека (HBV) – вирус с частично двухцепочечной молекулой ДНК, кодирующей несколько вирусных белков, необходимых для его жизненного цикла: обратная транскриптаза/ДНК-полимераза, капсидный белок, известный как коровый антиген гепатита B (HBsAg), белок X (HBx), и белки оболочки L, M и S, которые связываются с мембраной эндоплазматического ретикулума (ЭР) \cite{farazi_hepatocellular_2006}. 

В HBV-опосредованной опухолевой трансформации клетки задействовано несколько механизмов. ДНК HBV способна интегрировать в геном, при этом сайты интеграции преимущественно выявляются в участках ДНК человека, подверженных разрывам (в так называемых ломких сайтах хромосом) и в CpG-богатых участках. Кроме того, в геноме человека выделяют ряд генов, преимущественно представленных онкогенами и опухолевыми супрессорами, в том числе генов обратной транскриптазы теломеразы \textit{TERT}, \textit{MLL4}, \textit{CCNE1}, интеграция HBV в которые в опухолях, вероятно, ввиду получения клетками с измененным уровнем экспрессии этих генов селективных преимуществ, наблюдается с более высокой частотой по сравнению другими сайтами интеграции. Таким образом, интеграция генома HBV способна приводить к повышению геномной нестабильности, причем как к крупным перестройкам, так и к микроделециям в ДНК хозяина, возникающим в результате рекомбинации, и оказывать влияние на уровень экспрессии генов \cite{tokino_chromosome_1991, zhao_genomic_2016}. Во-вторых, белок HBx способен напрямую взаимодействовать с компонентами многих сигнальных каскадов, таких как Src тирозинкиназы, RAS/ERK, JNK, NF$\upkappa$B, Wnt/$\upbeta$-catenin \cite{feitelson_genetic_2002}. Также, HBx может связывать и инактивировать опухолевый супрессор p53, что способствует повышению пролиферации и выживаемости клеток \cite{feitelson_genetic_2002}. Наконец, HBx способен фосфорилировать и инактивировать опухолевый супрессор рRb, подавляя таким образом контроль над клеточным циклом и системы репарации ДНК \cite{arzumanyan_pathogenic_2013}.

При развитии вирусной инфекции иммунный ответ на постоянную репликацию вируса способствует развитию гепатита – происходит некроз ткани с последующей регенерацией и/или фиброзом печени. Во время регенерации ДНК вируса HBV все чаще интегрируются в ДНК хозяина, что приводит к дестабилизации генома и иным последствиям, в частности, к повышению продукции HBx. Вирус способствуют пролиферации и выживанию инфицированных клеток во время хронического заболевания печени и активирует несколько перекрывающихся сигнальных путей (например, RAS, PI3K, рецептор эпидермального фактора роста (EGFR) и рецептор инсулиноподобного фактора роста 1 (IGFR1)), которые способствуют злокачественной трансформации клеток \cite{arzumanyan_pathogenic_2013}. Другой предполагаемый механизм гепатоканцерогенеза, вызванного HBV,  связан с накоплением белков вируса в эндоплазматическом ретикулуме (ЭР), что провоцирует стресс ЭР \cite{choi_naturally_2019} и, в конечном счете, индукцию окислительного стресса, который может активировать пути передачи сигналов, стимулирующие пролиферацию и выживание клеток, возникновение мутаций за счет повышения уровня свободных радикалов и активировать звездчатые клетки печени. Таким образом, HBV может способствовать инициации ГК посредством множества механизмов \cite{farazi_hepatocellular_2006}.

\paragraph{ГК, вызванная инфекцией вируса гепатита C}\mbox{}\\

Вирус гепатита C человека (HCV) – вирус, геном которого представлен положительной цепью молекулы РНК, кодирующей неструктурные белки (NS2, NS3, NS4A, NS5A и NS5B), вирусную репликазу (коровый белок), ионный канал, а также гликопротеины E1 и E2 \cite{arzumanyan_pathogenic_2013}.

Патогенез ГК, ассоциированный с инфекцией HCV, имеет ряд отличительных особенностей по сравнению с процессом развития ГК на фоне инфекции HBV. Во-первых, HCV демонстрирует более высокую склонность к возникновению хронической инфекции – в хроническую стадию переходит около 10 \% случаев HBV против 60-80 \% HCV. Это может быть связано со способностью квази-видов HCV, геном которых содержит большое количество мутаций, обходить иммунную систему хозяина. Второе ключевое отличие заключается в более высокой частоте развития цирроза печени у пациентов с инфекцией HCV. Примерно у 10 \% пациентов с HCV после 10 лет инфицирования развивается цирроз печени, что примерно на порядок выше, чем при инфекции HBV. В-третьих, поскольку HCV является РНК-вирусом без промежуточной формы ДНК, он не может интегрироваться в геном хозяина \cite{farazi_hepatocellular_2006}.

Так же, как и в случае с инфекцией HBV, непрерывные циклы гибели гепатоцитов, вызванные иммунным ответом на HCV и последующей регенерацией, приводят к накоплению мутаций. Коровый белок HCV и неструктурный белок NS5A участвуют в процессе избегания иммуно-опосредованного уничтожения клеток путем взаимодействия с различными факторами, вовлеченными в этот процесс, – фактором некроза опухоли (TNF$\upalpha$), интерферонами (IFN$\upalpha$) \cite{farazi_hepatocellular_2006}. Кроме того, белки HCV NS3 и NS4A используют свою протеазную активность для расщепления и активации компонентов, которые участвуют в передаче сигнала иммунного ответа \cite{li_immune_2005}.

HCV, как и HBV, связывается с ЭР, вызывая тем самым ЭР стресс. Кроме того, было показано, что коровые белки HCV взаимодействуют с компонентами сигнального пути MAPK (такими как ERK, MEK и RAF) и, таким образом, способствуют пролиферации клеток \cite{macdonald_hepatitis_2003}. При этом при инфекции HCV наблюдается накопление жирных кислот и повышенное образование активных форм кислорода \cite{moriya_oxidative_2001}. Также было показано, что NS5A взаимодействует с p53 и инактивирует его путем секвестрирования в околоядерное пространство, влияя, таким образом, на сигнальные пути, регулируемые p53 \cite{farazi_hepatocellular_2006}. 

\paragraph{ГК, вызванная хроническим употреблением алкоголя}\mbox{}\\

Употребление алкоголя является важным фактором риска возникновения ГК. При хроническом употреблении алкоголя в результате активации моноцитов образуются провоспалительные цитокины \cite{mcclain_monocyte_2002}. В условиях хронического воздействия этанола гепатоциты проявляют повышенную чувствительность к цитотоксическим эффектам TNF$\upalpha$, что, с одной стороны, индуцирует гибель гепатоцитов, с другой - индуцирует регенерацию печени, активацию звездчатых клеток и продукцию ими компонентов внеклеточного матрикса и развитие цирроза, что, в конечном итоге, создает условия для развития ГК \cite{farazi_hepatocellular_2006}.

Алкоголь также повреждает печень путем индукции окислительного стресса, который способствует развитию ГК несколькими путями. Во-первых, окислительный стресс способствует развитию фиброза и цирроза. Про-канцерогенный эффект цирротического микроокружения был продемонстрирован на трансгенных мышах, гиперэкспрессирующих PDGF$\upbeta$, у которых развивался фиброз, способствующий возникновению ГК \cite{campbell_platelet-derived_2005}. Поскольку звездчатые клетки являются основным источником коллагена в поврежденной печени, индукция окислительного стресса культивируемых звездчатых клеток может способствовать увеличению пролиферации клеток и синтезу коллагена \cite{farazi_hepatocellular_2006}. Во-вторых, вызванный этанолом окислительный стресс может вызывать изменения сигнальных путей, связанных с возникновением ГК – снижение уровня фосфорилирования STAT1, снижение STAT1-направленной активации передачи сигналов IFN$\upalpha$ и потерю защитных эффектов IFN$\upalpha$ с последующим повреждением гепатоцитов. Окислительный стресс может также вызвать накопление онкогенных мутаций. Так, была описана ассоциация окислительного стресса при нарушении гомеостаза железа с повышением частоты мутаций в гене р53 при ГК \cite{marrogi_oxidative_2001}.

\paragraph{ГК, вызванная афлатоксином B\textsubscript{1}}\mbox{}\\

Употребление микотоксина -- афлатоксина B\textsubscript{1} -- также представляет повышенный риск развития ГК. Афлатоксин B\textsubscript{1}, по-видимому, функционирует как мутаген и связан со специфической мутацией р53, а также с активирующей мутацией в HRAS \cite{ozturk_p53_1991, riley_vitro_1997}. В отличие от ГК, вызванной HCV и алкоголем, четкой связи между воздействием афлатоксина B\textsubscript{1} и развитием цирроза не выявлено, что указывает на то, что основным фактором развития ГК в этом случае может быть мутагенное воздействие этого токсина. В некоторых регионах воздействие афлатоксина B\textsubscript{1} часто сочетается с инфекцией HBV, что повышает риск развития ГК в более раннем возрасте \cite{kew_synergistic_2003}.

\subsubsection{Генетические и эпигенетические события при ГК}

Молекулярный анализ 196 образцов ГК человека позволил выявить множество генетических и эпигенетических изменений в ключевых онкогенах и опухолевых супрессорах. К таким изменениям относятся точечные соматические мутации в часто мутирующих генах (\textit{\textit{TP53}}, \textit{CTNNB\textsubscript{1}} (кодирует $\upbeta$-катенин), промотор \textit{TERT}), изменение числа копий аллелей гена (\textit{CDKN2A}, \textit{MET}, \textit{RB\textsubscript{1}}), активация экспрессии генов при интеграции генома вируса гепатита B (\textit{TERT}), а также гиперметилирование промоторов (\textit{CDKN2A}). Кроме того, была обнаружена высокая частота мутаций в генах, кодирующих компоненты сигнальных путей WNT, SHH (Sonic Hedgehog), RTK/RAS \cite{cancer_genome_atlas_research_network._electronic_address:_wheelerbcm.edu_comprehensive_2017}.

Наиболее распространенными соматическими мутациями, обнаруженными в 44 \% исследуемых образцов, были мутации в промоторе \textit{TERT}. Пациенты с мутацией промотора \textit{TERT} были как правило старше (р=$0.0006$), преимущественно мужского пола (р=$0.006$), с большей вероятностью были инфицированы HCV (р=$0.04$) и реже были инфицированы HBV (р=$0.02$), чем пациенты без мутации. В исследуемых образцах наблюдался повышенный уровень экспрессии \textit{TERT}, однако корреляции между наличием мутации в промоторе гена и уровнем экспрессии выявлено не было \cite{cancer_genome_atlas_research_network._electronic_address:_wheelerbcm.edu_comprehensive_2017}.

Описана корреляция между наличием мутации в промоторе \textit{TERT} и инактивацией \textit{CDKN2A} за счет гиперметилирования промотора. \textit{CDKN2A} кодирует опухолевый супрессор p16\textsuperscript{INK4A}. Подавление экспрессии p16\textsuperscript{INK4A} в сочетании с повышенной экспрессией \textit{TERT} приводит к приобретению эпителиальными клетками опухолевого фенотипа, которое обусловлено повышением теломеразной активности и увеличением уровня пролиферации \cite{kiyono_both_1998}. У 31 \% пациентов были найдены мутации в \textit{TP53}. По-видимому, наличие мутаций не связано с низким уровнем экспрессии \textit{TP53} – только в одном образце с высокой экспрессией \textit{TP53} выявлена мутация в этом гене, в то время как в 23 \% образцов с низкой экспрессией \textit{TP53} ген не содержал ни одной мутации. Такой эффект, по-видимому, объясняется активностью ингибиторов p53, например MDM4, экспрессия которого была значительно повышена \cite{cancer_genome_atlas_research_network._electronic_address:_wheelerbcm.edu_comprehensive_2017}. С другой стороны, было показано, что большинство мутаций \textit{TP53} в опухолевых клетках являются миссенс-мутациями, затрагивающими примерно 190 кодонов, причем 8 наиболее частых вариантов составляют около 28 \% от всех встречающихся мутаций. 7 из 8 горячих точек мутаций приходятся на CpG динуклеотиды \cite{baugh_why_2018}.

В некоторых IDH1/2 мутантных опухолях выявлено снижение экспрессии микроРНК-122 (miR-122), которое может быть связано с гиперметилированием промотора этого гена \cite{cancer_genome_atlas_research_network._electronic_address:_wheelerbcm.edu_comprehensive_2017}. miR-122 является преобладающей в клетках печени, где выполняет функцию регулятора метаболизма жирных кислот \cite{liu_mir-122_2014}. Мишенями miR-122 являются белки, контролирующие адгезию к субстрату, клеточный рост, пролиферацию – дезинтегрин/металлопротеиназа домен-содержащие белки (ADAM10/ADAM17), рецептор инсулин-подобного ростового фактора (IGFR1) и циклин G1 (CCNG1).

Мутации в гене альбумина (\textit{ALB}) и гене аполипопротеина B (\textit{APOB}) выявлены в 13 \% и 10 \% ГК, соответственно. Кроме того, описано снижение уровня экспрессии \textit{ALB} и \textit{APOB} в ГК относительно нормальной ткани печени. Описана связь между снижением экспрессии этих генов и повышенным уровнем пролиферации и синтеза нуклеотидов \cite{cancer_genome_atlas_research_network._electronic_address:_wheelerbcm.edu_comprehensive_2017}.

Помимо характерных для ГК точечных соматических мутаций, изменений в копийности аллелей генов и в профиле метилирования, в литературе описаны эффекты сегрегации хромосом при ГК. Нарушение сегрегации хромосом во время митоза приводят к анеуплоидии, распространенной цитогенетической особенности разных типов опухолевых клеток, в том числе ГК. Белок Aurora kinase A и мишень этого белка, протеин киназа HURP,играющие важную роль в сегрегации хромосом,– гиперэкспрессированы в ГК \cite{farazi_hepatocellular_2006}.

\subsubsection{Роль аллель-специфической экспрессии при ГК}

В последние годы появился ряд работ, результаты которых свидетельствуют о роли неравновесного уровня экспрессии аллелей генов, или аллель-специфической экспрессии (АСЭ), в этиологии некоторых наследственно обусловленных форм опухолей \cite{galiatsatos_familial_2006, chan_heritable_2006, buzby_allele-specific_2017}. К настоящему моменту роль АСЭ в инициации и прогрессии ГК малоизучена, однако было описано влияние аллельного дисбаланса гена ингибитора роста, индуцированного оксидативным стрессом 1 (\textit{OSGIN1}), на прогрессию ГК \cite{liu_allele-specific_2014}. \textit{OSGIN1} является опухолевым супрессором, играющим важную роль в регуляции клеточной гибели. При возникновении замены c.G1494A в гене \textit{OSGIN1} экспрессируемый белок утрачивает способность к транслокации из ядра в митохондрию, что приводит к утрате его про-апоптотической функции. При этом в опухолевых клетках нередко наблюдается аллельный дисбаланс экспрессии этого гена ввиду потери гетерозиготности. Кроме изменения в локализации белка, обусловленной наличием однонуклеотидной замены, менялся характер экспрессии аллелей гена \textit{OSGIN1} в ГК – происходило уменьшение уровня экспрессии. При этом пациенты со сниженным в два и более раза уровнем экспрессии характеризовались сниженной общей и безрецидивной выживаемостью. Общая выживаемость в группе пациентов с рассматриваемым герминативным вариантом была наиболее низкой по сравнению с группами с нормальной и сниженной экспрессией. Кроме того, в опухолях пациентов, являющихся носителями рассматриваемого варианта, апоптотический индекс был ниже, чем в ГК других пациентов \cite{liu_allele-specific_2014}.

На ранних стадиях ГК была описана АСЭ гена инсулин-подобного ростового фактора 2 (\textit{IGF2}), а также повышенный уровень экспрессии этого гена \cite{takeda_allelic-expression_1996}. АСЭ гена \textit{IGF2} наблюдалась в опухолевой ткани, а также при дисплазии, причем уровень экспрессии гена и соотношение экспрессии его аллелей изменялись в зависимости от уровня дифференцировки клеток. Хорошо дифференцированные диспластические узелки характеризовалась значительным увеличением уровня экспрессии \textit{IGF2}, в то время как на ранней стадии развития ГК уровень экспрессии гена был незначительно выше, чем в неопухолевой ткани печени. В то же время, аллельный дисбаланс чаще наблюдался в умеренно-,  или высокодифференцированных опухолях, чем при дисплазии \cite{aihara_allelic_1998}.
 

\subsection{Рецептор гепатоцитарного фактора роста, MET}

MET – рецептор гепатоцитарного фактора роста (HGF), принадлежащий к семейству тирозин-киназных белков \cite{park_sequence_1987}. MET играет важную роль в процессах пролиферации, морфогенеза клеток, эпителиально-мезенхимальном переходе, а также играет ключевую роль в эмбриогенезе \cite{trusolino_met_2010}.  Основным лигандом рецептора является HGF, однако с MET способны связываться киназа фокальной адгезии (FAK), $\upbeta$-катенин, VEGFR (рецептор фактора роста эндотелия сосудов), EGFR (рецептор эпидермального фактора роста), CD44, интегрин $\upalpha$6$\upbeta$4, активирующие канонический и неканонические сигнальные пути соответственно \cite{garcia-vilas_updates_2018}. Помимо лигандов MET может взаимодействовать с внутриклеточными адаптерными белками, способствующими трансдукции сигнала внутри клетки, -- SHC, GAB1 (GRB2-ассоциированный связывающий белок 1), GRB2 (белок 2, связанный с рецептором ростового фактора), а также белками из  семейств Ras, STAT \cite{trusolino_met_2010}. Кроме того, рецептор может взаимодействовать с  дезинтегрин- и металлопротеаза-подобной внеклеточной протеазой (ADAM), расщепляющей MET на внеклеточный и внутриклеточный фрагменты \cite{foveau_down-regulation_2009}.

\subsubsection{Структура MET}

Рецептор MET  состоит из следующих функциональных доменов \cite{trusolino_met_2010}:
\begin{enumerate}
	\item Внеклеточные домены:
	\begin{itemize}
		\item \textbf{SEMA-домен}, представленный $\upalpha$- и $\upbeta$-субъединицами, необходим для связывания с HGF;
		\item \textbf{PSI-домен}, принимающий участие в связывании с HGF;
		\item 4 \textbf{иммунноглобулин-подобных домена} (IPT);
	\end{itemize} 
	\item Внутриклеточные домены:
	\begin{itemize}
		\item \textbf{юкстамембранный домен} (ЮД) участвует в связывании с убиквитин-лигазой;
		\item \textbf{тирозин-киназный домен} -- каталитический регион, модулирующий киназную активность;
		\item \textbf{мультифункицональный домен связывания} (МДС) участвует в связывании с белками-передатчиками сигнала внутрь клетки.
	\end{itemize} 
\end{enumerate}

На рисунке \ref{figure:met} проиллюстрированно строение рецептора.

\begin{figure}[h]
	\centering
	\includegraphics[width=\textwidth]{pics/met.png}
	\caption{\textbf{Строение рецептора MET.}
	Рецептор представлен внеклеточной и внутриклеточной частями. Слева-направо (от N- к C-концу): к внеклеточной части относятся SEMA-домен (представлен $\upalpha$- и $\upbeta$-субдоменами), PSI-домен, 4 иммунноглобулин-подобных домена, к внутриклеточной -- юкстамембранный домен (обозначен как ЮД), тирозин-киназный домен (обозначен как ЮД), мультифункциональный домен связывания (обозначен как МДС). Пунктирной линией обозначена трансмембранная часть белка.}
	\label{figure:met}
\end{figure}

При связывании рецептора с HGF происходит димеризация MET, что приводит к транс-фосфорилированию по двум сайтам -- Tyr1234 и Tyr1235. Следующий шаг активации рецептора -- это автофосфорилирование по двум аминокислотным остаткам МДС -- Tyr1349 и Tyr1356. Фосфорилирование двух последних сайтов -- необходимый шаг, так как в таком состоянии MET способен связываться с адаптерными белками \cite{trusolino_met_2010}.

\subsubsection{Партнеры связывания MET}

С активированным рецептором MET способны взаимодействовать PI3K (фосфоинозитид-3-киназа) \cite{ponzetto_multifunctional_1994}, GRB2  \cite{ponzetto_multifunctional_1994}, SHP2 \cite{fixman_pathways_1996} и SHC \cite{fixman_pathways_1996}, фосфолипаза C (PLC) \cite{ponzetto_multifunctional_1994}, транскрипционные факторы семейства STAT (STAT3) \cite{zhang_requirement_2002}. MET также связывается с  GAB1, мультиадаптeрным белком, который после фосфорилирования рецептора MET обеспечивает дополнительные сайты связывания для SHC, PI3K, SHP2, PLC и p120-Ras-ГТФаза-активирующего белка (p120-Ras-GAP) \cite{trusolino_met_2010}. MET взаимодействует с GAB1 напрямую, через уникальный сайт длиной 13 аминокислотных остатков, или опосредовано через GRB2 \cite{schaeper_coupling_2000, lock_identification_2000}. MET также способен взаимодействовать с интегрином $\upalpha$6$\upbeta$4 – рецептор может фосфорилировать $\upbeta$4-субъединицу по трем сайтам, создавая таким образом дополнительные сайты связывания для SHC, PI3K и SHP2. В таком комплексе GRB2 может связаться с SHC \cite{trusolino_met_2010}.

Внеклеточный домен изоформы v6 гиалуронового рецептора CD44 (CD44v6) образует тройной комплекс с MET и HGF. Образование комплекса необходимо для активации рецептора. Внутриклеточная часть CD44v6 связывает цитоплазматический домен МЕТ с актиновыми микрофиламентами через GRB2 и промежуточные белки, что способствует МЕТ-индуцированной активации RAS при участии фактора обмена гуаниновых нуклеотидов SOS (son of sevenless) \cite{orian-rousseau_hepatocyte_2007}.

\subsubsection{Сигнальные пути, регулируемые MET}

MET-зависимые сигналы, передаваемые от рецептора, оказывают влияние на многие сигнальные пути, которые способны регулировать тирозин-киназные рецепторы \cite{gherardi_targeting_2012}. Рецептор может связываться не только с HGF, активируя канонический сигнальный путь, но и с VEGFR, EGFR, CD44, интегрином $\upalpha$6$\upbeta$4. С этими лигандами рецептор может связаться только в состоянии мономера, а при димеризации способен образовать комплекс с $\upbeta$-катенином и киназой фокальной адгезии (FAK). При связывании с лигандом, отличным от HGF, происходит активация неканонических сигнальных путей \cite{garcia-vilas_updates_2018}.

Каскады митоген-активируемой протеинкиназы (MAPK) состоят из трех подсемейств, каждое из которых включает три протеинкиназы, которые последовательно активируют друг друга. Проксимальные элементы каскада активируются прямо или косвенно с помощью малых ГТФаз Ras. ГТФазы, в свою очередь, активируются факторами обмена нуклеотидов (GEF), в том числе, в случае активации Ras рецептором MET, белками SOS, и инактивируются белками, активирующими ГТФазу (GAP) – например, p120, который ингибируется в результате активации MET. Терминальные эффекторы рассматриваемых MAPK каскадов включают в себя киназы ERK1/2,  JNK1/2/3 и белки p38$\upalpha$,$\upbeta$,$\upgamma$,$\updelta$. Они транслоцируются в ядро, где влияют на активность различных факторов транскрипции. Стимуляция ERK киназ может приводить к активации пролиферации, белков JNK и p38 – к увеличению уровня пролиферации, дифференцировки клеток, а также к индукции апоптоза  \cite{trusolino_met_2010}. Однако JNK и p38 могут ингибировать процесс клеточной пролиферации. Такой разнонаправленный эффект обусловлен интенсивностью и продолжительностью сигнала, а также тканевой принадлежностью клеток, в которых активен JNK и/или p38 MAPK сигнальный каскад \cite{wagner_signal_2009}.

PI3K представляет собой липид-киназу, которая связывается с МДС МЕТ и катализирует образование фосфатидилинозитол-трифосфата (PtdIns3P). Производство PtdIns3P создает сайт связывания для протеин-киназы B (AKT), ключевыми эффекторами которой являются фосфоинозитид-зависимая киназа PDK-1 и белковый комплекс mTORC2 \cite{garcia-vilas_updates_2018}. После компартментализации на внутренней стороне плазматической мембраны AKT активируется и фосфорилирует несколько субстратов, вовлеченных в пролиферацию клеток, выживание и регуляцию размера клеток \cite{trusolino_met_2010}.

Активированный MET фосфорилирует STAT3 по Y705, что приводит к гомодимеризации STAT3 по домену SH2 и транслокации димера в ядро, где он действует как фактор транскрипции. В итоге, ускоряется трансформация клеток, повышается уровень пролиферации, а также ускоряется формирование сосудов (тубулогенез), обусловленное активацией VEGFA \cite{trusolino_met_2010}.

В ответ на МЕТ-опосредованную активацию PI3K-зависимых путей, а также, вероятно, Src, активируется I$\upkappa$B-киназный комплекс (IKK), который фосфорилирует находящийся в комплексе с транскрипционным фактором NF-$\upkappa$B и секвестрирующий его в цитоплазме ингибитор I$\upkappa$B. В результате фосфорилирования I$\upkappa$B подвергается протеасомной деградации, что приводит к ядерной транслокации NF-$\upkappa$B, где последний регулирует транскрипцию генов, стимулирующих пролиферацию, выживание клеток, а также ускорение формирования сосудов \cite{trusolino_met_2010}.

В таблице 1 приведена краткая информация о сигнальных путях, которые регулирует рецептор MET, а также представлен механизм регуляции \cite{trusolino_met_2010, gherardi_targeting_2012}.

\subsubsection{Сигнальные пути, регулирующие работу MET}

Существует несколько процессов, тонко регулирующих уровень экспрессии MET на плазматической мембране: регулируемый протеолиз и захват рецептора внутрь клетки \cite{trusolino_met_2010}. 

MET может подвергаться последовательному протеолитическому расщеплению по двум сайтам – один во внеклеточной части (один из IPT-доменов), а другой – во внутриклеточной (юкстамембранный домен – ЮД). Первое расщепление (известное как выделение (shedding)), которое осуществляется с помощью ADAM, приводит к образованию растворимого внеклеточного фрагмента, который ингибирует активность полноразмерного рецептора. Второе расщепление выполняется $\upgamma$-секретазой и приводит к появлению внутриклеточного фрагмента, который разрушается по убиквитин-независимому механизму протеасомой \cite{foveau_down-regulation_2009}.

Кроме того, в литературе описана эндоцитоз-зависимая регуляция активности MET, которая инициируется ассоциацией MET с белком CBL, убиквитин-лигазой, которая связывает MET и адаптерные белки, необходимые для формирования эндосомы. При этом MET в составе эндосомы подвергается лизосомальной деградации \cite{petrelli_endophilin-cin85-cbl_2002}.

С другой стороны, описаны случаи, когда при эндоцитозе МЕТ сохраняет способность активировать по крайней мере некоторые свои мишени. Так, интернализованный MET поддерживает активацию ERK1/2 через MEK1. Трафик интернализованного MET, а также его способность активировать ERK при этом регулируются протеинкиназой C$\upepsilon$ (PKC$\upepsilon$), которая также необходима для локализации активированного ERK в сайтах фокальной адгезии, где ERK может способствовать клеточной миграции \cite{liu_hepatocyte_2002, kermorgant_pkc_2004}. Дальнейший трафик МЕТ в околоядерные мембранные компартменты опосредуется PKC$\upalpha$. В этих компартментах активность фосфотирозинфосфатаз, которые воздействуют на STAT3, является низкой. Это способствует эффективному фосфорилированию STAT3 с помощью MET и транслокации STAT3 в ядро \cite{carpenter_stat3_2014}.

\begin{table}[H]
	\renewcommand{\arraystretch}{1.4} %increased line spacing
	\caption{\textbf{Сигнальные пути, регулируемые рецептором MET.}}
	\label{table:cascades}
	\enspace
	\begin{adjustbox}{max width=\textwidth}
		\begin{tabu} to \textwidth {XXX}
			\hline
			\multicolumn{1}{c}{\textbf{Сигнальный путь}} & 
			\multicolumn{1}{c}{\textbf{Механизм регуляции}} &
			\multicolumn{1}{c}{\textbf{Результат регуляции}}\\
			\hline
			MAPK каскады & Активация белков p38, ERK и JNK & Увеличение уровня пролиферации, дифференцировки клеток; индукция апоптоза \\
			PI3K-AKT & Активация AKT на внутренней стороне плазматической мембране &  Увеличенный уровень пролиферации клеток, выживания, а также положительная регуляция размера клеток \\
			STAT & Димеризация STAT3 при связывании с активированным MET & Увеличенный уровень пролиферации, а также ускорение формирования сосудов (тубулогенез) \\
			I$\upkappa$B$\upalpha$ - NF-$\upkappa$B комплекс & Активация IKK, последующее фосфорилирование I$\upkappa$B и ядерная транслокация NF-$\upkappa$B & Увеличенный уровень пролиферации, выживания клеток, а также ускорение формирования сосудов \\	
			\hline
		\end{tabu}
	\end{adjustbox}
\end{table}

\subsubsection{Роль MET в гепатоканцерогенезе}

Во многих опухолях транскрипция МЕТ индуцируется гипоксией и провоспалительными цитокинами или проангиогенными факторами, которые присутствуют в строме опухолей. Активация рецептора MET способствует опухолевой прогрессии путем передачи пролиферативных сигналов, а также сигналов к миграции клеток \cite{pennacchietti_hypoxia_2003, bhowmick_stromal_2004}.

При внесении в МДС MET мутации, приводящей к преимущественному связыванию с GRB2, но не с p85 (таким образом PI3K сигнальный путь не активируется), наблюдалось повышение уровня пролиферации, однако инвазии и метастазирования in vivo не наблюдалось. При внесении мутации, приводившей к преимущественной активации PI3K, клетки характеризовались высокой подвижностью, но инвазии и метастазирования не наблюдалось \cite{giordano_point_1997, bardelli_concomitant_1999}. Нативный белок в равной степени способен связываться и с GRB2, и с p85, активируя различные сигнальные пути в клетке. Таким образом, для инициации MET-зависимого инвазивного роста в опухолевых клетках необходима комбинированная активация нескольких сигнальных путей \cite{trusolino_met_2010}. 

HGF, лиганд рецептора MET, был первоначально описан благодаря его митогенным свойствам, дальнейшие исследования выявили его способность ингибировать апоптоз. Показано, что HGF обеспечивает защиту печени от массивного повреждения тканей  за счет подавления вызванного белком FAS апоптоза гепатоцитов \cite{kosai_abrogation_1998}. Антиапоптотическая роль HGF была описана у мышей с нокаутом MET в гепатоцитах, которые приобретали гиперчувствительность к повреждению печени при стимуляции Fas его агонистом JoS. Такие мыши демонстрируют замедленную регенерацию печени и более склонны к развитию фиброза печени \cite{giebeler_c-met_2009}. Эти наблюдения доказывают важное значение системы HGF-MET для регенерации печени.

В трансформированных клетках нерегулируемая передача сигналов MET происходит в случае его гиперэкспрессии, а также в результате возникновения мутаций, приводящих к конституитивной активации рецептора или препятствующих его деградации \cite{tovar_met_2017}. Так, амплификация рецептора наблюдалась в 1-5 \% случаев ГК \cite{takeo_examination_2001, kondo_clinical_2013, wang_genomic_2013}. В одном из исследований анеуплоидия по 7 хромосоме, на которой расположен ген MET, выявлена в 37 \% процентах случаев ГК \cite{kondo_clinical_2013}. В целом ряде исследований описана гиперэкспрессия MET в ГК \cite{boix_c-met_1994, suzuki_expression_1994, selden_expression_1994,daveau_hepatocyte_2003, noguchi_gene_1996}. При этом в рамках одного из исследований была продемонстрирована пониженная экспрессия HGF в опухолевой ткани по сравнению с прилежащей неопухолевой тканью и случаи снижения уровня мРНК MET в опухолевой ткани наряду со случаями гиперэкспрессии \cite{selden_expression_1994}. Серия иммунногистохимических экспериментов также продемонстрировала неоднородный характер экспрессии MET: в части работ обнаружена гиперэкспрессия рецептора \cite{kiss_analysis_1997, tavian_u-pa_2000}, в части – пониженная экспрессия как MET, так и HGF \cite{kiss_analysis_1997}. Таким образом, в большинстве исследований было обнаружено снижение уровней HGF в ГК, что свидетельствует о том, что активация MET должна быть в значительной степени независимой от лиганда, например, из-за мутаций в рецепторе, участия в неканонических сигнальных путях, гиперэкспрессии.

На крысиных клетках гепатомы FaO показано, что стимуляция MET его лигандом HGF приводит к ингибированию пролиферации клеток. В то же время, стимуляция MET ростовым фактором HGF приводила к увеличению инвазивного потенциала клеток \cite{shiota_hepatocyte_1992}. На клетках гепатомы человека HepG2 показано, что анти-пролиферативная активность ростового фактора коррелирует с активацией киназ ERK, а также с активацией транскрипции ингибиторов циклин-зависимых киназ  p16INK4A и p21Cip1, что приводит к ингибированию клеточной пролиферации \cite{shirako_up-regulation_2008}. С другой стороны, ингибирование MET приводит к увеличению уровня пролиферации и способности клеток к инвазии \cite{salvi_vitro_2007}. Причина такого неоднородного клеточного ответа остается неясной. Возможно, это связано с существованием альтернативных лигандов рецептора MET, запускающих внутриклеточный сигналинг. 

Одним из таких лигандов является предшественник протромбина дес-гамма-карбоксипротромбин (DCP). На клетках гепатомы человека Hep3B и SK-Hep-1 было показано, что DCP связывается с MET и вызывает стимуляцию пролиферации за счет активации JAK1/STAT3 сигнального пути, в то время как активации RAF/MAPK и PI3K/AKT не происходит \cite{suzuki_-gamma-carboxy_2005}. Если предложенный механизм считать верным, становится понятным, почему клеточный ответ на связывание DCP-MET отличается от клеточного ответа на образование комплекса HGF-MET. Кроме DCP и HGF, предполагаемым индуктором активации рецептора MET является рецептор эпителиального фактора роста, EGFR. В клетках гепатомы, не экспрессирующих HGF, наблюдалось фосфорилирование MET, вызванное ассоциацией двух рецепторов \cite{jo_cross-talk_2000}.

Кроме описанных in vitro экспериментов, были предприняты попытки продемонстрировать роль HGF-MET системы на мышиных моделях. 

Эмбрионы мышей, не экспрессирующие HGF и MET, не могли полностью дифференцироваться во взрослый организм и погибали in utero. Печень таких мышей характеризуется малым размером и почти полным отсутствием паренхиматозных клеток \cite{schmidt_scatter_1995, uehara_placental_1995}. Результаты работ по изучению влияния гиперэкспрессии и инактивации HGF и MET противоречивы. В части работ по изучению с использованием нокаутных по гену MET мышей сообщается о повышении частоты возникновения ГК, инициированной N-нитрозодиэтиламином (DEN), и увеличении размеров опухоли \cite{marx-stoelting_hepatocarcinogenesis_2009, takami_loss_2007}. Это может быть связано с тем, что в норме MET выполняет функцию белка, поддерживающего гомеостаз печени, и при потере экспрессии рецептора клетка может приобретать опухолевый фенотип. С другой стороны, был продемонстрирован про-опухолевый эффект в мышах, конститутивно гиперэкспрессирующих MET в гепатоцитах \cite{tward_genomic_2005}.  

Результаты экспериментов по гиперэкспрессии и инъекции HGF, проведенные на мышах, так же неоднородны. Часть работ указывает на значительное увеличение размеров опухоли при воздействии HGF вне зависимости от способа инициации канцерогенеза (наличие или отсутствие DEN индуктора) \cite{yaono_hepatocyte_1995, ogasawara_hepatocyte_1998}, тогда как в рамках других исследований демонстрируется обратный эффект \cite{liu_hepatocyte_1995}. При трансфекции генетического конструкта HGF с различными промоторами в эмбрионы мышей выяснилось, что в мышах с экспрессирующимся HGF под металлотионеиновым промотором происходит образование ГК \cite{horiguchi_hepatocyte_2002, sakata_hepatocyte_1996}, в то время как образования опухоли не наблюдается при экспрессии HGF под альбуминовым промотором \cite{shiota_hepatocyte_1994}. Однако, разнородные результаты могут быть вызваны, по меньшей мере, двумя причинами: уровень экспрессии HGF, достигнутый в сыворотке крови с использованием промотора металлотионеина, был в 2–5 раз выше по сравнению с тем, который был получен с промотором альбумина; кроме того, в одном из исследований использовали кДНК HGF мыши вместо кДНК HGF человека \cite{sakata_hepatocyte_1996}, использованной в другой статье \cite{shiota_hepatocyte_1994}; известно, что перекрестное взаимодействие между человеческим HGF и мышиным MET не является достаточным для обеспечения оптимальной активности рецептора \cite{giordano_met_2014}. 


В таблице \ref{table:met_exp} содержится обобщенная информация об in vitro и in vivo экспериментах с мутантными HGF и MET.

\begin{table}[H]
	\renewcommand{\arraystretch}{1.4} %increased line spacing
	\caption{\textbf{\textit{In vitro} и \textit{in vivo} эксперименты с мутантными HGF и MET.}}
	\label{table:met_exp}
	\enspace
	\begin{adjustbox}{max width=\textwidth}
		\begin{tabu}{XX}
			\hline
			\multicolumn{1}{c}{\textbf{Эксперимент}} & 
			\multicolumn{1}{c}{\textbf{Результат эксперимента}} \\
			\hline
			
			Нокаут MET во взрослых мышах & Повышение склонности к фиброзу печени, замедленная регенерация печени \cite{giebeler_c-met_2009} \\
			Нокаут MET в мышиных эмбрионах & Неполная дифференциовка во взрослый организм, гибель \textit{in utero}, малые размеры печени \cite{schmidt_scatter_1995, uehara_placental_1995} \\
			Нокаут MET & Повышение частоты возникновения ГК, инициированной DEN, увеличение размеров опухоли \cite{marx-stoelting_hepatocarcinogenesis_2009, takami_loss_2007} \\
			Ингибирование при помощи антисмысловой РНК MET и РНК интерференции & Увеличение уровня пролиферации и инвазивного потенциала клеток \cite{salvi_vitro_2007} \\	
			Трансфекция MET & Увеличение уровня пролиферации и инвазивного потенциала клеток \cite{tward_genomic_2005} \\	
			Инъекция HGF & Значительное увеличение размеров опухоли \cite{yaono_hepatocyte_1995, ogasawara_hepatocyte_1998} \\
			Инъекция HGF & Уменьшение размеров опухоли \cite{liu_hepatocyte_1995} \\
			Трансфекция HGF с металлотиониновым промотором & Образование ГК \cite{horiguchi_hepatocyte_2002, sakata_hepatocyte_1996} \\
			Трансфекция HGF с альбуминовым промотором & ГК не образуется \cite{shiota_hepatocyte_1994} \\
			
			\multicolumn{2}{c}{\textit{In vitro} эксперименты} \\
			Трансфекция HGF с альбуминовым промотором в FaO крысиные клетки гепатомы & Ингибирование клеточной пролиферации и увеличение инвазивного потенциала клеток \cite{shiota_hepatocyte_1992} \\
			Трансфекция HGF в FaO крысиные клетки гепатомы & Ингибирование клеточной пролиферации \cite{shiota_hepatocyte_1992} \\
			Обработка HGF клеточной культуры гепатомы человека HepG2, экспрессирующей HGF & Ингибирование клеточной пролиферации \cite{shirako_up-regulation_2008} \\

			\hline
		\end{tabu}
	\end{adjustbox}
\end{table}

\subsubsection{Мутации MET}

В трансформированных клетках нерегулируемая передача сигналов MET, описанная выше, происходит в случае гиперэкспрессии, а также активации рецептора при возникновении мутаций \cite{tovar_met_2017}. Впервые активирующие мутации MET были описаны у пациентов с различными формами рака почки \cite{schmidt_novel_1999}. Все найденные мутации находятся в каталитическом домене недалеко от сайтов транс-фосфорилирования (Tyr1234 и Tyr1235). 

В нескольких экспериментах на опухолевых аллографтах и клеточных культурах было показано, что мутации в тирозин-киназном домене индуцируют конститутивную активацию рецептора \cite{jeffers_activating_1997}. На мышиных моделях показано, что некоторые варианты приводят к возникновению различных типов сарком, карцином и лимфом. Например, у мышей с мутацией, приводящей к аминокислотной замене M1248T в MET, развивались различные карциномы и лимфомы, а у мышей с мутациями, приводящими к заменам D1226 и Y1228C – саркомы и лимфомы \cite{graveel_activating_2004}. Поскольку единственными различиями между этими животными были мутации MET, эти результаты позволяют предположить, что либо сама структура мутированной киназы, либо уровень экспрессии рецептора (или оба фактора), влияют на тканеспецифичность образования опухоли. В целом, эти исследования продемонстрировали, что активирующие мутации влияют не только на активность МЕТ, но и на активность сигнальных путей в различных типах тканей. 

После обнаружения активирующих мутаций в тирозин-киназном домене у пациентов с различными формами опухолей почки были проведены скринингоые исследования, в результате которых удалось обнаружить новые мутации как в тирозин-киназном домене, так и в других доменах рецептора, в том числе в юкстамембранном домене.

Первые мутации в юкстамембранном домене были выявлены в образцах опухолей желудка (P1009S) \cite{lee_novel_2000} и молочной железы (T1010I) \cite{lee_novel_2000}. Замена T1010I была обнаружена также в образцах опухолевой ткани немелкоклеточного рака легкого \cite{tengs_transforming_2006}, наследственной форме колоректального рака \cite{neklason_activating_2011}, рака простаты \cite{sethi_comprehensive_2013}, хронического лимфоцитарного рака \cite{brown_systematic_2008}, мелкоклеточного рака легких  \cite{ma_c-met_2003}, паппилярного рака почек \cite{schmidt_novel_1999}, рака щитовидной железы \cite{wasenius_met_2005}, крупноклеточного рака легких \cite{lee_novel_2000}, рака яичников \cite{tang_met_2014}. 

На клеточной линии фибробластов NIH3T3 было показано, что MET T1010I не проявляет конститутивного фосфорилирования \cite{schmidt_novel_1999}, при этом длительность фосфорилирования у белка с заменой и нативного белка не отличалась \cite{lee_novel_2000}. Также было показано, что клетки NIH3T3, экспрессирующие мутантный белок, не образуют участков фокальной адгезии и не демонстрируют значимого повышения клоногенного потенциала по сравнению с клетками, гиперэкспрессирующими MET дикого типа. Подкожная инъекция таких клеток бестимусным мышам приводит лишь к незначительному ускорению развития опухолей по сравнению с контролем  \cite{lee_novel_2000}. 

В то же время, в клеточной культуре гемопоэтических клеток мыши Ba\textbackslash F3 продемонстрировано значительное увеличение уровня конститутивного фосфорилирования MET T1010I, а также незначительное увеличение уровня пролиферации и миграции клеток \cite{ma_c-met_2003}. Клетки эпителия молочной железы человека MCF-10A демонстрировали повышенный уровень миграции и инвазии при экспрессии MET T1010I. При этом незначительно повышался уровень фосфорилирования STAT3, AKT и MAPK. При подкожной инъекции клеток MCF-10A, экспрессирующих MET T1010I, трансгенным иммунодефицитным мышам, экспрессирующим hHGF, развивающиеся опухоли характеризуются более высоким уровнем инвазии по сравнению с опухолями, возникшими из клеток, гиперэкспрессирующих WT MET \cite{liu_functional_2015}. 

Несмотря на неоднозначость имеющихся в литературе данных, наличие замены T1010I в MET может оказатся одним из факторов, способствующих прогрессии ГК. Мутация возникает в юкстамембранном домене, ответственном за связывание с убиквитин-лигазой E3 \cite{trusolino_met_2010}. Известно, что замена MET Y1003F в сайте связывания MET с убиквитин-лигазой E3, находящегося в том же домене, что и T1010I, приводит к трансформации фибробластов и эпителиальных клеток – такие варианты рецепторов могут интернализоваться с образованием эндосом при взаимодействии с лигандом HGF, но не могут быть убиквитинилированы \cite{peschard_mutation_2001}. Учитывая близкое расположение T1010I к сайту связывания с убиквитин-лигазой E3, несинонимичная замена в позиции 1010 может привести к аналогичному эффекту и, тем самым способствовать развитию ГК.



        \section{Материалы и методы}

\subsection{Клинические образцы}

Для исследования были использованы клинические образцы ткани ГК человека и прилежащей к опухоли неопухолевой ткани печени тех же пациентов, полученные во время проведения резекции опухолей пациентов с ГК в отделении опухолей печени и поджелудочной железы ФГБУ «НМИЦ им. Н. Н. Блохина» Минздрава РФ. Сразу после резекции взятые образцы замораживали в жидком азоте, после чего хранили их при температуре -70 \textdegree{}C. Геномную ДНК и тотальную РНК выделяли согласно протоколам производителей с использованием наборов Wizard SV Genomic DNA Purification System (Promega) с последующим переосаждением ДНК и PureLink RNA Mini Kit (Life Technologies) с применением ДНКазы, соответственно. Концентрацию ДНК и РНК определяли по оптической плотности раствора, которую измеряли на спектрофотометре NanoDrop1000 (Thermo Scientific, США) при длине волны 260 нм. Присутствие примесей в образцах оценивали по соотношению значений оптической плотности раствора ДНК, РНК при длинах волн 230, 260 и 280 нм. Данные полнотранскриптомного секвенирования клинических образцов ГК человека и соответствующих им неопухолевых тканей печени (bam-файлы) получены ранее в сотрудничестве с лабораторией эволюционной геномики ФББ МГУ.

\subsection{Поиск потенциально патогенных герминативных SNV в аллель-специфически экспрессирующихся генах}

Для нахождения герминативных однонуклеотидных вариантов (SNV) в аллель-специфически экспрессирующихся (АСЭ) генах были использованы транскриптомные данные 45 парных образцов гепатокарциномы и прилежащей ткани печени, представленные в двух биологических репликах. Транскриптомы представлены в виде выровненных картированных прочтений на геном человека сборки GRCh37. Списки генов, в которых был произведен поиск SNV, – онкогены (138 генов), опухолевые супрессоры (278 генов) и печень-специфические гены (156 генов), – были выгружены из базы данных соматических мутаций (COSMIC v87 \cite{forbes_cosmic:_2017}) и The Human Protein Atlas \cite{uhlen_pathology_2017}. 

Герминативные SNV были выявлены  при помощи набора инструментов GATK (Genome Analysis Toolkit, версия 4.0.10.1) \cite{mckenna_genome_2010}. В частности, были использованы HaplotypeCaller для поиска SNV в экзонах генов из составленного нами перечня, CombineGVCF для иерархического объединения gVCF-файлов, полученных на предыдущем шаге, GenotypeGVCF для получения корректных вероятностей генотипа, VariantFiltration для жесткой фильтрации SNV – пороговые значения QD, ReadPosRankSum, FS, MQ и MQRankSum были повышены. Кроме того, для повышения точности предсказания пороговое значение качества нуклеотида в прочтении было повышено с 10 до 20 единиц. В целях снижения вероятности получения ложноположительных результатов из дальнейшего рассмотрения исключались SNV, расположенные в участках с высокой гомологией с другими областями генома человека согласно информации, полученной в базах данных Segmental Duplications и SelfChain. Также исключались инделы, поскольку, несмотря на локальную пересборку участков в HaplotypeCaller, большая часть замен такого типа была выявлена на границах протяженных участков повторяющихся нуклеотидов и являлась артефактом.

Полученные SNV, отвечающие критериям качества на глубину покрытия, картирования, и равномерно представленные в прямой и обратной цепях парно-концевых прочтений, а также прошедшие дополнительную фильтрацию по глубине покрытия и долям референсного и альтернативного вариантов использовали для поиска АСЭ генов алгоритмом MBASED \cite{mayba_mbased:_2014}. Этот статистический метод позволяет предсказывать аллель-специфическую экспрессию гена на основе транскриптомных данных о покрытии множества сайтов, содержащих SNV, в том числе без опоры на информацию о гаплотипе. По результатам вычислений, проведенных с использованием MBASED в режиме анализа отдельных образцов, аллель-специфической экспрессией считали экспрессию гена, в которой доля мажорного аллеля превышала 70 \% при FDR < 5 \%. FDR вычисляли методом Бенджамини-Хохберга на основе данных о p-value, полученных алгоритмом MBASED для предсказанных долей мажорных аллелей.

Для получения списка потенциально патогенных SNV в АСЭ генах были установлены критерии функциональной значимости – замена должна быть несинонимичной, а частота варианта – низкой (< 5 \% в популяции). Кроме того, были использованы алгоритмы предсказания функциональной значимости (Polyphen-2 \cite{adzhubei_predicting_2013}), также учитывалось наличие данных об этом варианте в литературе.

\subsection{Олигонуклеотидные праймеры и зонды, использованные при проведении ПЦР}

Для дизайна зондов и фланкирующих праймеров были использованы референсные последовательности из базы данных NCBI, а также программа PrimerBlast \cite{ye_primer-blast:_2012}. Аллель-специфические зонды TaqMan, меченные FAM, для гена MET и его транскрипта NM\_001324402.1 комплементарны референсному или альтернативному варианту нуклеотидной последовательности, которые различаются одной позицией – rs56391007 (нуклеотидная замена c.C3029T, аминокислотная замена T1010I), а также содержат дополнительную замену вблизи от нуклеотида, специфичного к одному из аллелей. Для аллель-специфических зондов, меченных FAM, был подобран общий прямой зонд TaqMan, меченный HEX. Все зонды были также помечены с 3’ конца гасителем BHQ1. 

Последовательности праймеров и зондов, длины соответствующих ПЦР-продуктов приведены в таблице \ref{table:primers} (жирным шрифтом выделен нуклеотид, соответствующий замене c.C3029T). Праймеры и зонды были синтезированы фирмой «ДНК-синтез», Россия.

\begin{table}[H]
	\renewcommand{\arraystretch}{1.4} %increased line spacing
	\caption{\textbf{Специфические праймеры, использованные при проведении ПЦР.}}
	\label{table:primers}
	\enspace
	\begin{adjustbox}{max width=\textwidth}
		\begin{tabu} spread 0pt {XXXXX}
			\hline
			\multicolumn{1}{p{2.5cm}}{\textbf{Матрица}} & 
			\multicolumn{1}{p{8.5cm}}{\textbf{Последовательность прямого праймера/зонда (от 5' к 3')}} & 
			\multicolumn{1}{p{9.0cm}}{\textbf{Последовательность обратного праймера/зонда (от 5' к 3')}} & 
			\multicolumn{1}{p{3.0cm}}{\textbf{Длина ПЦР-продукта, пн}} &
			\multicolumn{1}{p{2.5cm}}{\textbf{Температура отжига, \textdegree{}C}}\\
			\hline
			Фланкирующие праймеры на \textit{MET} & CACTCCTCATTTGGATAG & GAAAAGTAGCTCGGTAG & 95 & 49 \\
			Зонды на \textit{MET} & HEX\_CTTGTAAGTGCCCGAAG\_BHQ1 & FAM\_AGCGCAA\textbf{\textit{C}}TACAGAAATGG\_BHQ1 \newline FAM\_AGCGCAA\textbf{\textit{T}}TACAGAAATGG\_BHQ1 & -- & 49 \\
			\hline
		\end{tabu}
	\end{adjustbox}
\end{table}

Рабочие растворы праймеров содержали прямой и обратный праймеры в концентрации 4 пкМ/мкл каждый. Концентрация прямого и обратного зондов в рабочем растворе составила 2,5 пкМ/мкл.

\subsection{Получение кДНК путем обратной транскрипции мРНК}

Обратную транскрипцию препаратов тотальной РНК проводили в 12,3 мкл смеси, содержащей 2,0 мкг РНК и 0,1 мкг случайных гексамерных олигонуклеотидов, которую денатурировали при 72 \textdegree{}C в течение 10 мин, охлаждали до комнатной температуры и добавляли 7,7 мкл смеси, состоящей из 4 мкл MMLV-буфера (Promega, США), по 2 мкл каждого из dNTP, 2,5 мМ (Силекс, Россия), 50 ед. обратной транскриптазы MMLV (Promega, США), 1 мкл 2 мМ дитиотритола (Sigma, США), и 2,5 ед. ингибитора рибонуклеаз RNasin Ribonuclease Inhibitor (Promega, США). Реакцию проводили при 42 \textdegree{}C в течение 60 мин и останавливали путем инактивации обратной транскриптазы при 95 \textdegree{}C в течение 10 минут. Объем реакционной смеси доводили деионизованной водой до 100 мкл и использовали аликвоты для постановки ПЦР со специфическими праймерами.

\subsection{Цифровая капельная ПЦР}

Количественное определение уровня экспрессии генов или их аллелей проводили методом цифровой капельной ПЦР с зондами TaqMan с использованием генератора капель Bio-Rad QX200 Droplet Generator, амплификатора C1000 Touch Thermal Cycler и счетчика капель Bio-Rad QX200 Droplet Reader (Bio-Rad Laboratories, США). Реакционная смесь 1 пробы объемом 25 мкл содержала: 10 мкл раствора кДНК в количестве, эквивалентном 10 нг исходной РНК, или 1 нг геномной ДНК, и 15 мкл смеси, содержащей 2 мкл специфических зондов, 4,5 мкл специфических праймеров, 5 мкл 2x Bio-Rad Supermix. Детекцию проводили в программе QuantaSoft (Bio-Rad Laboratories, США).

При проведении реакции амплификации были использованы следующие температурные режимы:

\begin{enumerate}
	\item 95 \textdegree{}C, 10 мин -- первичная денатурация;
	\item 40 циклов:
	\begin{itemize}
		\item 95 \textdegree{}C, 30 сек -- денатурация;
		\item T\textsubscript{отжига} = 49 \textdegree{}C, 1 мин -- отжиг праймеров/зондов (таблица \ref{table:primers});
		\item 72 \textdegree{}C, 30 сек -- элонгация;
	\end{itemize} 
	\item 72 \textdegree{}C, 3 мин -- полная элонгация;
\end{enumerate}

Для каждой комбинации образцов тканей, для которых в ходе скрининга было выявлено наличие полиморфизма rs56391007, и наборов праймеров/зондов было сделано и проанализировано по 3 технических реплики.


        \section{Результаты}

\subsection{Поиск аллель-специфически экспрессирующихся генов}

На основе данных транскриптомного секвенирования 45 пар образцов пациентов с гепатокарциномой (ГК), выполненного в двух биологических репликах, предоставленных в формате картированных на референсный геном человека (GRCh37) прочтений, с использованием набора инструментов Genome Analysis Toolkit, в онкогенах, опухолевых супрессорах и печень-специфических генах рассматриваемых образцов было выявлено 13137 однонуклеотидных вариантов (SNV) в экзонах генов. При помощи алгоритма MBASED \cite{mayba_mbased:_2014} были вычислены значения доли мажорного аллеля (MAF), позволяющие оценить наличие аллель-специфической экспрессии (АСЭ) в ГК, на основе данных о покрытии множества сайтов, содержащих SNV. В результате работы алгоритма АСЭ (MAF $\geq 0.7$, FDR $< 5$ \%) в ГК и неопухолевой ткани была обнаружена в 125 генах из 572 исследуемых (21.8 \%). При этом на гены со статистически значимой MAF приходилось 4189 SNV, в том числе 1617 SNV находилось в генах, проявлявших аллельный дисбаланс.

Для каждого гена, демонстрирующего АСЭ (MAF $\geq 0.7$, FDR $< 5$ \%) в ГК и неопухолевой ткани печени, были рассчитаны частоты возникновения аллельного дисбаланса. На рисунке \ref{fig:normal} показаны все гены, у которых АСЭ в неопухолевой ткани наблюдается в 2 и более раза чаще, чем в ГК, – всего было обнаружено 4 таких гена. Существенно больше, 28 генов, демонстрируют обратную картину – частота АСЭ в ГК в 2 и более раза выше, чем частота АСЭ в неопухолевой ткани (рисунок \ref{fig:tumor}). Исходя из полученных результатов, можно предположить, что при приобретении клеткой опухолевого фенотипа наблюдается усиление аллельного дисбаланса некоторых генов, которые до трансформации характеризовались неравновесным уровнем экспрессии аллелей. 

Было построено распределение генов по частоте аллельного дисбаланса в ГК и прилежащей ткани печени (рисунок \ref{fig:all_ase}). На рисунке по $O_x$ и $O_y$ отложены медианы значений MAF для отдельных генов. Значения MAF в неопухолевой и опухолевой тканях демонстрируют высокий уровень корреляции друг с другом (коэффициент корреляции Спирмена=$0.85$, p-value=$7.6\times10^{-48}$). Частоты аллельного дисбаланса в опухоли и неопухолевой ткани (посчитаны как отношение количества случаев АСЭ на количество образцов) демонстрируют гораздо меньший уровень корреляции – коэффициент корреляции Спирмена=$0.4$, p-value=$6.3\times10^{-8}$. На рисунке отмечены некоторые онкогены, в скобках указаны значения количества образцов с АСЭ в неопухолевой/опухолевой тканях (полный список онкогенов в таблице \ref{table:onco}).

В таблице приведен полный перечень названий онкогенов, которые проявляли аллельный дисбаланс в ГК и прилежащей неопухолевой ткани. В скобках дана информация о количестве образцов нормальной и опухолевой ткани, в которых была предсказана статистически значимая АСЭ (MAF $\geq 0.7$, FDR $< 5$ \%).

\begin{figure}[H]
	\centering
	\begin{subfigure}[h]{0.49\textwidth}
		\includegraphics[width=\linewidth]{pics/high_tumor_ase_frequency_new.png}
		\caption{}
		\label{fig:tumor}
	\end{subfigure}
	\begin{subfigure}[h]{0.49\textwidth}
		\includegraphics[width=\textwidth]{pics/high_normal_ase_frequency_new.png}
		\caption{}
		\label{fig:normal}
	\end{subfigure}
	\caption{\textbf{Результаты поиска АСЭ генов с использованием алгоритма MBASED}. 
	 \ref{fig:tumor} -- Гены, у которых аллель-специфическая экспрессия (АСЭ) в ГК наблюдается в 2 раза чаще (и больше, чем в 2 раза), чем в неопухолевой ткани (на рисунке \ref{fig:normal} отражена обратная картина -- частота АСЭ выше в неопухолевой ткани). В скобках указано число образцов с АСЭ в неопухолевой ткани и ГК.}
	\label{fig:ase_freq}
\end{figure}

\begin{table}[H]
	\renewcommand{\arraystretch}{1.4} %increased line spacing
	\caption{\textbf{Список онкогенов с АСЭ в ГК и неопухолевой ткани.}}
	\label{table:onco}
	\enspace
	\centering
	\begin{adjustbox}{max width=\textwidth}
		\normalsize{
		\begin{tabu} spread 0pt {XXXXX}
			\hline
			\multicolumn{5}{c}{Онкогены. Число (опухолевая/неопухолевая ткани) образцов с АСЭ} \\
			\hline
			MET (1/8) & FGFR2 (2/7) & CSF1R (2/4) & CD74 (17/21) & TFE3 (1/2) \\
			CTNNB1 (1/6) & BIRC6 (7/13) & AKT1 (1/2) & NOTCH1 (2/2) & A1CF (6/4) \\
			ERBB3 (3/7) & GNAS (2/9) & ELK4 (1/1) & EGFR (1/4) & CCND1 (2/3) \\
			ERBB2 (1/5) & EPAS1 (7/4) & FGFR3 (1/4) & DDX5 (1/4) & MAPK1 (1/6) \\
			AR (5/1) & HIF1A (2/8) & ATP1A1 (3/4) & SND1 (5/11) & SKI (2/1) \\
			ESR1 (1/1) & BRD4 (4/3) & NFE2L2 (1/3) & TRRAP (2/2) & CHD4 (3/11) \\
			KDR (4/1) & KMT2D (2/5) & MTOR (1/7) & BCL9L (1/3) &  \\
			\hline
		\end{tabu}
		}
	\end{adjustbox}
\end{table}



\subsection{Идентификация потенциально патогенных герминативных вариантов в АСЭ генах}

После получения списка генов, проявляющих дифференциальную экспрессию аллелей, была выполнена оценка изменений в экспрессии аллелей, содержащих  однонуклеотдные герминативные миссенс- и нонсенс-варианты в них. На рисунке \ref{fig:sun} представлена информация о статистически значимых MAF аллелей, содержащих герминативные миссенс- и нонсенс-варианты в онкогенах, опухолевых супрессорах и печень-специфических генах (FDR $< 5$ \%, количество полученных вариантов -- 880) для пар образцов опухолевой и прилежащей ткани печени. В 50 \% случаев наблюдается преобладание MAF в неопухолевой ткани, в 40 \% случаев -- преобладание MAF в опухоли, в остальных 10 \% случаев наблюдается одинаковая доля мажорного аллеля в опухоли и прилежащей ткани печени. Полученные результаты свидетельствуют о том, что при трансформации неопухолевой клетки происходит изменение уровня экспрессии аллелей генов. Неравновесный уровень экспрессии выражается в отклоняющемся от 0.5 значении доли мажорного аллеля (MAF). При этом в случае патогенного аллеля изменения уровня его экспрессии может быть функционально значимым.


Для поиска таких вариантов мы установили следующие критерии: предсказание функциональной значимости варианта алгоритмом Polyphen-2 \cite{adzhubei_predicting_2013}, низкая частота встречаемости варианта в популяции ($< 5$ \%), а также наличие экспериментальных данных о функциональной значимости в литературе. В результате было выявлено незначительное количество потенциально патогенных герминативных вариантов в АСЭ генах, в том числе:

\begin{itemize}
	\item у 7 пациентов (в том числе у 3 пациентов с АСЭ) rs56325023 в гене MBL2, кодирующем лектин, связывающий маннозу, который ассоциирован с дефицитом MBL2 в сыворотке и предрасполагает к развитию аутоиммунных и инфекционных заболеваний и встречается в популяции с частотой 0.5 \%;
	\item у 4 пациентов (в том числе у 2 пациентов с АСЭ) rs56391007 в гене MET;
	\item у 3 пациентов (в том числе у 2 пациентов с АСЭ) rs72550870 в гене MASP2, кодирующем сериновую протеазу, дефицит которой приводит к аутоиммунным заболеваниям и хроническому воспалению и встречается в популяции с частотой 2.2 \%;
	\item у 1 пациента (в том числе у 1 пациента с АСЭ) rs121913407 в гене CTNNB1, приводящий к конститутивной активации $\upbeta$-катенина.
\end{itemize}

\begin{figure}[H]
	\centering
	\includegraphics[width=.8\textwidth]{pics/sun_ru.png}
	\caption{\textbf{Распределение всех генов всех исследуемых образцов по значению доли мажорного аллеля (MAF)}. 50 \% случаев -- преобладание MAF в неопухолевой ткани, 40 \% -- преобладание MAF в ГК, 10 \% -- значения MAF в опухоли и неопухолевой ткани одинаковы.}
	\label{fig:sun}
\end{figure}


На основании полученных данных для дальнейшей экспериментальной проверки был выбран герминативный вариант rs56391007 (c.C3029T) в гене MET, встречающийся в 0.793 \% популяции \cite{lek_analysis_2016} и приводящий к аминокислотной замене T1010I. Этот вариант был выявлен в трех случаях из основной выборки, в одном из которых содержащий его аллель, исходя из расчетов, проявлял АСЭ в ГК.

\begin{figure}[H]
	\centering
	\includegraphics[width=.8\textwidth]{pics/all_ase_onc_ru.png}
	\caption{\textbf{Распределение генов по числу случаев аллельного дисбаланса в опухоли и неопухолевой ткани печени}. Некоторые онкогены вынесены на рисунке (полный список в таблице \ref{table:onco}); в скобках указано число образцов с АСЭ в неопухолевой ткани и ГК.}
	\label{fig:all_ase}
\end{figure}

Ранее несколькими группами исследователей на различных модельных системах было описано влияние исследуемой замены на фосфорилирование белка MET \cite{schmidt_novel_1999, ma_c-met_2003, liu_functional_2015}, образование фокальных адгезионных контактов \cite{lee_novel_2000}, скорость приобретения клетками опухолевого фенотипа \cite{lee_novel_2000}, а также на изменение уровня миграции, инвазии и пролиферации клеток \cite{lee_novel_2000, liu_functional_2015}. В то же время, экспериментальных работ, исследующих функциональную роль MET T1010I в клетках культур гепатомы до настоящего момента опубликовано не было.

\subsection{Экспериментальная проверка предсказанного аллельного дисбаланса гена \textit{MET} с заменой C3029T}

Для проведения экспериментальной верификации наличия предсказанной АСЭ в гене MET с рассматриваемой заменой мы использовали метод цифровой капельной ПЦР (ddPCR) с зондами TaqMan. Были подобраны и синтезированы три флуоресцентных зонда – один, меченный флуорофором HEX, на общий участок, не содержащий замены, а также пара аллель-специфических зондов, меченных флуорофором FAM. Для подтверждения статуса герминативных вариантов была проведена серия ПЦР с ДНК, полученной из образцов основной (45 пар образцов) и дополнительной (14 пар образцов ГК и неопухолевой ткани печени) панелей. Для оценки числа копий аллелей производилась нормировка на количество копий аллелей в ДНК, выделенной из периферических мононуклеарных клеток крови гетерозиготного пациента. Для проверки соответствия расчетных данных об уровне дисбаланса фактическому уровню, а также для установления уровня дисбаланса в экспрессии аллелей MET пациентов из расширенной выборки, гетерозиготных по рассматриваемой позиции, была проведена ddPCR с кДНК этих пациентов.

На рисунке \ref{fig:example} представлены типовые результаты проведенных экспериментов. График показывает распределение капель по интенсивности флуоресценции в каналах FAM и HEX. На верхней панели представлены результаты для образца ткани пациента, гетерозиготного по рассматриваемой позиции, на нижней (отрицательный контроль) – гомозиготного по референсному варианту. Серый кластер соответствует каплям, свечение которых детектируется только в канале HEX и, таким образом, характеризует общее количество ампликонов MET, в то время как желтый кластер содержит капли, свечение которых наблюдается в обоих каналах. Таким образом, данный подход позволяет точно оценить представленность аллеля в ДНК и в экспрессии гена.

Для оценки полученных результатов было построено распределение расстояний между кластерами (типовые кластеры представлены на рисунке \ref{fig:example}) для образцов, в которых была выявлена замена (рисунок \ref{fig:hist}) – всего было обнаружено 5 пар образцов с вариантом c.C3029T. Расстояние 1-2 соответствует расстоянию между зеленым и желтым кластерами, расстояние 1-3 – между зеленым и серым кластерами, расстояние 2-3 – между желтым и серым.  Практически во всех случаях наблюдаются дискретные пики, за исключением альтернативного варианта кДНК. 

Наличие герминативного варианта MET C3029T было экспериментально подтверждено в 5 парах образцов, то есть в 7 \% случаев ГК из расширенной панели (OR$=9.85$; $95$ \% CI [$3.96$; $24.54$]). Предсказанный в одной паре образцов аллельный дисбаланс экспрессии MET был подтвержден экспериментально и стал следствием амплификации в опухолевой ткани потенциально патогенного аллеля, что подтверждено результатами цифровой капельной ПЦР с нормировкой на ДНК, выделенной из клеток периферической крови гетерозиготного по этому SNV пациента, с ДНК, выделенной из нормальной и опухолевой ткани пациента с аллельным дисбалансом MET. При исследовании расширенной панели был выявлен дополнительный случай АСЭ MET C3029T, не связанной с изменением числа копий аллеля. 

Таким образом, использованный нами подход позволяет выявить аллельный дисбаланс экспрессии герминативных SNV в отсутствие информации о гаплотипе. Неравновесный уровень экспрессии аллелей онкогенов, генов опухолевых супрессоров и печень-специфических генов был обнаружен в 21.8 \% исследованных генов. Было построено распределение значений статистически значимых MAF (FDR $< 5$ \%) для исследуемых генов. В 50 \% случаев ГК наблюдалось преобладание MAF в неопухолевой ткани, в 40 \% -- в опухоли (рисунок \ref{fig:sun}). Это свидетельствует об изменении уровня экспрессии аллелей генов при приобретении клеткой опухолевого фенотипа. Были обнаружены 28 генов, характеризующихся преобладанием частоты АСЭ в ГК по сравнению с неопухолевой тканью печени (рисунок \ref{fig:tumor}), в то время как всего для 4 генов было показано преобладание частоты АСЭ в неопухолевой ткани. Таким образом, можно предположить, что при трансформации клетки наблюдается усиление АСЭ генов, демонстрирующих аллельный дисбаланс до приобретения клеткой опухолевого фенотипа.

В ходе экспериментальной проверки наличия АСЭ MET c c.C3029T в основной и дополнительной панелях была подтверждена АСЭ гена – среди 5 пар образцов с MET c.C3029T было обнаружено 2 случая ГК с АСЭ этого варианта в опухолевой ткани (3.3 \% случаев). Учитывая распространенность рассматриваемого варианта среди индивидов с другими типами опухолей и наличие данных о его функциональности в некоторых модельных системах, наличие герминативного варианта MET c.C3029T в 8 \% случаев ГК нашей выборки и аллельный дисбаланс в экспрессии этого варианта могут рассматриваться как потенциальный фактор инициации и/или прогрессии ГК. В дальнейшем планируется провести исследование влияния рассматриваемого варианта на биологические свойства клеток путем экзогенной экспрессии мутантного MET в клетках культур гепатомы человека HepG2 и Huh7. Кроме того, будет проверена чувствительность клеток культур гепатомы, экспрессирующих MET c.C3029T, к воздействию ингибитора рецепторов тирозинкиназ кабозантиниба, обладающего селективностью в отношении MET и недавно одобренного для терапии ГК, для выяснения целесообразности назначения такой терапии пациентам, являющимся носителями этого варианта.

\begin{figure}[H]
	\centering
	\includegraphics[width=\textwidth]{pics/new_scatter_stitch_ru.png}
	\caption{\textbf{Распределение капель по интенсивности флуоресценции в каналах FAM и HEX}. Верхняя и нижняя панели -- образцы ткани пациентов, гетеро- и гомозиготного по C3029T \textit{MET} соответственно. Серый кластер -- капли, свечение которых детектируется только в канале HEX, зеленый кластер -- капли, флуоресценция которых не наблюдается ни в одном из каналов, желтый кластер -- дважды положительный кластер. На графиках представлены распределения только для геномной ДНК.}
	\label{fig:example}
\end{figure}

\begin{figure}[H]
	\centering
	\includegraphics[width=\textwidth]{pics/hist_stitch_ru.png}
	\caption{\textbf{Распределение расстояний между кластерами для образцов с заменой C3029T \textit{MET}}. Расстояние 1-2 соответствует расстоянию между зеленым и желтым кластерами, расстояние 1-3 -- между зеленым и серым, расстояние 2-3 -- между желтым и серым (рисунок \ref{fig:example}).}
	\label{fig:hist}
\end{figure}

\newpage

\section{Выводы}

\begin{enumerate}
	\item При помощи описанного подхода, включающего набор инструментов GATK и алгоритм MBASED, в основной панели, состоящей из 45 парных образцов ГК и неопухолевой ткани печени, было обнаружено 13137 однонуклеотидных вариантов. Cреди 572 генов, включающих в себя онкогены, опухолевые супрессоры и печень-специфические гены, обнаружено 125 генов, экспрессия которых носит аллель-специфический характер;
	\item Медианные значения долей мажорных аллелей в экспрессии генов в неопухолевой и опухолевой тканях демонстрируют высокий уровень корреляции (коэффициент корреляции Спирмена=$0.85$, p-value=$7.6\times10^{-48}$). При этом на уровне отдельных генов и образцов возможны различия в уровне аллельного дисбаланса. Так, в 50 \% случаев в ГК наблюдалось преобладание экспрессии нереференсного аллеля в неопухолевой ткани, в 40 \% -- в опухоли. Таким образом, АСЭ может приводить к преимущественной экспрессии потенциально патогенных вариантов;
	\item В результате поиска потенциально патогенных вариантов в АСЭ генах выявлен и подтвержден методом цифровой капельной ПЦР встречающийся у 8 \% пациентов с ГК герминативный вариант MET c.C3029T. Частота АСЭ MET c.C3029T составляет 3.3 \% для опухолевой и прилежащей тканей печени, что, в совокупности с его предполагаемой функциональной значимостью, может способствовать инициации и/или прогрессии ГК. Кроме того, у пациентов с ГК в АСЭ генах были выявлены другие потенциально патогенные варианты, ассоциированные с развитием аутоиммунных заболеваний, хронической инфекции и воспалением.
\end{enumerate} \newpage
        \bibliographystyle{mybib.bst}  
        \bibliography{mybib.bib}
\end{document}