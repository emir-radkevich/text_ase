\section{Литература}

\subsection{Канцерогенез}



\subsection{Гепатокарцинома}

\subsection{Рецептор гепатоцитарного фактора роста, MET}

Рецептор MET -- это рецептор гепатоцитарного фактора роста (HGF). В 1987 году было показано \cite{park_sequence_1987}, что MET принадлежит к семейству тирозин-киназных белков. Однако, в то время считалось, что лигандом MET является не только HGF \cite{nakamura_molecular_1989}, но и так называемый scatter factor (SF), белковый фактор подвижности эпителия, полученный из фибробластов \cite{stoker_scatter_1987}. К 1991 году удалось доказать, что HGF и SF -- это один и тот же белок \cite{weidner_evidence_1991}. Интересной особенностью HGF-MET системы является разнообразный клеточный ответ, вызванный опосредованным взаимодействием активированного рецептора MET с белками различных сигнальных путей клетки \cite{trusolino_met_2010}.

Рецептор MET -- это тирозин-киназный рецептор, состоящий из следующих функциональных доменов \cite{trusolino_met_2010}:
\begin{enumerate}
	\item Внеклеточные домены:
	\begin{itemize}
		\item \textbf{SEMA-домен}, представленный $\alpha$- и $\beta$-субъединицами, необходим для связывания с HGF;
		\item \textbf{PSI-домен}, принимающий участие в связывании с HGF;
		\item 4 \textbf{иммунноглобулин-подобных домена};
	\end{itemize} 
	\item Внутриклеточные домены:
		\begin{itemize}
		\item \textbf{юкстамембранный домен} (ЮД) участвует в связывании с убиквитин-лигазой;
		\item \textbf{тирозин-киназный домен} -- каталитический регион, модулирующий киназную активность;
		\item \textbf{мультифункицональный домен связывания} (МДС) участвует в связывании с белками-передатчиками сигнала внутрь клетки.
	\end{itemize} 
\end{enumerate}

Внутриклеточные домены могут находиться в фосфорилированом состоянии, что влияет на работу рецептора. При связывании рецептора с HGF происходит димеризация MET, что приводит к транс-фосфорилиированию по двум сайтам -- Tyr1234 и Tyr1235. Следующий шаг в механизме работы рецептора -- это автофосфорилирование по двум аминокислотным остаткам МДС -- Tyr1349 и Tyr1356. Фосфорилирование двух последниз сайтов -- необходимый шаг, так как в таком состоянии MET способен связываться с адапторными белками -- эффекторы, содержащие Src-гомолог-2 домен (PI3K) \cite{ponzetto_multifunctional_1994}, белок 2, связанный с рецептором ростового фактора (GRB2) \cite{ponzetto_multifunctional_1994}, SHP2 \cite{fixman_pathways_1996} и SHC \cite{fixman_pathways_1996}, фосфолипаза C$\gamma$ (PLC$\gamma$) \cite{ponzetto_multifunctional_1994}, транскрипционные факторы семейства STAT (STAT3) \cite{zhang_requirement_2002}. 

MET также связывается с GRB2-ассоциированным связывающим белком 1 (GAB1), мультиадаптерным белком, который после фосфорилирования рецептора MET, обеспечивает дополнительные сайты связывания для SHC, PI3K, SHP2, PLC$\gamma$ и p120 ras-GTPase-активирующего белок (p120-ras-GAP) \cite{trusolino_met_2010}. MET взаимодействует с GAB1 напрямую, через уникальный сайт длиной 13 аминокислотных остатков, или опосредовано через GRB2 \cite{schaeper_coupling_2000} \cite{lock_identification_2000}.

Правильное функционирование рецептора необходимо для различных морфогенетических событий как в эмбриональной, так и во взрослой жизни. Рецептор принимает участие в управлении злокачественным прогрессированием нескольких различных типов опухолей. Для этого MET распространяет сложную систему сигнальных каскадов, в результате чего
перестройка экспрессии генов

\subsection{Мутации в MET}

