\section{Результаты}

\subsection{Поиск аллель-специфически экспрессирующихся генов}

На основе данных транскриптомного секвенирования 45 пар образцов пациентов с гепатокарциномой (ГК), выполненного в двух биологических репликах, предоставленных в формате картированных на референсный геном человека (GRCh37) прочтений, с использованием набора инструментов Genome Analysis Toolkit, в онкогенах, опухолевых супрессорах и печень-специфических генах рассматриваемых образцов было выявлено 13137 однонуклеотидных вариантов (SNV) в экзонах генов. При помощи алгоритма MBASED \cite{mayba_mbased:_2014} были вычислены значения доли мажорного аллеля (MAF), позволяющие оценить наличие аллель-специфической экспрессии (АСЭ) в ГК, на основе данных о покрытии множества сайтов, содержащих SNV. В результате работы алгоритма АСЭ (MAF $\geq 0.7$, FDR $< 5$ \%) в ГК и неопухолевой ткани была обнаружена в 125 генах из 572 исследуемых (21.8 \%). При этом на гены со статистически значимой MAF приходилось 4189 SNV, в том числе 1617 SNV находилось в генах, проявлявших аллельный дисбаланс.

Для каждого гена, демонстрирующего АСЭ (MAF $\geq 0.7$, FDR $< 5$ \%) в ГК и неопухолевой ткани печени, были рассчитаны частоты возникновения аллельного дисбаланса. На рисунке \ref{fig:normal} показаны все гены, у которых АСЭ в неопухолевой ткани наблюдается в 2 и более раза чаще, чем в ГК, – всего было обнаружено 4 таких гена. Существенно больше, 28 генов, демонстрируют обратную картину – частота АСЭ в ГК в 2 и более раза выше, чем частота АСЭ в неопухолевой ткани (рисунок \ref{fig:tumor}). Исходя из полученных результатов, можно предположить, что при приобретении клеткой опухолевого фенотипа наблюдается усиление аллельного дисбаланса некоторых генов, которые до трансформации характеризовались неравновесным уровнем экспрессии аллелей. 

Было построено распределение генов по частоте аллельного дисбаланса в ГК и прилежащей ткани печени (рисунок \ref{fig:all_ase}). На рисунке по $O_x$ и $O_y$ отложены медианы значений MAF для отдельных генов. Значения MAF в неопухолевой и опухолевой тканях демонстрируют высокий уровень корреляции друг с другом (коэффициент корреляции Спирмена=$0.85$, p-value=$7.6\times10^{-48}$). Частоты аллельного дисбаланса в опухоли и неопухолевой ткани (посчитаны как отношение количества случаев АСЭ на количество образцов) демонстрируют гораздо меньший уровень корреляции – коэффициент корреляции Спирмена=$0.4$, p-value=$6.3\times10^{-8}$. На рисунке отмечены некоторые онкогены, в скобках указаны значения количества образцов с АСЭ в неопухолевой/опухолевой тканях (полный список онкогенов в таблице \ref{table:onco}).

В таблице приведен полный перечень названий онкогенов, которые проявляли аллельный дисбаланс в ГК и прилежащей неопухолевой ткани. В скобках дана информация о количестве образцов нормальной и опухолевой ткани, в которых была предсказана статистически значимая АСЭ (MAF $\geq 0.7$, FDR $< 5$ \%).

\begin{figure}[H]
	\centering
	\begin{subfigure}[h]{0.49\textwidth}
		\includegraphics[width=\linewidth]{pics/high_tumor_ase_frequency_new.png}
		\caption{}
		\label{fig:tumor}
	\end{subfigure}
	\begin{subfigure}[h]{0.49\textwidth}
		\includegraphics[width=\textwidth]{pics/high_normal_ase_frequency_new.png}
		\caption{}
		\label{fig:normal}
	\end{subfigure}
	\caption{\textbf{Результаты поиска АСЭ генов с использованием алгоритма MBASED}. 
	 \ref{fig:tumor} -- Гены, у которых аллель-специфическая экспрессия (АСЭ) в ГК наблюдается в 2 раза чаще (и больше, чем в 2 раза), чем в неопухолевой ткани (на рисунке \ref{fig:normal} отражена обратная картина -- частота АСЭ выше в неопухолевой ткани). В скобках указано число образцов с АСЭ в неопухолевой ткани и ГК.}
	\label{fig:ase_freq}
\end{figure}

\begin{table}[H]
	\renewcommand{\arraystretch}{1.4} %increased line spacing
	\caption{\textbf{Список онкогенов с АСЭ в ГК и неопухолевой ткани.}}
	\label{table:onco}
	\enspace
	\centering
	\begin{adjustbox}{max width=\textwidth}
		\normalsize{
		\begin{tabu} spread 0pt {XXXXX}
			\hline
			\multicolumn{5}{c}{Онкогены. Число (опухолевая/неопухолевая ткани) образцов с АСЭ} \\
			\hline
			MET (1/8) & FGFR2 (2/7) & CSF1R (2/4) & CD74 (17/21) & TFE3 (1/2) \\
			CTNNB1 (1/6) & BIRC6 (7/13) & AKT1 (1/2) & NOTCH1 (2/2) & A1CF (6/4) \\
			ERBB3 (3/7) & GNAS (2/9) & ELK4 (1/1) & EGFR (1/4) & CCND1 (2/3) \\
			ERBB2 (1/5) & EPAS1 (7/4) & FGFR3 (1/4) & DDX5 (1/4) & MAPK1 (1/6) \\
			AR (5/1) & HIF1A (2/8) & ATP1A1 (3/4) & SND1 (5/11) & SKI (2/1) \\
			ESR1 (1/1) & BRD4 (4/3) & NFE2L2 (1/3) & TRRAP (2/2) & CHD4 (3/11) \\
			KDR (4/1) & KMT2D (2/5) & MTOR (1/7) & BCL9L (1/3) &  \\
			\hline
		\end{tabu}
		}
	\end{adjustbox}
\end{table}



\subsection{Идентификация потенциально патогенных герминативных вариантов в АСЭ генах}

После получения списка генов, проявляющих дифференциальную экспрессию аллелей, была выполнена оценка изменений в экспрессии аллелей, содержащих  однонуклеотдные герминативные миссенс- и нонсенс-варианты в них. На рисунке \ref{fig:sun} представлена информация о статистически значимых MAF аллелей, содержащих герминативные миссенс- и нонсенс-варианты в онкогенах, опухолевых супрессорах и печень-специфических генах (FDR $< 5$ \%, количество полученных вариантов -- 880) для пар образцов опухолевой и прилежащей ткани печени. В 50 \% случаев наблюдается преобладание MAF в неопухолевой ткани, в 40 \% случаев -- преобладание MAF в опухоли, в остальных 10 \% случаев наблюдается одинаковая доля мажорного аллеля в опухоли и прилежащей ткани печени. Полученные результаты свидетельствуют о том, что при трансформации неопухолевой клетки происходит изменение уровня экспрессии аллелей генов. Неравновесный уровень экспрессии выражается в отклоняющемся от 0.5 значении доли мажорного аллеля (MAF). При этом в случае патогенного аллеля изменения уровня его экспрессии может быть функционально значимым.


Для поиска таких вариантов мы установили следующие критерии: предсказание функциональной значимости варианта алгоритмом Polyphen-2 \cite{adzhubei_predicting_2013}, низкая частота встречаемости варианта в популяции ($< 5$ \%), а также наличие экспериментальных данных о функциональной значимости в литературе. В результате было выявлено незначительное количество потенциально патогенных герминативных вариантов в АСЭ генах, в том числе:

\begin{itemize}
	\item у 7 пациентов (в том числе у 3 пациентов с АСЭ) rs56325023 в гене MBL2, кодирующем лектин, связывающий маннозу, который ассоциирован с дефицитом MBL2 в сыворотке и предрасполагает к развитию аутоиммунных и инфекционных заболеваний и встречается в популяции с частотой 0.5 \%;
	\item у 4 пациентов (в том числе у 2 пациентов с АСЭ) rs56391007 в гене MET;
	\item у 3 пациентов (в том числе у 2 пациентов с АСЭ) rs72550870 в гене MASP2, кодирующем сериновую протеазу, дефицит которой приводит к аутоиммунным заболеваниям и хроническому воспалению и встречается в популяции с частотой 2.2 \%;
	\item у 1 пациента (в том числе у 1 пациента с АСЭ) rs121913407 в гене CTNNB1, приводящий к конститутивной активации $\upbeta$-катенина.
\end{itemize}

\begin{figure}[H]
	\centering
	\includegraphics[width=.8\textwidth]{pics/sun_ru.png}
	\caption{\textbf{Распределение всех генов всех исследуемых образцов по значению доли мажорного аллеля (MAF)}. 50 \% случаев -- преобладание MAF в неопухолевой ткани, 40 \% -- преобладание MAF в ГК, 10 \% -- значения MAF в опухоли и неопухолевой ткани одинаковы.}
	\label{fig:sun}
\end{figure}


На основании полученных данных для дальнейшей экспериментальной проверки был выбран герминативный вариант rs56391007 (c.C3029T) в гене MET, встречающийся в 0.793 \% популяции \cite{lek_analysis_2016} и приводящий к аминокислотной замене T1010I. Этот вариант был выявлен в трех случаях из основной выборки, в одном из которых содержащий его аллель, исходя из расчетов, проявлял АСЭ в ГК.

\begin{figure}[H]
	\centering
	\includegraphics[width=.8\textwidth]{pics/all_ase_onc_ru.png}
	\caption{\textbf{Распределение генов по числу случаев аллельного дисбаланса в опухоли и неопухолевой ткани печени}. Некоторые онкогены вынесены на рисунке (полный список в таблице \ref{table:onco}); в скобках указано число образцов с АСЭ в неопухолевой ткани и ГК.}
	\label{fig:all_ase}
\end{figure}

Ранее несколькими группами исследователей на различных модельных системах было описано влияние исследуемой замены на фосфорилирование белка MET \cite{schmidt_novel_1999, ma_c-met_2003, liu_functional_2015}, образование фокальных адгезионных контактов \cite{lee_novel_2000}, скорость приобретения клетками опухолевого фенотипа \cite{lee_novel_2000}, а также на изменение уровня миграции, инвазии и пролиферации клеток \cite{lee_novel_2000, liu_functional_2015}. В то же время, экспериментальных работ, исследующих функциональную роль MET T1010I в клетках культур гепатомы до настоящего момента опубликовано не было.

\subsection{Экспериментальная проверка предсказанного аллельного дисбаланса гена \textit{MET} с заменой C3029T}

Для проведения экспериментальной верификации наличия предсказанной АСЭ в гене MET с рассматриваемой заменой мы использовали метод цифровой капельной ПЦР (ddPCR) с зондами TaqMan. Были подобраны и синтезированы три флуоресцентных зонда – один, меченный флуорофором HEX, на общий участок, не содержащий замены, а также пара аллель-специфических зондов, меченных флуорофором FAM. Для подтверждения статуса герминативных вариантов была проведена серия ПЦР с ДНК, полученной из образцов основной (45 пар образцов) и дополнительной (14 пар образцов ГК и неопухолевой ткани печени) панелей. Для оценки числа копий аллелей производилась нормировка на количество копий аллелей в ДНК, выделенной из периферических мононуклеарных клеток крови гетерозиготного пациента. Для проверки соответствия расчетных данных об уровне дисбаланса фактическому уровню, а также для установления уровня дисбаланса в экспрессии аллелей MET пациентов из расширенной выборки, гетерозиготных по рассматриваемой позиции, была проведена ddPCR с кДНК этих пациентов.

На рисунке \ref{fig:example} представлены типовые результаты проведенных экспериментов. График показывает распределение капель по интенсивности флуоресценции в каналах FAM и HEX. На верхней панели представлены результаты для образца ткани пациента, гетерозиготного по рассматриваемой позиции, на нижней (отрицательный контроль) – гомозиготного по референсному варианту. Серый кластер соответствует каплям, свечение которых детектируется только в канале HEX и, таким образом, характеризует общее количество ампликонов MET, в то время как желтый кластер содержит капли, свечение которых наблюдается в обоих каналах. Таким образом, данный подход позволяет точно оценить представленность аллеля в ДНК и в экспрессии гена.

Для оценки полученных результатов было построено распределение расстояний между кластерами (типовые кластеры представлены на рисунке \ref{fig:example}) для образцов, в которых была выявлена замена (рисунок \ref{fig:hist}) – всего было обнаружено 5 пар образцов с вариантом c.C3029T. Расстояние 1-2 соответствует расстоянию между зеленым и желтым кластерами, расстояние 1-3 – между зеленым и серым кластерами, расстояние 2-3 – между желтым и серым.  Практически во всех случаях наблюдаются дискретные пики, за исключением альтернативного варианта кДНК. 

Наличие герминативного варианта MET C3029T было экспериментально подтверждено в 5 парах образцов, то есть в 7 \% случаев ГК из расширенной панели (OR$=9.85$; $95$ \% CI [$3.96$; $24.54$]). Предсказанный в одной паре образцов аллельный дисбаланс экспрессии MET был подтвержден экспериментально и стал следствием амплификации в опухолевой ткани потенциально патогенного аллеля, что подтверждено результатами цифровой капельной ПЦР с нормировкой на ДНК, выделенной из клеток периферической крови гетерозиготного по этому SNV пациента, с ДНК, выделенной из нормальной и опухолевой ткани пациента с аллельным дисбалансом MET. При исследовании расширенной панели был выявлен дополнительный случай АСЭ MET C3029T, не связанной с изменением числа копий аллеля. 

Таким образом, использованный нами подход позволяет выявить аллельный дисбаланс экспрессии герминативных SNV в отсутствие информации о гаплотипе. Неравновесный уровень экспрессии аллелей онкогенов, генов опухолевых супрессоров и печень-специфических генов был обнаружен в 21.8 \% исследованных генов. Было построено распределение значений статистически значимых MAF (FDR $< 5$ \%) для исследуемых генов. В 50 \% случаев ГК наблюдалось преобладание MAF в неопухолевой ткани, в 40 \% -- в опухоли (рисунок \ref{fig:sun}). Это свидетельствует об изменении уровня экспрессии аллелей генов при приобретении клеткой опухолевого фенотипа. Были обнаружены 28 генов, характеризующихся преобладанием частоты АСЭ в ГК по сравнению с неопухолевой тканью печени (рисунок \ref{fig:tumor}), в то время как всего для 4 генов было показано преобладание частоты АСЭ в неопухолевой ткани. Таким образом, можно предположить, что при трансформации клетки наблюдается усиление АСЭ генов, демонстрирующих аллельный дисбаланс до приобретения клеткой опухолевого фенотипа.

В ходе экспериментальной проверки наличия АСЭ MET c c.C3029T в основной и дополнительной панелях была подтверждена АСЭ гена – среди 5 пар образцов с MET c.C3029T было обнаружено 2 случая ГК с АСЭ этого варианта в опухолевой ткани (3.3 \% случаев). Учитывая распространенность рассматриваемого варианта среди индивидов с другими типами опухолей и наличие данных о его функциональности в некоторых модельных системах, наличие герминативного варианта MET c.C3029T в 8 \% случаев ГК нашей выборки и аллельный дисбаланс в экспрессии этого варианта могут рассматриваться как потенциальный фактор инициации и/или прогрессии ГК. В дальнейшем планируется провести исследование влияния рассматриваемого варианта на биологические свойства клеток путем экзогенной экспрессии мутантного MET в клетках культур гепатомы человека HepG2 и Huh7. Кроме того, будет проверена чувствительность клеток культур гепатомы, экспрессирующих MET c.C3029T, к воздействию ингибитора рецепторов тирозинкиназ кабозантиниба, обладающего селективностью в отношении MET и недавно одобренного для терапии ГК, для выяснения целесообразности назначения такой терапии пациентам, являющимся носителями этого варианта.

\begin{figure}[H]
	\centering
	\includegraphics[width=\textwidth]{pics/new_scatter_stitch_ru.png}
	\caption{\textbf{Распределение капель по интенсивности флуоресценции в каналах FAM и HEX}. Верхняя и нижняя панели -- образцы ткани пациентов, гетеро- и гомозиготного по C3029T \textit{MET} соответственно. Серый кластер -- капли, свечение которых детектируется только в канале HEX, зеленый кластер -- капли, флуоресценция которых не наблюдается ни в одном из каналов, желтый кластер -- дважды положительный кластер. На графиках представлены распределения только для геномной ДНК.}
	\label{fig:example}
\end{figure}

\begin{figure}[H]
	\centering
	\includegraphics[width=\textwidth]{pics/hist_stitch_ru.png}
	\caption{\textbf{Распределение расстояний между кластерами для образцов с заменой C3029T \textit{MET}}. Расстояние 1-2 соответствует расстоянию между зеленым и желтым кластерами, расстояние 1-3 -- между зеленым и серым, расстояние 2-3 -- между желтым и серым (рисунок \ref{fig:example}).}
	\label{fig:hist}
\end{figure}

\newpage

\section{Выводы}

\begin{enumerate}
	\item При помощи описанного подхода, включающего набор инструментов GATK и алгоритм MBASED, в основной панели, состоящей из 45 парных образцов ГК и неопухолевой ткани печени, было обнаружено 13137 однонуклеотидных вариантов. Cреди 572 генов, включающих в себя онкогены, опухолевые супрессоры и печень-специфические гены, обнаружено 125 генов, экспрессия которых носит аллель-специфический характер;
	\item Медианные значения долей мажорных аллелей в экспрессии генов в неопухолевой и опухолевой тканях демонстрируют высокий уровень корреляции (коэффициент корреляции Спирмена=$0.85$, p-value=$7.6\times10^{-48}$). При этом на уровне отдельных генов и образцов возможны различия в уровне аллельного дисбаланса. Так, в 50 \% случаев в ГК наблюдалось преобладание экспрессии нереференсного аллеля в неопухолевой ткани, в 40 \% -- в опухоли. Таким образом, АСЭ может приводить к преимущественной экспрессии потенциально патогенных вариантов;
	\item В результате поиска потенциально патогенных вариантов в АСЭ генах выявлен и подтвержден методом цифровой капельной ПЦР встречающийся у 8 \% пациентов с ГК герминативный вариант MET c.C3029T. Частота АСЭ MET c.C3029T составляет 3.3 \% для опухолевой и прилежащей тканей печени, что, в совокупности с его предполагаемой функциональной значимостью, может способствовать инициации и/или прогрессии ГК. Кроме того, у пациентов с ГК в АСЭ генах были выявлены другие потенциально патогенные варианты, ассоциированные с развитием аутоиммунных заболеваний, хронической инфекции и воспалением.
\end{enumerate}